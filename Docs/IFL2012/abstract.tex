\begin{abstract}
Functional Reactive Programming is a promising \new{approach} 
\rn{systems -- seems weird to identify it with the implementations... model, approach?} 
for writing interactive and time-dependent programs. Signal-function FRP 
is variant of FRP 
% a subclass of these systems 
which provides advantages in modularity and correctness, but
has proven difficult to implement efficiently.

\textred{The abstraction of signal vectors provides the necessary type apparatus to
distinguish components of the input and output of signal functions which benefit
from a push-based implementation from those which benefit from a pull-based
implementation, and to combine both implementation strategies in a single system.}

\rn{Without understanding it a priori, the above para is hard to
  parse.  How about the following?}

\new{One important implementation trade-off is whether evaluation 
% of a composition 
of signal functions is {\em push-based} or {\em pull-based}.
The {\em signal-vector} technique addresses this by providing a type-level
  description of the inputs and outputs of signal functions, 
  distinguishing} components that benefit from a push-based
implementation from those that benefit from a pull-based
implementation.  This makes it possible to combine both implementation
strategies in a single system.

We describe a \new{novel} \rn{is it novel? how? need clear contributions}
 signal-function FRP system which provides push-based evaluation
for events, pull-based evaluation for signals, and a simple monadic evaluation
interface that permits the system to easily integrate with one or more
IO systems.
\end{abstract}
