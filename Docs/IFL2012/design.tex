\section{System Design and Interface}
\label{section:System_Design_and_Interface}

The purpose of FRP is to provide an efficient, declarative abstraction for creating
reactive programs. To this end, there are three goals that the TimeFlies
system is intended to meet:

An FRP system should permit efficient evaluation. Since FRP programs are
expected to  interact with an outside world in real time, efficiency cannot
simply be measured by the runtime of a program. Thus, when speaking of
efficiency, we are expressing a desire that the system utilize as few system
resources as possible for the task at hand, while responding as quickly as
possible to external inputs and producing output at a consistently high sample
rate.

A composable abstraction is one in which values in that abstraction may be
combined in such a way that reasoning about their combined actions involves
little more than reasoning about their individual actions. In a signal function
system, the only interaction between composed signal functions ought to be that
the output of one is the input of another. Composability permits a particularly
attractive form of software engineering in which successively larger systems are
created from by combining smaller systems, without having to reason about the 
implementation of the components of the systems being combined.

Further, an FRP program is not a closed system; it interacts with the outside
world. Since we cannot anticipate every possible form of input and output that
the system will be asked to interact with, we must interface with IO systems.
In particular, most libraries for user interaction (e.g. GUI and graphics
libraries such as GTK+ and GLUT) and most libraries for time-dependent IO
(e.g. audio and video systems) make use of the event-loop model. In this
model, event handlers are registered for various forms of input, and then a
loop is initiated which checks for, or blocks waiting for, input and then
actuates the appropriate handlers to compute on the input and actuate outputs.

We would like for the FRP system to be easy to integrate with such IO systems,
while being flexible enough to enable its use with other forms of IO systems,
such as systems which use blocking IO and threads, rather than event loops.

\subsection{Semantics}
\label{subsection:System_Design_and_Interface-Semantics}

A rigorous semantics of signal-function FRP with signal vectors remains
unattempted, though a categorical semantics for signal-function FRP using
{\em fan categories} has been explored~\cite{Jeffrey2012}. In both these
semantics and their implementation, events are represented (following
AFRP~\cite{Courtney2001-1} and Yampa~\cite{Nilsson2005}) as option-valued
signals. A push-based implementation of these semantics is proposed, but the
unit being pushed is a {\em segment} of signals, and not event occurrences. This
approach requires processing the whole signal function at each time step,
whether an event occurs or not, and checking for its occurrence at each step.

Another concern is that this approach limits the event rate to the sampling rate.
The rate of sampling should, at some level, not matter to the FRP system. Events
which occur between sampling intervals are never observed by the
system. Since the signal function is never evaluated except at the sampling
intervals, no events can otherwise be observed.

This raises another concern. Events are not instantaneous in this
formulation. If events are signals, the sampling instant must fall within
the interval where there is an event occurrence present for that event to be
observed. If events are instantaneous, the probability of observing an event 
occurrence is zero. So in current signal-function semantics, events occurrences
implicitly have some duration which contains the point of sampling.

Therefore, TimeFlies employs the N-Ary FRP type formulation to represent signals and
events as distinct entities in the inputs and outputs of signal functions. This means
we are now free to choose our representation of events, and to separate it from the
representation and evaluation of signals.

This freedom yields the additional ability to make events independent of the
sampling interval altogether. The semantics of event handling in TimeFlies is
that an event occurrence is responded to immediately, and does not wait for the
next sampling instant. This allows events to be instantaneous, and further,
allows multiple occurrences for an event within a single sampling interval.

There are two tradeoffs intrinsic to this approach. The first is that events are
only partially ordered temporally. There is no way to guarantee the order of
observation of event occurrences occurring in the same sampling interval.
Further, the precise time of an event occurrence cannot be observed, only the 
time that the signal function was last sampled prior to the occurrence.

In return for giving up total ordering and precise observation of the timing of 
events, we obtain the ability to employ push-based evaluation for event
occurrences, and the ability to non-deterministically merge event occurrences.
When events being input to a non-deterministic merge have simultaneous
occurrences, we arbitrarily select one to occur first. This does not violate any
guarantee about time values, since they will both have the same time value in
either case, and does not violate any guarantee about ordering, since no
guarantee of their order is given.

\subsection{Types}
\label{subsection:System_Design_and_Interface-Types}

In a strongly and statically typed functional language, types are a key part of
any interface. Types provide a mechanism for describing and ensuring properties
of the interface's components and about the systems created with these
components. 

In order to type signal functions, we must be able to describe their input and
output. In most signal function systems, a signal function takes exactly one
input signal and produces exactly one output signal. Multiple inputs or outputs
are handled by making the input or output a signal of tuples, and combinators
which combine or split the inputs or outputs of a signal assume this. Events are
represented at the type level as a particular type of signal, and at the value
level as an option: either an event occurrence or not.

This method of typing excludes push-based evaluation at the outset.
It is not possible to construct a ``partial tuple'' nor in general is it
possible to construct only part of any type of value. Push-based evaluation
depends on evaluating only that part of the system which is updated, which means
evaluating only that part of the input which is updated.

In order to permit the construction of partial inputs and outputs, we make use
of signal vectors. Signal vectors are uninhabited types which describe the input
and output of a signal function. Singleton vectors are parameterized over the
type carried by the signal or by event occurrences. The definition of the signal
vector type is shown in Figure~\ref{figure:signal_vector_types}. 

Having an uninhabited signal vector type allows us to construct representations
of inputs and outputs which are hidden from the user of the system, and are
designed for partial representations.

\begin{figure}
\begin{code}
data SVEmpty    -- An empty signal vector component,
                -- neither event nor signal
data SVSignal a -- A signal, carrying values of type a
data SVEvent a  -- An event, whose occurrences carry values of type a
data SVAppend svLeft svRight -- The combination of the signal vectors
                             -- svLeft and svRight
\end{code}
\hrule
\caption{Signal vector types.}
\label{figure:signal_vector_types}
\end{figure}

Signal functions with signal vectors as input and output types form a
Haskell {\tt GArrow}~\cite{Megacz2011}. Specifically, the signal function
type constructor (with the initialization parameter fixed) forms the arrow
type, the {\tt SVAppend} type constructor forms the product type, and the
{\tt SVEmpty} type constructor forms the unit type.
%
The representation of signal functions is discussed in
Section~\ref{subsection:Implementation-Signal_Functions}.
\begin{figure}
\begin{code}
-- Signal functions
-- init: The initialization type for 
--   the signal function, always NonInitialized
--   for exported signal functions
-- svIn: The input signal vector
-- svOut: The output signal vector
data SF init svIn svOut

data NonInitialized

type svIn :~> svOut = SF NonInitialized svIn svOut
type svLeft :^: svRight = SVAppend svLeft svRight
\end{code}
\hrule
\caption{Signal function types.}
\label{figure:signal_function_types}
\end{figure}

Our interface for evaluating signal functions takes the form of a monad
transformer~\cite{Jones1995}. This formulation permits us to provide specific
temporal actions to perform on a signal function in sequence, and in conjunction
with actions taken in Haskell's {\tt IO} monad, or another monad. Actions are
provided in this monad to push an event to the signal function's input, step
the signal function's time, or modify a signal on the signal functions input to
be observed at the next time step. Outputs are handled by a set of actuation
functions in the underlying monad, supplied by the user of the interface. This
set's type is parameterized over the signal function's output signal vector.

\subsection{Combinators}
\label{section:System_Design_and_Interface-Combinators}

TimeFlies includes basic signal functions and a library of signal function
combinators, grouped into lifting operations from pure functions to signal
functions, routing, reactivity, feedback, event processing, joining, and time
dependence.

These signal functions can be combined in serial (the output of one becomes
the input of the other) or in parallel (their input and output vectors are
combined). Signal functions are independently reactive or non-reactive, which
is the basis of their compositionality. Reactivity is introduced using the
{\tt switch} combinator, which takes an initial signal function, watches
one of its outputs for an event occurrence carrying a new signal function,
and replaces itself with this new signal function. Its type is:

\begin{code}
switch ::    (svIn :~> (svOut :^: SVEvent (svIn :~> svOut)))
          -> svIn :~> svOut
\end{code}

The full interface is described in the API documentation~\cite{TimefliesAPI}.

\subsection{Evaluation Interface}
\label{subsection:System_Design_and_Interface-Evaluation_Interface}

The evaluation interface provides a modified state monad which holds a signal
function, together with some additional information, as its state.
Rather than monadic instructions to put and get the state, the monad provides instructions
to trigger an input event, update an input signal, and trigger sampling of
signals in the signal function. Additional state includes the current set of
modifications to the input signals (since the last sample) and a set of
handlers which actuate effects based on output events or changes to the output
signal.

Sets which correspond to signal vectors are built with ``smart'' constructors.
For instance, to construct a set of handlers, individual handling functions are
lifted to handler sets with the {\tt signalHandler} and {\tt eventHandler}
functions, and then combined with each other and {\tt emptyHandler} leaves
using the {\tt combineHandlers} function.

Building the initial input sample is similar, but {\tt sampleEvt} leaves do
not carry a value.

In order to initialize the state, the user must supply a set of handlers, the
signal function to evaluate, and initial values for all of the signal inputs.

This state can then be passed to a monadic action which will supply input to
the signal function. Inputs are constructed using a simple interface with
functions to construct sample updates and event occurrences, and to specify
their place in the vector.

The {\tt SFEvalT} monad is actually a monad transformer, that is, it is
parameterized over an underlying monad whose actions may be lifted to
{\tt SFEvalT}. In the usual case, this will be the {\tt IO} monad.

{\tt SFEvalT} actions are constructed using combinators to push events,
update inputs, and step time, as well as actions lifted from the underlying
monad (used to obtain these inputs). 

{\tt SFEvalT} actions are run using a supplied function which takes a signal
function's evaluation state and an {\tt SFEvalT} action and provides an action
in its underlying monad. This action includes the actions to actuate the outputs,
and returns the new evaluation state, along with the return value of the
{\tt SFEvalT} here.

Using this interface, the state can be stored in a mutable reference, or in a
closure. Handlers can be registered with the IO interface which push events, and
the actuation function can be constructed from the primitives given by the IO
library for actuation.
