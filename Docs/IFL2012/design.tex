\section{System Design and Interface}
\label{section:System_Design_and_Interface}

The purpose of FRP is to provide an efficient, declarative abstraction for creating
reactive programs. To this end, there are three goals that the TimeFlies
system is intended to meet. \rn{What are the three!?  Enumerate.  Push
based, composable, IO-system agnostic?}

Efficient evaluation is the motivation for push-based evaluation of events.
Since FRP programs are expected to  interact with an outside world in real time,
efficiency cannot simply be measured by the runtime of a program. Thus, when speaking of efficiency,
we are expressing a desire that the system utilize as few system resources as possible
for the task at hand, while responding as quickly as possible to external inputs and
producing output at a consistently high sample rate.

\todo{At this point we still need a clear notion of what a pull-based
  evaluation of events would actually do.}

A composable abstraction is one in which values in that abstraction may be
combined in such a way that reasoning about their combined actions involves
little more than reasoning about their individual actions. In a signal function
system, the only interaction between composed signal functions ought to be that
the output of one is the input of another. Composability permits a particularly
attractive form of software engineering in which successively larger systems are
created from by combining smaller systems, without having to reason about the 
implementation of the components of the systems being combined.

\rn{Can the above be made into a slightly more technical claim?
  Composable abstractions enable composing $O(N)$ components while
  reasoning about the $O(N)$ local interactions, rather than the
  $O(N^2)$ global interactions between all components?  Second, I
  think not needing to know the implementation is separate from
  ``composability''; surely there is some name for that: ADTs,
  ``encapsulation'', information hiding, etc.}

% It is fine for a system to be composable with regards to itself, but
% an FRP system must
\new{Further, an FRP program is not a closed system; it}
interacts with the outside world. Since we cannot anticipate every
possible form of input and output that the system will be asked to interact
with, we must interface with Haskell's IO system. 
\rn{First mention of Haskell... need some transition?}
In particular, most libraries
for user interaction (e.g. GUI and graphics libraries such as GTK+ and GLUT) and
most libraries for time-dependent IO (e.g. audio and video systems) make use of
the event-loop abstraction. In this abstraction, event handlers are registered
\new{centrally}; 
\new{once initiated, the running event-loop detects events and invokes handlers; and
  handlers are responsible for rendering output.}
%% and then a command is issued to run a loop which detects events
%% and runs the handlers, and uses the results of the handlers to render the
%% appropriate output. 

\rn{Is the event loop really an {\em abstraction}?  What is
  abstracted?  Or is it just a ``model'' or an API?}

We would like for the FRP system to be easy to integrate with such IO systems,
while being flexible enough to enable its use with other forms of IO systems,
such as simple imperative systems, threaded systems, or network servers.
\rn{vague... not sure what these mean.  Threaded = multiple event
  loop?  Alternative to an event loop is... what?... blocking IO operations?}

\subsection{Semantics}
\label{subsection:System_Design_and_Interface-Semantics}

A rigorous 
%and formal elucidation of the 
semantics of signal-function FRP with signal
vectors remains unattempted, though a categorical semantics for signal-function
FRP using {\em fan categories} has been explored~\cite{Jeffrey2012}. In both
these semantics and their implementation, events are represented (following
AFRP~\cite{Courtney2001-1} and Yampa~\cite{Nilsson2005}) as option-valued
signals. A push-based implementation of these semantics is proposed, but the
unit being pushed is a {\em segment} of signals, and not event occurrences. This
approach requires processing the whole signal function at each time step,
whether an event occurs or not, and checking for its occurrence at each step.

\rn{Might want to make it clear -- the ``whole'' signal function means
  ``all signal functions in the program'' (i.e. which are composed
  to form the global signal function) right?}

Another concern is that this approach limits the event rate to the sampling rate.
The rate of sampling should, at some level, not matter to the FRP system. Events
which occur between sampling intervals are never observed by the
system. \rn{An FRP system has sampling intervals?  Where were we told
  this!?}  
Since the signal function is never evaluated except at the sampling intervals, no
events can otherwise be observed.

This \new{raises another} concern. Events are not instantaneous in this
formulation. If events are signals, the sampling instant must fall within
the interval where there is an event occurrence present for that event to be
observed. If events are instantaneous, the probability of observing an event 
occurrence is zero. \rn{So.... how do existing systems/semantics
  compensate for this?  They allow events to have temporal extent?}

Therefore, TimeFlies employs the N-Ary FRP type formulation to represent signals and
events as distinct entities in the inputs and outputs of signal functions. This means
we are now free to choose our representation of events, and to separate it from the
representation and evaluation of signals.
%
This freedom yields the additional ability to make events independent of the sampling
interval altogether. The semantics of event handling in TimeFlies is that an event occurrence
is responded to immediately, and does not wait for the next sampling
instant. 
\rn{Could you flesh out how this is a {\em semantics} issue and not
  merely a more efficient or accurate implementation?  Ah, that's
  answered in the next para, but might be good to use the word
  semantics somewhere and connect this explicitly.} 
This allows events to be instantaneous, and further, allows multiple
events to occur within a single sampling interval.
\rn{Clarification---even in traditional approaches multiple {\bf
    different} events could occur in one sampling round right?  It's
  just that {\em each} event [stream] could only have one occurrence per
  sampling interval?}


There are two tradeoffs intrinsic to this approach. The first is that events are only partially ordered
temporally. There is no way to guarantee the order of observation of event occurrences occurring in the
same sampling interval. Further, the precise time of an event occurrence cannot be observed, only the 
time of the last \rn{[signal?]} sample prior to the occurrence.
\rn{That last sentence had an ambiguous parse wherein the first time I read it I thought it was saying that you
couldn't tell the timestamp of event N only of the preceding event $N-1$, which of course makes no sense,
but it might be good to clarify what {\em sample} you're referring to.}

In return for giving up total ordering and precise observation of the timing of events, we obtain the
ability to employ push-based evaluation for event occurrences, and the ability to non-deterministically
merge event occurrences. When events being input to a non-deterministic merge have simultaneous occurrences,
we arbitrarily select one to occur first. This does not violate any guarantee about time values, since
they will both have the same time value in either case, and does not violate any guarantee about ordering,
since no guarantee of their order is given.
\rn{Nice.}

\subsection{Types}
\label{subsection:System_Design_and_Interface-Types}

In a strongly and statically typed functional language, types are a key part of
any interface. Types provide a mechanism for describing and ensuring properties
of the interface's components and about the systems created with these
components. 

In order to type signal functions, we must be able to describe their input and
output. In most signal function systems, a signal function takes exactly one
input signal and produces exactly one output signal. Multiple inputs or outputs are handled
by making the input or output a signal of tuples, and combinators which combine or split
the inputs or outputs of a signal assume this.  Events are represented at the
type level as a particular type of signal, and at the value level as an option:
either an event occurrence or not. 
\rn{That's a bit confusing, because of the issue of whether we're talking about {\em event occurrences} or
  event streams.  At runtime there is only a representation for the former, which is a simple {\tt Maybe},
  right?}

This method of typing excludes push-based evaluation at the outset.
It is not possible to construct a ``partial tuple'' nor in general is it
possible to construct only part of any type of value. Push-based evaluation
depends on evaluating only that part of the system which is updated, which means
evaluating only that part of the input which is updated.

In order to permit the construction of partial inputs and outputs, we make use
of signal vectors. Signal vectors are uninhabited types which describe the input
and output of a signal function. Singleton vectors are parameterized over the
type carried by the signal or by event occurrences. The definition of the signal
vector type is shown in Figure~\ref{figure:signal_vector_types}. 

Having an uninhabited signal vector type allows us to construct representations
of inputs and outputs which are hidden from the user of the system, and are
designed for partial representations.

\begin{figure}
\begin{code}
data SVEmpty    -- An empty signal vector component,
                -- neither event nor signal
data SVSignal a -- A signal, carrying values of type a
data SVEvent a  -- An event, whose occurrences carry values of type a
data SVAppend svLeft svRight -- The combination of the signal vectors
                             -- svLeft and svRight
\end{code}
\hrule
\caption{Signal vector types.}
\label{figure:signal_vector_types}
\end{figure}

Signal functions with signal vectors as input and output types form a
Haskell {\tt GArrow}~\cite{Megacz2011}. Specifically, the signal function
type constructor (with the initialization parameter fixed) forms the arrow
type, the {\tt SVAppend} type constructor forms the product type, and the
{\tt SVEmpty} type constructor forms the unit type.
%
The representation of signal functions is discussed in
Section~\ref{subsection:Implementation-Signal_Functions}.
\begin{figure}
\begin{code}
-- Signal functions
-- init: The initialization type for 
--   the signal function, always NonInitialized
--   for exported signal functions
-- svIn: The input signal vector
-- svOut: The output signal vector
data SF init svIn svOut

data NonInitialized

type svIn :~> svOut = SF NonInitialized svIn svOut
type svLeft :^: svRight = SVAppend svLeft svRight
\end{code}
\hrule
\caption{Signal function types.}
\label{figure:signal_function_types}
\end{figure}

Our interface for evaluating signal functions takes the form of a monad
transformer~\cite{Jones1995}. This formulation permits us to provide specific
temporal actions to perform on a signal function in sequence, and in conjunction
with actions taken in Haskell's {\tt IO} monad, or another monad. Actions are
provided in this monad to push an event to the signal function's input, step
the signal function's time, or modify a signal on the signal functions input to
be observed at the next time step. Outputs are handled by a vector \rn{vector? in the same sense as a
  signal vector?} of actuation
functions in the underlying monad, supplied by the user of the interface. This
vector is structured by parameterizing it over the signal function's output
signal vector. \rn{huh? more concrete way to say it?}

\subsection{Combinators}
\label{section:System_Design_and_Interface-Combinators}

Signal functions are constructed from combinators, which are primitive signal
functions and operations to combine these primitives. These combinators are
grouped as basic signal functions, lifting operations for pure functions,
routing, reactivity, feedback, event processing, joining, and time dependence.

The basic signal functions
provide very simple operations. The {\tt identity} signal function, as expected,
simply copies its input to its output. The {\tt constant} signal function
produces the provided value as a signal at all times. The {\tt never} signal
function has an event output which never produces occurrences. The {\tt asap}
function produces an event occurrence with the given value at the first time
step after it is switched into the network. The {\tt after} function waits for
the specified amount of time before producing an event occurrence.

Two combinators are provided to lift pure functions to signal functions.
The {\tt pureSignalTransformer} combinator applies the pure function pointwise
to a signal. The {\tt pureEventTransformer} combinator applies the function to
the value carried by each occurrence of an input event.

The routing combinators are used to combine signal functions, and to re-arrange
and discard components of signal vectors in order to connect signal functions.

Only those routing combinators which modify or combine signal functions
({\tt (>>>)}, {\tt first}, {\tt second}) are reactive (may replace themselves in
response to events, and then only if they inherit their reactivity from the
signal function(s) which are inputs to combinator. The rest do not
react to or modify the input in any way, except to re-arrange it, copy it, or
discard it altogether.

Reactivity is introduced by means of the {\tt switch} combinator.
The design of this combinator
allows modular packaging of reactivity. A signal function can determine
autonomously when to replace itself, based only on its input and state,
by emitting an event occurrence carrying its replacement. The combinator
consumes and hides the event carrying the replacement signal function, 
so the reactivity is not exposed by the resulting reactive signal function.

It is often useful for a signal function to receive a portion of its
own output as input. This is especially useful when we have two
signal functions which we would like to mutually interact. We cannot
just serially compose them, we must also bring the output of the second
back around to the first. Many signal-processing algorithms also depend
on feedback. This ability is provided by the {\tt loop} operator. 

This combinator provides decoupling for signals
(the input signal is the output signal at the previous time-step)
but not events (event occurrences are supplied to the combinator immediately).
This means that the programmer has the responsibility to ensure that feedback
does not generate an infinite sequence of events in a single time-step.

The {\tt filter} and {\tt filterList} combinators are stateless signal functions
which apply a function to each event occurrence. The {\tt filter} combinator
expects an option-typed output, while the {\tt list} combinator expects a list
output. If the function yields a non-empty output, one or more output occurrences
are produced with the elements of that output. If the output is empty, there
are no output occurrences.

The {\tt accumulate} and {\tt accumulateList} functions are similar, except that
they maintain a state value which is replaced after each input event occurrence
by an additional output from the function, and supplied to the function in
addition to the value of each event occurrence.

The joining combinators provide the ability to combine two event
streams, two signals, or a signal and an event stream.

The {\tt union} combinator is a non-deterministic merge of two event streams.
Simultaneity of events is only observable up to the sampling rate, but events
can occur in between sampling intervals. Thus, if two events occur
simultaneously, one occurrence is arbitrarily selected to be first in the output
stream.

The {\tt combineSignals} combinator applies a binary function pointwise to two
signals, and produces the result of this application as a third signal. The
combining function is necessary because we will have two input samples at each
time step, and must produce exactly one output sample.

The {\tt capture} combinator adds the last-sampled value of a signal at the time
of an event occurrence to that event occurrence.

These three combinators together provide the ability to combine elements of
signal vectors. By combining these combinators, along with the {\tt cancelLeft}
and {\tt cancelRight} routing combinators, arbitrary signal vectors can be
reduced.

A set of combinators are provided for making use of time-dependence in a signal
function. These combinators allow the modification of signals and events with
respect to time, and the observation of the current time value.

The simplest time-dependent combinator is {\tt time}, which simply outputs
the time since it began to be evaluated. This does not necessarily correspond to
the time since the global signal function began to be evaluated, since the
signal function in which the {\tt time} combinator is used may have been
introduced through {\tt switch}.

The {\tt delay} signal function allows events to be delayed. An initial delay
time is given, and event occurrences on the right input can carry a new delay
time. Event occurrences on the left input are stored and output when their delay
time has passed. Changing the delay time does not affect occurrences already
waiting.

The {\tt integrate} combinator outputs the rectangle-rule approximation of the
integral of its input signal with respect to time.

\subsection{Evaluation Interface}
\label{subsection:System_Design_and_Interface-Evaluation_Interface}

The evaluation interface provides a modified state monad which holds a signal
function, together with some additional information, as its state.
Rather than monadic instructions to put and get the state, the monad provides instructions
to trigger an input event, update an input signal, and trigger sampling of
signals in the signal function. Additional state includes the current set of
modifications to the input signals (since the last sample) and a set of
handlers which actuate effects based on output events or changes to the output
signal.

Sets which correspond to signal vectors are built with ``smart'' constructors.
For instance, to construct a set of handlers, individual handling functions are
lifted to handler sets with the {\tt signalHandler} and {\tt eventHandler}
functions, and then combined with each other and {\tt emptyHandler} leaves
using the {\tt combineHandlers} function.

Building the initial input sample is similar, but {\tt sampleEvt} leaves do
not carry a value.

In order to initialize the state, the user must supply a set of handlers, the
signal function to evaluate, and initial values for all of the signal inputs.

This state can then be passed to a monadic action which will supply input to
the signal function. Inputs are constructed using a simple interface with
functions to construct sample updates and event occurrences, and to specify
their place in the vector.

The {\tt SFEvalT} monad is actually a monad transformer, that is, it is
parameterized over an underlying monad whose actions may be lifted to
{\tt SFEvalT}. In the usual case, this will be the {\tt IO} monad.

{\tt SFEvalT} actions are constructed using combinators to push events,
update inputs, and step time, as well as actions lifted from the underlying
monad (used to obtain these inputs). An action in the underlying monad
which produces the result and a new state is obtained with the {\tt runSFEvalT}
function.
