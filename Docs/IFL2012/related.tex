\section{Related Work}
\label{section:Related_Work}

Signal Function FRP was introduced as a model for Graphical User Interfaces~\cite{Courtney2001-1}.
The system was originally termed ``AFRP'' (Arrowized FRP). Yampa is a rewrite of
AFRP where signal functions apply a number of ad-hoc optimizatons to themselves
as they evolve. Yampa demonstrated a modest performance improvement
over AFRP~\cite{Nilsson2005}.

Reactive is a classic FRP system which implements push-based evaluation for events
by transforming behaviors to ``reactive~normal~form,'' where a behavior
is a non-reactive behavior running inside a switch, whose event stream carries
behaviors in reactive normal form. The system is evaluated by forking a Haskell
thread to repeatedly sample the non-reactive behavior, and then blocking on the
evaluation of the first occurrence in the event stream. When this occurrence
is yielded, the evaluation thread for the behavior is killed and a new
thread forked to evaluate the new behavior~\cite{Elliott2009}.

An alternate push-based system, based on {\em signal segments}, was presented in
work on the Curry-Howard correspondence between a form of temporal logic (LTL)
and FRP~\cite{Jeffrey2012}. However, this system continues to represent events
as option-valued signals. It therefore cannot be push-based in its response to
individual events, and cannot be push-pull, evaluating signals by pull-based
evaluation and events by push-based evaluation. It is push-based not in the
sense of evaluating events only when they occur, but rather in that it provides
a means to push signal segments (samples of a signal over intervals of time) to
the system and wait on its response.


