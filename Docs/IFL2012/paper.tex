\documentclass[draft]{llncs}

\usepackage{fancyvrb}                                       % For code environment
\DefineVerbatimEnvironment{code}{Verbatim}{fontsize=\small} %

\title{Push-Pull Signal-Function Functional Reactive Programming}
\titlerunning{Push-Pull Signal-Function FRP}

\author{Edward Amsden}
\institute{Rochester Institute of Technology\\\email{eca7215@cs.rit.edu}}

\begin{document}
\maketitle

\begin{abstract}
Functional Reactive Programming is a promising class of systems for writing
interactive and time-dependent programs. Signal-function FRP is a subclass of
these systems which provides advantages of modularity and correctness, but
has proven difficult to efficiently implement.

The abstraction of signal vectors provides the necessary type apparatus to
distinguish components of the input and output of signal functions which benefit
from a push-based implementation from those which benefit from a pull-based
implementation, and to combine both implementation strategies in a single system.

We describe a signal-function FRP system which provides push-based evaluation
for events, pull-based evaluation for signals, and a simple monadic evaluation
interface which permits the system to be easily integrated with one or more
IO systems.
\end{abstract}

\section{Introduction}
\label{section:Introduction}

Functional Reactive Programming (FRP) is a class of systems for describing
reactive programs. Reactive programs are programs which, rather than taking
a single input and producing a single output, must accept multiple inputs and 
alter temporal behavior, including the production of multiple outputs, based
on these inputs.

An FRP system will provide a means of manipulating {\em behaviors} and
{\em events}. Behaviors are often referred to as {\em signals} in FRP literature,
but the definition is the same. A behavior or signal is, semantically, a
function from time to a value. An event is a discrete, possibly infinite, and 
time-ordered sequence of occurrences, which are times paired with values.

FRP systems can generally be categorized as ``classic FRP,'' 
which corresponds to the originally described FRP system in that behaviors
and events are manipulated directly and are first-class values in the FRP
system, or ``signal-function FRP,'' in which behaviors (generally termed
signals in this approach) and events are not first-class values, but signal
functions are first class values. Signal functions are time-dependent and
reactive transformers of signals, events, or combinations of signals and events.

FRP combines behaviors and events through the use of {\em switching}, in which
a behavior (in classic FRP) or a signal function (in signal-function FRP) is
replaced by a new behavior or signal function carried by an event occurrence.

Classic FRP was first described as an system for interactive animations~\cite{Elliott1997}.
Recent work on classic FRP has focused on efficient implementation. One approach to
efficiency is to separate the evaluation of behaviors and events, since suitable 
strategies give best performance in each case. Push-based evaluation evaluates a
system only when input is available, and is thus suitable for discrete inputs
such as events. Pull-based evaluation evaluates the system as quickly as
possible, polling for input, and is preferable for behaviors and signals.
The initial implementations of FRP made use of pull-based evaluation for both
behaviors and events. Reactive~\cite{Elliott2009}, as well as more recent systems such as
``reactive-banana''~\cite{Apfelmus}, make use of push-based evaluation for
events and pull-based evaluation for behaviors. This is known as ``push-pull''
evaluation.

All implementations of signal-function FRP to date~\cite{Courtney2001-1,Nilsson2002,Nilsson2005,Sculthorpe2011}
have used pull-based evaluation for both signals and events. This is due to
the ease of implementation of pull-based evaluation, and the types used for
signal functions which do not permit distinguishing signals and events, or
constructing only part of the input (for instance, one event occurrence.)

A recent extension of signal-function FRP called N-Ary FRP~\cite{Sculthorpe2011}
describes a method of typing signal functions which, as we will show, enables
the push-based evaluation of events in a signal-function FRP system. The notion
of signal vectors allows the representation of signal function inputs and
outputs as combinations of signals and events, rather than a single signal which
may contain multiple values, including option values for events. Signal vectors
are uninhabited types, which can be used to type partial or full representations
of the signal function inputs and outputs.

We present TimeFlies,\footnote{The sentence ``Time flies like an arrow.'' is a 
favorite quotation of one of the author's philosophy instructors, used to
demonstrate the ambiguity of language. The origin of the quotation is unknown.}
a push-pull signal-function FRP system. We hope to demonstrate the feasibility
of such an approach to FRP, and provide a basis for further research into
efficient implementation of signal-function FRP. We also describe a powerful
evaluation interface for TimeFlies, which permits us to use TimeFlies to
describe applications which make use of multiple and differing IO libraries.

Section~\ref{section:System_Design} describes design choices for the system,
and provides an overview of the interface. Section~\ref{section:Implementation}
describes how the system is implemented, and how the separation of evaluation
between events and signals is achieved. Section~\ref{section:Discussion} is a
discussion of the usefulness of our implementation. 
Section~\ref{section:Ongoing_and_Further_Work} describes the current and future
work on this system. Section~\ref{section:Related_Work} gives an overview of
related efforts. Section~\ref{section:Conclusion} concludes.

\section{System Design}
\label{section:System_Design}

Our goal is to produce a composable and efficient FRP system. Signal function
FRP has an advantage in terms of composability, because it permits the
construction of self-contained objects through both input and output composition,
rather than purely through output composition as in classic FRP. Signal-function
FRP also avoids problematic properties common to classic FRP systems such as a
large class of time and space leaks~\cite{Liu2007}.

Efficient implementations of signal-function FRP have been approached through
runtime optimization~\cite{Nilsson2005}, but all implementations have been
pull-based for both signals and events. A truly efficient implementation will
likely combine run-time optimization with push-based evaluation for events.

The concept of N-Ary FRP was to encode additional safety properties into the
types of a signal-function FRP system~\cite{Sculthorpe2011}. This system
introduced the concept of {\em signal vectors} as input and output types for
signal functions. Signal vectors are combinations of signals and events. In
N-Ary FRP, signal functions are represented at the type level, and type-level
functions (type families in Haskell) are used to specify the representations
of signal functions.

In our system, we construct representations of signal vectors using
Generalized Algebraic Datatypes~\cite{Cheney2003,Xi2003}. The use of GADTs
enables us to use signal vectors to instantiate type parameters in the types
of signal vector representations, as well as in the types of signal functions.
GADTs are available as an extension in the Glasgow Haskell Compiler~\cite{PeytonJones2006}.

This approach permits us to construct partial inputs and outputs for signal
functions. For instance, we can construct a representation for a single event
in a signal vector, and another representation of updated values for a subset of
the signals in the signal vector, and yet another carrying values for all
signals in a signal vector. The signal vector types are shown in
Figure~\ref{figure:signal_vectors}.

\begin{figure}[t]
\begin{code}
data SVEmpty
data SVSignal a
data SVEvent a
data SVAppend svLeft svRight
\end{code}
\hrule
\caption{Signal vectors.}
\label{figure:signal_vectors}
\end{figure}

With the ability to construct partial representations of signal vectors, we can
represent signal functions with a datatype carrying multiple functions, one for
each type of input. This allows us to separate the pull-based processing of
signals from the push-based reaction to events.

The exposed interface is a set of combinators for constructing signal functions,
as well as combinators for describing the evaluation of a signal function. The
interface is shown in Fig.~\ref{figure:signal_function_interface}.

\begin{figure}
\begin{code}

-- Signal Functions
type :~> svIn svOut

-- Infix type alias for SVAppend
type :^: svLeft svRight = SVAppend svLeft svRight

-- Basic signal functions
identity :: sv :~> sv
constant :: a -> SVEmpty :~> SVSignal a
never    :: SVEmpty :~> SVEvent a
asap     :: a -> SVEmpty :~> SVEvent a
after    :: Double -> a -> SVEmpty :~> SVEvent a

-- Lifting pure functions
pureSignalTransformer :: (a -> b) -> SVSignal a :~> SVSignal a
pureEventTransformer  :: (a -> b) -> SVEvent a :~> SVEvent a

-- Composition and routing
(>>>)           :: (svIn :~> svMiddle) -> (svMiddle :~> svOut) 
                   -> (svIn :~> svOut)
first           :: (svIn :~> svOut) -> (svIn :^: sv) :~> (svOut :^: sv)
second          :: (svIn :~> svOut) -> (sv :^: svIn) :~> (sv :^: svOut) 
swap            :: (svLeft :^: svRight) :~> (svRight :^: svLeft)
copy            :: sv :~> (sv :^: sv)
ignore          :: sv :~> SVEmpty
cancelLeft      :: (SVEmpty :^: sv) :~> sv
cancelRight     :: (sv :^: SVEmpty) :~> sv
uncancelLeft    :: sv :~> (SVEmpty :^: sv)
uncancelRight   :: sv :~> (sv :^: SVEmpty)
associate       :: ((sv1 :^: sv2) :^: sv3) :~> (sv1 :^: (sv2 :^: sv3)) 
unassociate     :: (sv1 :^: (sv2 :^: sv3)) :~> ((sv1 :^: sv2) :^: sv3)

-- Reactivity
switch :: (svIn :~> (svOut :^: SVEvent (svIn :~> svOut))) 
          -> svIn :~> svOut

-- Feedback
loop :: ((svIn :^: svLoop) :~> (svOut :^: svLoop)) -> svIn :~> svOut

-- Time dependence
time  :: SVEmpty :~> Double
delay :: Double -> (SVEvent a :^: SVEvent Double) :~> SVEvent a
class TimeIntegrate

-- Joining
union          :: (SVEvent a :^: SVEvent a) :~> SVEvent a
combineSignals :: (a -> b -> c) 
                  -> (SVSignal a :^: SVSignal b) :~> SVSignal c
capture        :: (SVSignal a :^: SVEvent b) :~> SVEvent a

-- Events
filter         :: (a -> Maybe b) -> SVEvent a :~> SVEvent b
\end{code}
\hrule
\caption{Signal function interface.}
\label{figure:signal_function_interface}
\end{figure}

Signal functions are produced by combining primitive signal functions using
the {\tt >>>} (sequential composition), {\tt first}, and {\tt second} routing
combinators. Other routing signal functions are provided, but they are intended
to be combined with, rather than to modify, other signal functions.

\subsection{Evaluation Interface}
The evaluation interface provides a monad~\cite{PeytonJones1993,PeytonJones2001}
for specifying input to signal functions and handling their output. This
interface also allows input actions to be separated from output actions, and
from each other, so that a signal function may receive input from multiple IO
systems and have its output handled by multiple output systems.

The evaluation interface is a monad transformer~\cite{Liang1995}. This allows
a signal function evaluation to be constructed from primitive evaluation actions
(event pushing, input signal updating, and sampling) as well as actions from the
underlying monad. These actions can then be run in the underlying monad
to actuate the signal function. Because the actuation takes a signal function's
state as input and produces a new state as output, in addition to the monadic
side effects of handling the signal function's output, different evaluation
actions may be taken on the same signal function from distinct locations within
an application's code. This allows the evaluation interface to be easily
integrated with event-loop style systems as well as traditional imperative
IO systems.

The evaluation interface consists of the monad transformer type, functions for
constructing event inputs, initial input signal samples, and input signal
updates, functions for constructing the vector of output handlers for a signal
function, and evaluation actions. The full interface is shown in Fig.~\ref{figure:evaluation_interface}.

\begin{figure}
\begin{code}
-- Input helpers
class SVRoutable -- Instances: SVSignalUpdate,
                 -- SVEventOccurrence
svLeft :: (SVRoutable r) => 
          r svLeft -> r (SVAppend svLeft svRight)
svRight :: (SVRoutable r) =>
           r svRight -> r (SVAppend svLeft svRight)

-- Signal inputs
data SVSignalUpdate
data SVSample
sample :: a -> SVSample (SVSignal a)
sampleEvt :: SVSample (SVEvent a)
sampleNothing :: SVSample SVEmpty
combineSamples :: SVSample svLeft -> svSample svRight 
                  -> svSample (svLeft :^: svRight)
svSig :: a -> SVSignalUpdate (SVSignal a)

-- Event inputs
data SVEventInput
svOcc :: a -> SVEventInput (SVEvent a)
-- Actuation
update :: (Monad m) => SVSignalUpdate svIn 
           -> SFEvalT svIn svOut m ()
push   :: (Monad m) => SVEventInput svIn
          -> SFEvalT svIn svOut m ()
sample :: SVEvalT svIn svOut m ()
-- Running
data SVHandler
emptyHandler :: SVHandler m SVEmpty
eventHandler :: (a -> m ()) 
                -> SVHandler m (SVEvent a)
signalHandler :: (a -> m ()) 
                 -> SVHandler m (SVSignal a)
combineHandlers :: SVHandler m svLeft
                   -> SVHandler m svRight
                   -> SVHandler m (svLeft :^: svRight)
initSFEvalT :: SVHandler m svOut -> SVSample svIn 
               -> Double ->(svIn :~> svOut)
               -> SFEvalState m svIn svOut
runSFEvalT :: SFEvalT svIn svOut m a
              -> SFEvalState m svIn svOut
              -> m a
\end{code}
\hrule
\caption{Evaluation interface.}
\label{figure:evaluation_interface}
\end{figure}

\section{Implementation}
\label{section:Implementation}

We now turn our attention to the implementation of the signal function
system. We will discuss representations of inputs, outputs, and signal functions,
as well as the implementations of specific signal function combinators.

\subsection{Input and Output Representations}
\label{subsection:Input_and_Output_Representations}
As discussed in Section~\ref{section:System_Design}, our implementation requires
representations of the inputs and outputs of signal functions. Since signal
functions are typed using signal vectors, the representation types must be
parameterized over signal vectors. But signal vectors are uninhabited types, and
in ordinary Haskell we cannot declare a data constructor which uses a component
of a type parameter. Further, we cannot restrict which types may instantiate a
type parameter for a particular data constructor.

GADTs lift these restrictions, permitting type constraints to be applied
to individual data constructors. In the declaration of a GADT, a type signature
is provided for each data constructor, including the types of the constructor's
parameters. When a GADT is pattern-matched, the types of the parameters are
inferred using the constraints given in this signature. This is called type
refinement.

Using GADTs, we can construct representations of signal vectors to use in our
implementation. 

When constructing a representation, we consider how to represent each
component of a signal vector, whether it is a signal element, an empty
element, or an append node. For instance, in order to represent a signal
sample for all signals in an element, we construct the GADT shown in Figure~\ref{figure:signal_sample_representation}.

\begin{figure}
\begin{code}
data SVSample sv where
  SVSample        ::    a -> SVSample (SVSignal a)
  SVSampleEvt     ::    SVSample (SVEvent a)
  SVSampleNothing ::    SVSample SVEmpty
  SVSampleBoth    ::    SVSample svLeft
                     -> SVSample svRight
                     -> SVSample (svLeft :^: svRight)
\end{code}
\hrule
\caption{A representation which carries a value for each signal in a signal vector.}
\label{figure:signal_sample_representation}
\end{figure}

\subsection{Signal Function Representations}
\label{subsection:Signal_Function_Representations}
With representations for signal function inputs and outputs in place, we turn
to the representation of signal functions. A signal function must be able to
respond to time increments and new signal samples, as well as event occurrences.
Since we wish to be able to ``push'' event occurrences independently of
sampling, events are handled using the last-updated time and signal sample.

When a signal function is asked to respond to either of these types of inputs,
it must produce an output consisting of zero or more event occurrences, a new
signal function with the same type (to enable reactivity and state), and, if
it is responding to a time and sample update, a delta which represents updates
to the sample of its output signals.

Finally, there is a special case for a signal function at time 0. It must be
provided with an initial input sample before it can respond to incremental time
and sample updates or to event occurrences.

Signal functions are represented as a datatype with two constructors. The first
constructor wraps a function from an input sample to an output sample and an
initialized signal function. The second wraps two functions. The first accepts
a time delta and a signal delta, and produces a signal delta, a list of event
occurrences, and a new signal function as the output. The second accepts an
event occurrence and produces a list of event occurrences and a new signal
function as the output. The signal function representation is shown in Figure~\ref{figure:signal_function_representation}.

\begin{figure}[t]
\begin{code}
data Initialized
data NonInitialized

type :~> svIn svOut = SF NonInitialized svIn svOut

data SF init svIn svOut where
  SF :: (SVSample svIn -> (SVSample svOut, SF Initialized svIn svOut))
        -> SF NonInitialized svIn svOut
  SFInit :: (Double -> SVDelta svIn 
             -> (SVDelta svOut, [SVOccurrence svOut],
                 SF Initialized svIn svOut))
            -> (SVOccurrence svIn -> ([SVOccurrence svOut],
                                      SF Initialized svIn svOut))
            -> SF Initialized svIn svOut
\end{code}
\hrule
\caption{Signal function representation}
\label{figure:signal_function_representation}
\end{figure}

This representation allows a signal function to respond to events and time
updates separately, and does not enforce the representation of events during
time updates, as previous signal function systems do.

\subsection{Signal Function Implementations}
\label{subsection:Signal_Function_Implementations}

Now that we have established representations for signal functions and their
inputs and outputs, we show several examples of how signal function combinators
are implemented.

The {\tt identity} signal function's initialization function takes a sample,
and returns that sample as its output sample. The initialized function returned
is the {\tt identityInit} function, which is not exported by the module.
The time and sample function ignores the provided time delta, produces the
input signal delta as the output signal delta, and returns {\tt identityInit} as
the new signal function. The event function takes an event occurrence and
returns a singleton list containing that occurrence, along with the
{\tt identityInit} function as the replacement signal function. The
implementation is shown in Figure~\ref{figure:identity_signal_function_implementation}.

\begin{figure}[t]
\begin{code}
identity :: sv :~> sv
identity = SF (\sample -> (sample, identityInit))

identityInit :: SF Initialized sv sv
identityInit = SFInit (\_ delta -> (delta, [], identityInit))
                      (\occ -> ([occ], identityInit))
\end{code}
\hrule
\caption{{\tt identity} signal function implementation.}
\label{figure:identity_signal_function_implementation}
\end{figure}

The {\tt constant} signal function is initialized with a value, which it then
produces as its output forever. Thus, the initialized version of {\tt constant}
never produces any output other than itself as a replacement
signal function, and an empty signal delta. The implementation of {\tt constant}
is shown in Fig.~\ref{figure:constant_signal_function_implementation}.

\begin{figure}
\begin{code}
constant :: a -> SVEmpty :~> SVSignal a
constant x = SF (\_ -> (constantInit, sample x))

constantInit :: SF Initialized SVEmpty (SVSignal a)
constantInit = SFInit (\_ _ -> (deltaNothing, [], constantInit))
                      (\_ -> ([], constantInit))
\end{code}
\hrule
\caption{{\tt constant} signal function implementation.}
\label{figure:constant_signal_function_implementation}
\end{figure}

The {\tt asap} signal function implementation requires a parameter for the
initialized version of the signal function, to specify the value of the event
occurrence. It replaces itself with the initialized version of the {\tt never}
signal function after the first time step. The implementation is shown in
Fig.~\ref{figure:asap_signal_function_implementation}.

\begin{figure}
\begin{code}
asap :: a -> SVEmpty :~> SVEvent a
asap x = SF (\_ -> (sampleEvt, asapInit x))

asapInit :: a -> SF Initialized SVEmpty (SVEvent a)
asapInit x = SFInit (\_ _ -> (deltaNothing, [occurrence x], neverInit))
                    (\_ -> (never))
\end{code}
\hrule
\caption{{\tt asap} signal function implementation.}
\label{figure:asap_signal_function_implementation}
\end{figure}

The {\tt filter} function accepts events, applies a {\tt Maybe} predicate to
their values, and produces the value produced by the predicate as an event
occurrence, or no occurrence if the value is {\tt Nothing}. It is implemented
by having the event occurrence response function apply the {\tt maybe} function
from the Haskell prelude to the value returned by the predicate, as shown in 
Fig.~\ref{figure:filter_signal_function_implementation}

\begin{figure}[t]
\begin{code}
filter :: (a -> Maybe b) -> SVEvent a :~> SVEvent b
filter p = SF (\_ -> (sampleEvt, filterInit p))

filterInit :: (a -> Maybe b) -> SF Initialized (SVEvent a) (SVEvent b)
filterInit p = SFInit (\_ _ -> (deltaNothing, [], filterInit p))
                      (\evtOcc -> (maybe [] ((:[]) . occurrence) $
                                   fromOccurrence evtOcc,
                                   filterInit p))
\end{code}
\hrule
\caption{{\tt filter} signal function implementation.}
\label{figure:filter_signal_function_implementation}
\end{figure}

The {\tt associate} signal function is a routing signal function. It transforms
a signal vector which is left-associated at the top level to one that is
right-associated at the top level. Its implementation is representative of all
of the routing functions, and is shown in Fig.~\ref{figure:associate_signal_function_implementation}.

\begin{figure}[t]
\begin{code}
associate :: SVAppend (SVAppend sv1 sv2) sv3
             :~> SVAppend sv1 (SVAppend sv2 sv3)
associate = 
  SF (\sigSample -> let (sigSampleLeft, sigSampleRight) = 
                          splitSample sigSample
                        (sigSampleLeftLeft, sigSampleLeftRight) =
                          splitSample sigSampleLeft
                    in (combineSamples sigSampleLeftLeft 
                        (combineSamples sigSampleLeftRight sigSampleRight),
                        associateInit))

associateInit :: SF Initialized (SVAppend (SVAppend sv1 sv2) sv3)
                                (SVAppend sv1 (SVAppend sv2 sv3))
associateInit = SFInit (\_ sigDelta -> let (sigDeltaLeft, sigDeltaRight) =
                                             splitDelta sigDelta
                                           (sigDeltaLeftLeft,
                                            sigDeltaLeftRight) =
                                             splitDelta sigDeltaLeft
                                       in (combineDeltas sigDeltaLeftLeft
                                           (combineDeltas sigDeltaLeftRight
                                                          sigDeltaRight),
                                           [], associateInit)) 
                       (\evtOcc -> (case chooseOccurrence evtOcc of
                                      Left leftOcc -> 
                                        case chooseOccurrence leftOcc of
                                          Left leftLeftOcc ->
                                            [occLeft leftLeftOcc]
                                          Right leftRightOcc ->
                                            [occRight $ occLeft leftRightOcc]
                                      Right rightOcc ->
                                        [occRight $ occRight $ rightOcc],
                                    associateInit))
\end{code}
\hrule
\caption{{\tt associate} signal function implementation.}
\label{figure:associate_signal_function_implementation}
\end{figure}

The {\tt switch} function is the basic combinator used to introduce reactivity.
It is given a signal function whose output is the append of two signal vectors.
The left side of the signal vector is passed on as the output of the reactive
signal function, and the right side is an event carrying signal functions. When
an event occurrence is present on the right-side event output, the signal
function carried by this occurrence replaces the signal function constructed
by switch. Due to space concerns, the implementation is elided, but a brief
description will suffice. The signal function stores the input sample
provided during initialization, and updates it with deltas. When an event
occurrence carrying a signal function is produced by either the event or 
time function of the wrapped signal function, the stored signal sample is used
to initialize the new signal function. If this occurs while handling an event
input, the sample output by the new signal function's initialization is stored
and is combined with its first output delta to produce the output delta at the
next time step.

The {\tt loop} function allows a signal function to see a component of its own
output as input. This is primarily useful when a signal function has components
which mutually depend on each others outputs, such as in physics simulations
or games. Care must be taken that the feedback output is not immediately
dependent on the feedback input, or sampling the signal function will not
terminate. Loop is implemented by generating a recursive list (which will
be infinite if the feedback is not decoupled) of event inputs, and by splitting
the output signal delta and using the right side output delta as the right
side input delta within a recursive let-binding. (The let-binding in Haskell
always admits recursive bindings.) If the system is not completely decoupled,
this will result in non-termination during an evaluation step. Decoupling can
be achieved using {\tt delay} primitive.

\subsection{Evaluation Interface Implementation}
\label{subsection:Evaluation_Interface_Implementation}

The evaluation interface must maintain state including the current signal
function, the current time (to produce time deltas), updates to the input
sample which have yet to be sampled, and the output handlers.

The output handlers are stored in a structure similar to that for a signal
sample. The difference is that both signal and event leaves contain values, and
these values are functions from the leaf type to another type. The handler
datatype is shown in Fig.~\ref{figure:handler_datatype}.

\begin{figure}[t]
\begin{code}
data SVHandler out sv where
  SVHandlerEmpty  :: SVHandler out SVEmpty
  SVHandlerSignal :: (a -> out) -> SVHandler out (SVSignal a)
  SVHandlerEvent  :: (a -> out) -> SVHandler out (SVEvent a)
  SVHandlerBoth   :: SVHandler out svLeft -> SVHandler out svRight 
                     -> SVHandler out (SVAppend svLeft svRight)
\end{code}
\hrule
\caption{Handler datatype.}
\label{figure:handler_datatype}
\end{figure}

We now need a datatype to hold the various components of the evaluation state.
This type is shown in Fig~\ref{figure:evaluation_state_datatype}.

\begin{figure}[t]
\begin{code}
data SFEvalState m svIn svOut 
  = SFEvalState { 
                  esSF :: SF Initialized svIn svOut,
                  esOutputHandlers :: SVHandler (m ()) svOut,
                  esLastTime :: Double,
                  esDelta :: SVDelta svIn
                }

\end{code}
\hrule
\caption{Evaluation state datatype.}
\label{figure:evaluation_state_datatype}
\end{figure}

The evaluation interface itself is a monad transformer, which we implement
as a Haskell newtype wrapping the {\tt StateT} monad transformer. We do not
export the raw {\tt put} and {\tt get} actions of the state monad, but instead
implement the push, update, and sampling operations for signal function
evaluation using {\tt put} and {\tt get}. The evaluation monad transformer is
shown in Fig.~\ref{figure:signal_function_evaluation_monad_transformer}.
Instances of typeclasses including {\tt Monad} and {\tt MonadTrans} are derived
using the ``GeneralizedNewtypeDeriving'' extension to the Glasgow Haskell
Compiler.

\begin{figure}[t]
\begin{code}
newtype SFEvalT svIn svOut m a = StateT (SFEvalState m svIn svOut) m a
\end{code}
\hrule
\caption{Signal function evaluation monad transformer.}
\label{figure:signal_function_evaluation_monad_transformer}
\end{figure}

\section{Discussion}
\label{section:Discussion}

The system presented here, TimeFlies, demonstrates how using signal vectors
to type inputs and outputs enables push-based evaluation of events in a
signal-function system. We take advantage of this representation in several ways.

First, by separating components of inputs and outputs in the types, we are free
to create distinct, and often partial, representations of the input or output
of a signal function. This enables us to represent only the event occurrence
being pushed at that time.

Second, this separation also permits us to separate the process of gathering
the input to a signal function, and the process of handling its output, into
different points in a program. Using the evaluation interface described, an
event occurrence may be pushed onto one input of a signal function from one
point in a program (e.g. a mouse click handler), an input signal
may be updated in another (e.g. a mouse movement handler), and finally the
system may be sampled in a third place (e.g. an animation or audio timed
callback).

Finally, this approach enables further work on the implementation of the signal
function system to be separated from changes in the interface. By enabling
differing representations of the inputs and outputs of signal functions, we are
free to change these representations without the need to further constrain the
input and output types.

\section{Ongoing and Further Work}
\label{section:Ongoing_and_Further_Work}

TimeFlies, the system described here, has been implemented, but not
extensively tested. The immediate goal is to create a real-time application
which will permit a performance and implementation comparison of TimeFlies with
Yampa, the current state-of-the-art pull-based signal-function system.

In the future, we hope to apply run-time optimizations, using the technique used
for Yampa, to create a push-pull self-optimizing signal-function system. Further,
we hope to use this system as a basis for exploring signal-function FRP as a
basis for general-purpose application frameworks.

\section{Related Work}
\label{section:Related_Work}

Signal Function FRP was introduced as a model for Graphical User Interfaces~\cite{Courtney2001-1}.
The system was originally termed ``AFRP'' (Arrowized FRP). Yampa is a rewrite of
AFRP where signal functions apply a number of ad-hoc optimizatons to themselves
as they evolve. Yampa demonstrated a modest performance improvement
over AFRP~\cite{Nilsson2005}.

Reactive is a classic FRP system which implements push-based evaluation for events
by transforming behaviors to ``reactive~normal~form,'' where a behavior
is a non-reactive behavior running inside a switch, whose event stream carries
behaviors in reactive normal form. The system is evaluated by forking a Haskell
thread to repeatedly sample the non-reactive behavior, and then blocking on the
evaluation of the first occurrence in the event stream. When this occurrence
is yielded, the evaluation thread for the behavior is killed and a new
thread forked to evaluate the new behavior~\cite{Elliott2009}. 

\section{Conclusion}
\label{section:Conclusion}

We have presented TimeFlies, a system for push-pull signal-function Functional
Reactive Programming, and have shown how the use of a signal vectors as input
and output types for signal functions, together with GADT-based representations
of the inputs and outputs, permits the implementation of a push-pull system.

We have also described a general and flexible monadic evaluation interface for
TimeFlies, which permits us to interface the TimeFlies system with different
styles of IO systems, including multiple IO systems in the same application.

This opens up the exciting possibility that a signal-function FRP could become
an efficient and general framework for writing interactive applications.

\bibliographystyle{splncs}
\bibliography{thesis}

\end{document}