\section{Discussion}
\label{section:Discussion}

The system presented here, TimeFlies, demonstrates how using signal vectors
to type inputs and outputs enables push-based evaluation of events in a
signal-function system. We take advantage of this representation in several ways.

First, by separating components of inputs and outputs in the types, we are free
to create distinct, and often partial, representations of the input or output
of a signal function. This enables us to represent only the event occurrence
being pushed at that time.

Second, this separation also permits us to separate the process of gathering
the input to a signal function, and the process of handling its output, into
different points in a program. Using the evaluation interface described, an
event occurrence may be pushed onto one input of a signal function from one
point in a program (e.g. a mouse click handler), an input signal
may be updated in another (e.g. a mouse movement handler), and finally the
system may be sampled in a third place (e.g. an animation or audio timed
callback).

Finally, this approach enables further work on the implementation of the signal
function system to be separated from changes in the interface. By enabling
differing representations of the inputs and outputs of signal functions, we are
free to change these representations without the need to further constrain the
input and output types.
