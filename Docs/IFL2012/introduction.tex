\section{Introduction}
\label{section:Introduction}

Functional Reactive Programming (FRP) is a class of systems for describing
reactive programs. Reactive programs are programs which, rather than taking
a single input and producing a single output, must accept multiple inputs and 
alter temporal behavior, including the production of multiple outputs, based
on these inputs.

An FRP system will provide a means of manipulating {\em behaviors} and
{\em events}. Behaviors are often referred to as {\em signals} in FRP literature,
but the definition is the same. A behavior or signal is, semantically, a
function from time to a value. An event is a discrete, possibly infinite, and 
time-ordered sequence of occurrences, which are times paired with values.

FRP systems can generally be categorized as ``classic FRP,'' 
which corresponds to the originally described FRP system in that behaviors
and events are manipulated directly and are first-class values in the FRP
system, or ``signal-function FRP,'' in which behaviors (generally termed
signals in this approach) and events are not first-class values, but signal
functions are first class values. Signal functions are time-dependent and
reactive transformers of signals, events, or combinations of signals and events.

FRP combines behaviors and events through the use of {\em switching}, in which
a behavior (in classic FRP) or a signal function (in signal-function FRP) is
replaced by a new behavior or signal function carried by an event occurrence.

Classic FRP was first described as an system for interactive animations~\cite{Elliott1997}.
Recent work on classic FRP has focused on efficient implementation. One approach to
efficiency is to separate the evaluation of behaviors and events, since suitable 
strategies give best performance in each case. Push-based evaluation evaluates a
system only when input is available, and is thus suitable for discrete inputs
such as events. Pull-based evaluation evaluates the system as quickly as
possible, polling for input, and is preferable for behaviors and signals.
The initial implementations of FRP made use of pull-based evaluation for both
behaviors and events. Reactive~\cite{Elliott2009}, as well as more recent systems such as
``reactive-banana''~\cite{Apfelmus}, make use of push-based evaluation for
events and pull-based evaluation for behaviors. This is known as ``push-pull''
evaluation.

All implementations of signal-function FRP to date~\cite{Courtney2001-1,Nilsson2002,Nilsson2005,Sculthorpe2011}
have used pull-based evaluation for both signals and events. This is due to
the ease of implementation of pull-based evaluation, and the types used for
signal functions which do not permit distinguishing signals and events, or
constructing only part of the input (for instance, one event occurrence.)

A recent extension of signal-function FRP called N-Ary FRP~\cite{Sculthorpe2011}
describes a method of typing signal functions which, as we will show, enables
the push-based evaluation of events in a signal-function FRP system. The notion
of signal vectors allows the representation of signal function inputs and
outputs as combinations of signals and events, rather than a single signal which
may contain multiple values, including option values for events. Signal vectors
are uninhabited types, which can be used to type partial or full representations
of the signal function inputs and outputs.

We present TimeFlies,\footnote{The sentence ``Time flies like an arrow.'' is a 
favorite quotation of one of the author's philosophy instructors, used to
demonstrate the ambiguity of language. The origin of the quotation is unknown.}
a push-pull signal-function FRP system. We hope to demonstrate the feasibility
of such an approach to FRP, and provide a basis for further research into
efficient implementation of signal-function FRP. We also describe a powerful
evaluation interface for TimeFlies, which permits us to use TimeFlies to
describe applications which make use of multiple and differing IO libraries.

Section~\ref{section:System_Design_and_Interface} describes design choices for the system,
and provides an overview of the interface. Section~\ref{section:Implementation}
describes how the system is implemented, and how the separation of evaluation
between events and signals is achieved. Section~\ref{section:Discussion} is a
discussion of the usefulness of our implementation. 
Section~\ref{section:Ongoing_and_Further_Work} describes the current and future
work on this system. Section~\ref{section:Related_Work} gives an overview of
related efforts. Section~\ref{section:Conclusion} concludes.

