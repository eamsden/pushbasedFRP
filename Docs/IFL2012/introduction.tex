\section{Introduction}
\label{section:Introduction}

Functional Reactive Programming (FRP) is a class of systems for describing
interactive programs. These are programs which, rather than batch-processing
inputs to produce an output, instead map time-varying inputs onto outputs.

For example, consider an application for monitoring thunderstorms. Such an
application might be employed by a sports venue, or a manufacturing center with
sensitive equipment. The application receives input from several sources,
including barometers (pressure), anemometers (wind speed and direction), rain
gauges, and lightning detectors. The output of the application is a pictoral map
with information from the sensors displayed, as well as an overlay showing an
estimate of the sizes and motions of storms.

An FRP system provides a means of manipulating {\em signals} and
{\em events}. Signals are often referred to as {\em behaviors} in earlier FRP literature,
but the definition is the same.  Semantically, a signal or behavior is a
function from time to a value. An event is a ordered sequence of discrete
occurrences, \textred{carrying both values and timestamps}. In our example, inputs from
the barometers, anemometers, and rain gauges would be signals. The
information carried by the signal would be the measurement made by the
instrument, perhaps tagged with the instrument's location. The input from the
lighting detector would be an event, whose occurrences would each carry the
location and intensity of a lightning strike.

One key distinction in FRP systems concerns how signals and events are
evaluated. There are two possibilities. Pull-based evaluation uses repeated and
frequent updates in a loop to check for updated values (polling).
% (similar to busy waiting in concurrency, except that for signals, there should be no waiting). 
\rn{``In concurrency'' seems awkward.  Maybe ``in concurrent
  programming'', or just nix.  What do you think of this alternative?}
Push-based
evaluation blocks or waits to be notified about an updated value (blocking).
% , similar to notification in concurrency. 
Pull-based evaluation is appropriate for signals,
which approximate continuous values and should have a value available whenever
they are sampled. Push-based evaluation is appropriate for events, which must be
reacted to quickly without consuming unnecessary resources to wait. Turning
again to our example: it is appropriate for our system to repeatedly query the
pressure, wind speed, and rain levels \new{to update the real-time model}. The system should respond quickly to
lightning strikes, but should not use resources in repeatedly sampling for the
occurrence (or not) of a lightning strike.  The system should instead be notified
when a strike has occurred, \new{and that notification should {\em not} result in 
recomputing all signals upon each strike event.}

\rn{Shall we rename {\em behavior} to {\em signal} below this point?}

FRP systems can generally be categorized as ``classic FRP,'' in which behaviors
and events are manipulated directly as first-class values,
or {\em signal-function FRP}. In signal-function FRP, the programmer cannot
directly manipulate signals, rather, the programmer manipulates time-dependent,
reactive transformers of signals, events, or combinations of signals and events.
%
FRP combines behaviors and events through the use of {\em switching}, in which
a behavior (in classic FRP) or a signal function (in signal-function FRP) is
replaced by a new behavior or signal function carried by an event occurrence.

An FRP program is generally evaluated by supplying a behavior or signal function
to an evaluation function, which provides input (in the case of a signal
function) and actuates its outputs as they are produced. For our weather system
example, the evaluation function is responsible for sampling the sensors,
receiving notification from the lightning strike detectors and pushing the
occurrences to the signal function, and drawing the picture which is the output
signal of the signal function.

Classic FRP was first described as a system for interactive animations~\cite{Elliott1997}.
%% Recent work on classic FRP has focused on efficient implementation. One approach
%% to efficiency is to separate the evaluation of behaviors and events, since
%% as was discussed above, the suitable strategy for evaluating signals is not the
%% same as that for events. 
\rn{Let's not be non-specific: ``focus on efficient implementation.
  One approach...'', if there aren't actually other examples of
  ``efficient implementation'' and we're really just talking about push-pull.}
%%
The initial implementations of FRP made use of
pull-based evaluation for both behaviors and events.
``Reactive''~\cite{Elliott2009}, as well as more recent systems such as
``reactive-banana''~\cite{Apfelmus}, make use of push-based evaluation for
events and pull-based evaluation for behaviors. This is known as ``push-pull''
evaluation \new{and is asymptotically more efficient}.

Almost all implementations of signal-function FRP to
date~\cite{Courtney2001-1,Nilsson2002,Nilsson2005,Sculthorpe2011} have used
pull-based evaluation for both signals and events. This is due to the ease of
implementation of pull-based evaluation, and because the types used for signal
functions do not permit distinguishing signals and events, or constructing only
part of the input (for instance, an occurrence on one of the event inputs). One
signal function system~\cite{Jeffrey2012} does permit the user to push
{\em segments} of signals, but, like the others, does not distinguish events
from signals in a way that permits events to be separately pushed.

A recent extension of signal-function FRP called N-Ary FRP~\cite{Sculthorpe2011}
describes a method of typing signal functions which, as we will show, enables
the push-based evaluation of events in a signal-function FRP system. The notion
of {\em signal vectors} allows the \new{(type-level)} representation of signal-function inputs and
outputs as combinations of signals and events, rather than a single
signal \new{of composite} values, including option values for events. Signal vectors
are uninhabited types, which can be used to describe partial or full
representations of the signal function inputs and outputs.
\rn{But what's the limitation of N-Ary FRP?! It doesn't allow full
  push-pull right?  Or wait, the SV technique ``enables'' push-pull,
  but the previous sculthorpe implementation just didn't do it?  Clarify.}

\new{In this paper we introduce}
TimeFlies,\footnote{The sentence ``Time flies like an arrow.'' is a 
favorite quotation of one of the author's philosophy instructors, used to
demonstrate the ambiguity of language. The origin of the quotation is unknown.},
a push-pull signal-function FRP system, implemented in the Haskell
programming language. Our contributions are as follows:
% \rn{Should be the paper contributions, not the system!}

\begin{itemize}
  \item We show that typing signal-function FRP with signal vectors allows
        push-pull evaluation of signal functions. \rn{FIRST?}
  \item We demonstrate a programmable evaluation interface that allows the user
        of TimeFlies to take advantage of this behavior while interfacing
        TimeFlies to a wide range of IO systems.
  \item We give benchmark results which demonstrate that the push-pull
        evaluation strategy makes TimeFlies more efficient than the current
        pull-based implementation of signal-function FRP.
\end{itemize}

Section~\ref{section:System_Design_and_Interface} describes design choices for
the system, and provides an overview of the interface.
Section~\ref{section:Implementation} describes how the system is implemented,
and how the separation of evaluation between events and signals is achieved.
Section~\ref{section:Discussion} is a discussion of the performance of our
implementation in the implementation of a network application. Section~\ref{section:Related_Work} gives an overview of
related efforts. Section~\ref{section:Conclusion} concludes.

