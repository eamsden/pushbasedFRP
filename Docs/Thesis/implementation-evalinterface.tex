\section{Evaluation Interface}
\label{section:Implementation-Evaluation_Interface}
The evaluation interface provides the means of evaluating a signal function
with inputs and producing effects in response to the signal function's outputs.
We would like to produce a set of constructs that interacts well with Haskell's
system for external IO.

\subsection{Constructs and Interface}
\label{section:Implementation-Evaluation_Interface-Constructs_and_Interface}

The evaluation interface is exported as shown in
Section~\ref{section:System_Design_and_Interface-Evaluator}.

The {\tt SFEvalState} type constructor parameterizes over input types for signal
functions, and underlying monads for a monad transformer, but is not itself
a monad transformer. It describes the state of signal function evaluation.
It consists of a record with four members: the current signal function,
the set of handlers for outputs, the current input signal delta, and the last
sample time.

The {\tt SFEvalT} monad transformer is a newtype wrapper around the {\tt StateT}
monad available in the Haskell {\tt transformers} package. The {\tt StateT} monad
wraps any other monad, providing a state which may be read using the {\tt get}
action and replaced with the {\tt put} action. An action in the underlying
monad is obtained by supplying a {\tt StateT} value and an initial state
value to the {\tt runStateT} function.

The {\tt SFEvalT} monad transformer does not make the {\tt get} and {\tt put}
actions available, but uses them in its implementation of the {\tt push},
{\tt update}, and {\tt sample} actions.

The {\tt push} action pushes an event onto the input of the signal function,
resulting in immediate evaluation of the relevant components signal function
and possible actuation of handlers specified for output events. It is
implemented by fetching the {\tt SFEvalState}, applying the event continuation
of the signal function contained by the {\tt SFEvalState} to the pushed event,
thus obtaining a new signal function and a list of events, applying the handlers
contained in the {\tt SFEvalState} to the output events, replacing the signal
function in the {\tt SFEvalState}, and replacing the {\tt SFEvalState}.

The {\tt update} action updates the current input signal delta, which will be
sampled at the next {\tt step} action. It has no immediately observable effect.
It simply updates the value of one signal in the input signal vector, a delta
of which is stored in the {\tt SFEvalState} record.

The {\tt step} action takes a time delta, and calls the signal continuation of
the stored signal function. It actuates the outputs using the stored handlers,
replaces the stored delta with an empty delta, and stores the resulting new
signal function in the state.
