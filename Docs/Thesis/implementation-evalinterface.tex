\section{Evaluation Interface}
\label{section:Implementation-Evaluation_Interface}
The evaluation interface provides the means of evaluating a signal function
with inputs and producing effects in response to the signal function's outputs.
We would like to produce a set of constructs that interacts well with Haskell's
system for external IO.

The evaluation interface translates between signal functions and the standard
Haskell construct for sequencing effects and external inputs, namely,
{\em monads}~\cite{PeytonJones1993}. The inspiration for monads is drawn from
the rather esoteric domain of category theory, but the concept as applied to
programming languages is actually rather simple.

In a programming language with type constructors, a monad is simply a 1-arity
type constructor together with two operations. We shall call this type
constructor $m$. The first operation takes a value of any type
(call the type $a$) and produces a value of type $m \: a$ ($m$ applied to $a$).
The second operation takes a value of type $m \: a$ and a function from $a$ to
$m \: b$, and produces a $m \: b$. This means that this second operation,
monadic application, applies the function to the value in the context of the
monad.

For our purposes, there are two interesting properties of monads. One is that
for a specific type constructor, opaque primitives may be defined which
can be used, along with return, to build functions for the second argument
of the monadic application operation. In our case, we would like operations
to push an event to a signal function, to update the signal components of
the signal function, and to increment the time and sample the signal function.

This property is what enables monads to be used for input and output. A Haskell
{\tt main} program is an IO action. An IO action is a value of the monadic type
IO. This action is comprised of a primitive action (either a {\tt return} or a 
true IO primitive), possibly together with a function from the output of that
action to another IO action. Note that values with the IO type may appear
anywhere in a Haskell program, but they are only execute when they are returned
as a part of the {\tt main} IO program.

The Haskell typeclass for monads is defined:

\begin{code}
class Monad m where
  return :: a -> m a
  -- | Infix operator for monadic application
  (>>=)  :: m a -> (a -> m b) -> m b 
\end{code}

The other interesting property of monads is the existence of a class of monads
called monad transformers. These are arity-2 type constructors which take as
their first parameter a monad type constructor. The partial application of
such a type constructor to any monad type constructor then results in a monad.
A construct is then provided to lift values of the underlying monad to values
in the transforming monad, which may then be sequenced as normal. This leads
to the powerful concept of monad transformer stacks, where monads with many
different capabilities are combined without any more plumbing than explicit
{\tt lift} operations. 

This property will be useful because we wish our evaluation interface to be
part of a monad stack, the base of which is Haskell's IO monad. By formulating
the evaluation interface as a monad transformer, we need only define the
operations relevant to the evaluation of signal functions, and we can depend
on the constructs of the IO monad to interact with whatever inputs and outputs
are necessary. In some cases, we may not wish to use the IO monad at all (e.g
for testing or simulation). In this case, we can parameterize over another
monad, such as the Identity monad (which has no special operations and whose
context is just the value), or the State monad (which maintains an implicit 
state accessible by {\tt get} and {\tt put} operations).

The monad tranformer typeclass is defined as:

\begin{code}
class MonadTrans t where
  lift :: Monad m => m a -> t m a
\end{code}

Haskell also supports {\em do}-notation for monads. Do notation simply
makes monadic application implicit between lines and reverses the syntatic
order of lambda bindings, providing a way to use monads in a manner similar
to imperative programming. (See Figure~\ref{figure:monad-example} for an
example.)

\begin{figure}
\begin{code}
-- Do notation:
do putStr "Hello: "
   string <- getStr
   putStrLn $ "Nice to meet you " ++ string
   time <- getCurrentTime
   print time
   putStrLn "Look at the time! Bye."

-- Translates to:
putStr "Hello: " >> getStr >>= 
(\string -> (putStrLn $ "Nice to meet you " ++ string) >>
 getCurrentTime >>=
 (\time -> print time >> putStrLn "Look at the time! Bye."))
\end{code}
\hrule
\caption{An example of monadic $do$-notation and its equivalent monadic
expression.}
\label{figure:monad-example}
\end{figure}

\subsection{Constructs and Interface}
\label{section:Implementation-Evaluation_Interface-Constructs_and_Interface}

The evaluation interface is exported as follows:

\begin{code}
data SFEvalState svIn m
data SFEvalT svIn m
data HandlerSet sv m
data EventOccurrence sv
data SignalDelta sv

sd      :: a -> SignalDelta (SVSignal a)
sdLeft  :: SignalDelta svl -> SignalDelta (SVAppend svl svr)
sdRight :: SignalDelta svr -> SignalDelta (SVAppend svl svr)
sdBoth  :: SignalDelta svl -> SignalDelta svr 
        -> SignalDelta (SVAppend svl svr)

eo      :: a -> EventOccurrence (SVEvent a)
eoLeft  :: EventOccurrence svl -> EventOccurrence (SVAppend svl svr)
eoRight :: EventOccurrence svr -> EventOccurrence (SVAppend svl svr)

sdHS    :: (a -> m ()) -> HandlerSet m (SVSignal a)
eoHS    :: (a -> m ()) -> HandlerSet m (SVEvent a)
hsLeft  :: HandlerSet m svl -> HandlerSet m (SVAppend svl svr)
hsRight :: HandlerSet m svr -> HandlerSet m (SVAppend svl svr)
hsBoth  :: HandlerSet m svl -> HandlerSet m svr 
        -> HandlerSet m (SVAppend svl svr)

initSFEval :: HandlerSet svOut m 
           -> SF NonInitialized svIn svOut 
           -> SFEvalState svIn m

runSFEvalT :: SFEvalT svIn m a -> SFEvalState svIn m -> m a

push   :: EventOccurrence svIn -> SFEvalT svIn m ()
update :: SignalDelta svIn -> SFEvalT svIn m ()
sample :: Double -> SFEvalT svIn m ()
\end{code}

The {\tt SFEvalState} type constructor parameterizes over input types for signal
functions, and underlying monads for a monad transformer, but is not itself
a monad transformer. It describes the state of signal function evaluation.
It consists of a record with four members: the current signal function,
the set of handlers for outputs, the current input signal delta, and the last
sample time.

The {\tt SFEvalT} monad transformer is a newtype wrapper around the {\tt StateT}
monad available in the Haskell {\tt transformers} package. The StateT monad
wraps any other monad, providing a state which may be read using the {\tt get}
action and replaced with the {\tt put} action. An action in the underlying
monad is obtained by supplying a {\tt StateT} value and an initial state
value to the {\tt runStateT} function.

The {\tt SFEvalT} monad transformer does not make the {\tt get} and {\tt put}
actions available, but uses them in its implementation of the {\tt push},
{\tt update}, and {\tt sample} actions.

The {\tt push} action pushes an event onto the input of the signal function,
resulting in immediate evaluation of the relevant components signal function
and possible actuation of handlers specified for output events. It is
implemented by fetching the {\tt SFEvalState}, applying the event continuation
of the signal function contained by the {\tt SFEvalState} to the pushed event,
thus obtaining a new signal function and a list of events, applying the handlers
contained in the {\tt SFEvalState} to the output events, replacing the signal
function in the {\tt SFEvalState}, and replacing the {\tt SFEvalState}.
