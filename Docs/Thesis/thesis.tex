%%%%%%%%%%%%%%%%%%%%%%%%%%%%%%%%%%%%%%%%%%%%%%%%%%%%%%%%%%%%%%%%%%%%%%
%%  disstemplate.tex, to be compiled with latex.             %
%%  08 April 2002   Version 4                    %
%%%%%%%%%%%%%%%%%%%%%%%%%%%%%%%%%%%%%%%%%%%%%%%%%%%%%%%%%%%%%%%%%%%%%%
%%                                   %
%%  Writing a Doctoral Dissertation with LaTeX at            %
%%  the University of Texas at Austin                %
%%                                   %
%%  (Modify this ``template'' for your own dissertation.)        %
%%                                   %
%%%%%%%%%%%%%%%%%%%%%%%%%%%%%%%%%%%%%%%%%%%%%%%%%%%%%%%%%%%%%%%%%%%%%%


\documentclass[12pt]{report}    % The documentclass must be ``report''.

\usepackage{utdiss2}        % Dissertation package style file.


%%%%%%%%%%%%%%%%%%%%%%%%%%%%%%%%%%%%%%%%%%%%%%%%%%%%%%%%%%%%%%%%%%%%%%
% Optional packages used for this sample dissertation. If you don't  %
% need a capability in your dissertation, feel free to comment out   %
% the package usage command.                         %
%%%%%%%%%%%%%%%%%%%%%%%%%%%%%%%%%%%%%%%%%%%%%%%%%%%%%%%%%%%%%%%%%%%%%%

\usepackage{amsmath,amsthm,amsfonts,amscd}
                % Some packages to write mathematics.
\usepackage{eucal}      % Euler fonts
\usepackage{fancyvrb}                                       % For code environment
\DefineVerbatimEnvironment{code}{Verbatim}{fontsize=\small} %
\usepackage{makeidx}        % Package to make an index.
\usepackage{psfig}          % Allows inclusion of eps files.
\usepackage{epsfig}             % Allows inclusion of eps files.
\usepackage{citesort}           %
\usepackage{url}        % Allows good typesetting of web URLs.
%\usepackage{draftcopy}     % Uncomment this line to have the
                % word, "DRAFT," as a background
                % "watermark" on all of the pages of
                % of your draft versions. When ready
                % to generate your final copy, re-comment
                % it out with a percent sign to remove
                % the word draft before you re-run
                % Makediss for the last time.



\author{Edward Amsden }    % Required

\address{13567 McCartyville Road \\ Anna, Ohio 45302}  % Required

\title{TimeFlies: \\ Push-Pull Signal-Function \\ Functional Reactive Programming}
                                                    % Required

%%%%%%%%%%%%%%%%%%%%%%%%%%%%%%%%%%%%%%%%%%%%%%%%%%%%%%%%%%%%%%%%%%%%%%
% NOTICE: The total number of supervisors and other members %%%%%%%%%%
%%%%%%%%%%%%%%% MUST be seven (7) or less! If you put in more, %%%%%%%
%%%%%%%%%%%%%%% they are put on the page after the Committee %%%%%%%%%
%%%%%%%%%%%%%%% Certification of Approved Version page. %%%%%%%%%%%%%%
%%%%%%%%%%%%%%%%%%%%%%%%%%%%%%%%%%%%%%%%%%%%%%%%%%%%%%%%%%%%%%%%%%%%%%

%%%%%%%%%%%%%%%%%%%%%%%%%%%%%%%%%%%%%%%%%%%%%%%%%%%%%%%%%%%%%%%%%%%%%%
%
% Enter names of the supervisor and co-supervisor(s), if any,
% of your dissertation committee. Put one name per line with
% the name in square brackets. The name on the last line, however,
% must be in curly braces.
%
% If you have only one supervisor, the entry below will read:
%
%   \supervisor
%       {Supervisor's Name}
%
% NOTE: Maximum three supervisors. Minimum one supervisor.
% NOTE: The Office of Graduate Studies will accept only two supervisors!
%
%
\supervisor
    {Dr. Matthew Fluet}

%%%%%%%%%%%%%%%%%%%%%%%%%%%%%%%%%%%%%%%%%%%%%%%%%%%%%%%%%%%%%%%%%%%%%%
%
% Enter names of the other (non-supervisor) members(s) of your
% dissertation committee. Put one name per line with the name
% in square brackets. The name on the last line, however, must
% be in curly braces.
%
% NOTE: Maximum six other members. Minimum zero other members.
% NOTE: The Office of Graduate Studies may restrict you to a total
%   of six committee members.
%
%
\committeemembers
%    [Erwin Schr\"odinger]
    [Arthur Nunes-Harwitt, Reader]
    {Dr. Zach Butler, Observer}
%    {Arthur Schawlow}

%%%%%%%%%%%%%%%%%%%%%%%%%%%%%%%%%%%%%%%%%%%%%%%%%%%%%%%%%%%%%%%%%%%%%%

\previousdegrees{B.S.}
     % The abbreviated form of your previous degree(s).
     % E.g., \previousdegrees{B.S., MBA}.
     %
     % The default value is `B.S., M.S.'

\graduationmonth{August}
     % Graduation month, either May, August, or December, in the form
     % as `\graduationmonth{May}'. Do not abbreviate.
     %
     % The default value (either May, August, or December) is guessed
     % according to the time of running LaTeX.

\graduationyear{2013}
     % Graduation year, in the form as `\graduationyear{2001}'.
     % Use a 4 digit (not a 2 digit) number.
     %
     % The default value is guessed according to the time of
     % running LaTeX.

%\typist{...}
     % The name(s) of typist(s), put `the author' if you do it yourself.
     % E.g., `\typist{Maryann Hersey and the author}'.
     %
     % The default value is `the author'.


%%%%%%%%%%%%%%%%%%%%%%%%%%%%%%%%%%%%%%%%%%%%%%%%%%%%%%%%%%%%%%%%%%%%%%
% Commands for master's theses and reports.              %
%%%%%%%%%%%%%%%%%%%%%%%%%%%%%%%%%%%%%%%%%%%%%%%%%%%%%%%%%%%%%%%%%%%%%%
%
% If the degree you're seeking is NOT Doctor of Philosophy, uncomment
% (remove the % in front of) the following two command lines (the ones
% that have the \ as their second character).
%
\degree{Master of Science} \degreeabbr{M.S.}

% Uncomment the line below that corresponds to the type of master's
% document you are writing.
%
%\masterreport
\masterthesis


%%%%%%%%%%%%%%%%%%%%%%%%%%%%%%%%%%%%%%%%%%%%%%%%%%%%%%%%%%%%%%%%%%%%%%
% Some optional commands to change the document's defaults.      %
%%%%%%%%%%%%%%%%%%%%%%%%%%%%%%%%%%%%%%%%%%%%%%%%%%%%%%%%%%%%%%%%%%%%%%
%
%\singlespacing
%\oneandonehalfspacing

%\singlespacequote
\oneandonehalfspacequote

\topmargin 0.125in  % Adjust this value if the PostScript file output
            % of your dissertation has incorrect top and
            % bottom margins. Print a copy of at least one
            % full page of your dissertation (not the first
            % page of a chapter) and measure the top and
            % bottom margins with a ruler. You must have
            % a top margin of 1.5" and a bottom margin of
            % at least 1.25". The page numbers must be at
            % least 1.00" from the bottom of the page.
            % If the margins are not correct, adjust this
            % value accordingly and re-compile and print again.
            %
            % The default value is 0.125"

        % If you want to adjust other margins, they are in the
        % utdiss2-nn.sty file near the top. If you are using
        % the shell script Makediss on a Unix/Linux system, make
        % your changes in the utdiss2-nn.sty file instead of
        % utdiss2.sty because Makediss will overwrite any changes
        % made to utdiss2.sty.

%%%%%%%%%%%%%%%%%%%%%%%%%%%%%%%%%%%%%%%%%%%%%%%%%%%%%%%%%%%%%%%%%%%%%%
% Some optional commands to be tested.                   %
%%%%%%%%%%%%%%%%%%%%%%%%%%%%%%%%%%%%%%%%%%%%%%%%%%%%%%%%%%%%%%%%%%%%%%

% If there are 10 or more sections, 10 or more subsections for a section,
% etc., you need to make an adjustment to the Table of Contents with the
% command \longtocentry.
%
%\longtocentry



%%%%%%%%%%%%%%%%%%%%%%%%%%%%%%%%%%%%%%%%%%%%%%%%%%%%%%%%%%%%%%%%%%%%%%
%   Some math support.                       %
%%%%%%%%%%%%%%%%%%%%%%%%%%%%%%%%%%%%%%%%%%%%%%%%%%%%%%%%%%%%%%%%%%%%%%
%
%   Theorem environments (these need the amsthm package)
%
%% \theoremstyle{plain} %% This is the default

\newtheorem{thm}{Theorem}[section]
\newtheorem{cor}[thm]{Corollary}
\newtheorem{lem}[thm]{Lemma}
\newtheorem{prop}[thm]{Proposition}
\newtheorem{ax}{Axiom}

\theoremstyle{definition}
\newtheorem{defn}{Definition}[section]

\theoremstyle{remark}
\newtheorem{rem}{Remark}[section]
\newtheorem*{notation}{Notation}

%\numberwithin{equation}{section}


%%%%%%%%%%%%%%%%%%%%%%%%%%%%%%%%%%%%%%%%%%%%%%%%%%%%%%%%%%%%%%%%%%%%%%
%   Macros.                              %
%%%%%%%%%%%%%%%%%%%%%%%%%%%%%%%%%%%%%%%%%%%%%%%%%%%%%%%%%%%%%%%%%%%%%%
%
%   Here some macros that are needed in this document:


\newcommand{\latexe}{{\LaTeX\kern.125em2%
                      \lower.5ex\hbox{$\varepsilon$}}}

\newcommand{\amslatex}{\AmS-\LaTeX{}}

\chardef\bslash=`\\ % \bslash makes a backslash (in tt fonts)
            %   p. 424, TeXbook

\newcommand{\cn}[1]{\texttt{\bslash #1}}

\makeatletter       % Starts section where @ is considered a letter
            % and thus may be used in commands.
\def\square{\RIfM@\bgroup\else$\bgroup\aftergroup$\fi
  \vcenter{\hrule\hbox{\vrule\@height.6em\kern.6em\vrule}%
                                              \hrule}\egroup}
\makeatother        % Ends sections where @ is considered a letter.
            % Now @ cannot be used in commands.

\makeindex    % Make the index

%%%%%%%%%%%%%%%%%%%%%%%%%%%%%%%%%%%%%%%%%%%%%%%%%%%%%%%%%%%%%%%%%%%%%%
%       The document starts here.                %
%%%%%%%%%%%%%%%%%%%%%%%%%%%%%%%%%%%%%%%%%%%%%%%%%%%%%%%%%%%%%%%%%%%%%%

\begin{document}

%\copyrightpage          % Produces the copyright page.


%
% NOTE: In a doctoral dissertation, the Committee Certification page
%       (with signatures) is BEFORE the Title page.
%   In a masters thesis or report, the Signature page
%       (with signatures) is AFTER the Title page.
%
%   If you are writing a masters thesis or report, you MUST REVERSE
%   the order of the \commcertpage and \titlepage commands below.
%
\titlepage              % Produces the title page.
\commcertpage           % Produces the Committee Certification
            %   of Approved Version page (doctoral)
            %   or Signature page (masters).
            %       20 Mar 2002 cwm




%%%%%%%%%%%%%%%%%%%%%%%%%%%%%%%%%%%%%%%%%%%%%%%%%%%%%%%%%%%%%%%%%%%%%%
% Dedication and/or epigraph are optional, but must occur here.      %
%%%%%%%%%%%%%%%%%%%%%%%%%%%%%%%%%%%%%%%%%%%%%%%%%%%%%%%%%%%%%%%%%%%%%%
%
%\begin{dedication}
%\index{Dedication@\emph{Dedication}}%
%Dedicated to my wife Shirley.
%\end{dedication}


\begin{acknowledgments}     % Optional
\index{Acknowledgments@\emph{Acknowledgments}}%
I wish to thank the many people who have supported me as I pursued this work. In particular,
my supervisor Dr. Matthew Fluet was both encouraging and challenging as I explored the question posed
by this thesis, and through the long process of revisions and preparing for my defense.
Professor Arthur Nunes-Harwitti's and Dr. Zack Butler's feedback and contributions were invaluable.
I would also like to thank Dr. Ryan Newton for his understanding and encouragement as I worked concurrently
on the beginning of my PhD studies and this thesis. I thank my family, my mother and father in particular, for
their encouragement and support of my education, and for all of the teaching they so diligently gave me to
enable me to reach this point. Finally and most importantly, \textit{Soli Deo gloria}.
\end{acknowledgments}


% The abstract is required. Note the use of ``utabstract'' instead of
% ``abstract''! This was necessary to fix a page numbering problem.
% The abstract heading is generated automatically.
% Do NOT use \begin{abstract} ... \end{abstract}.
%
\utabstract
\index{Abstract}%
\indent
\begin{abstract}
Functional Reactive Programming is a promising class of systems for writing
interactive and time-dependent programs. Signal-function FRP is a subclass of
these systems which provides advantages of modularity and correctness, but
has proven difficult to efficiently implement.

The abstraction of signal vectors provides the necessary type apparatus to
distinguish components of the input and output of signal functions which benefit
from a push-based implementation from those which benefit from a pull-based
implementation, and to combine both implementation strategies in a single system.

We describe a signal-function FRP system which provides push-based evaluation
for events, pull-based evaluation for signals, and a simple monadic evaluation
interface which permits the system to be easily integrated with one or more
IO systems.
\end{abstract}




\tableofcontents   % Table of Contents will be automatically
                   % generated and placed here.

%\listoftables     % List of Tables and List of Figures will be placed
\listoffigures     % here, if applicable.



%%%%%%%%%%%%%%%%%%%%%%%%%%%%%%%%%%%%%%%%%%%%%%%%%%%%%%%%%%%%%%%%%%%%%%
% Actual text starts here.                       %
%%%%%%%%%%%%%%%%%%%%%%%%%%%%%%%%%%%%%%%%%%%%%%%%%%%%%%%%%%%%%%%%%%%%%%
%
% Including external files for each chapter makes this document simpler,
% makes each chapter simpler, and allows for generating test documents
% with as few as zero chapters (by commenting out the include statements).
% This allows quicker processing by the Makediss command file in case you
% are not working on a specific, long and slow to compile chapter. You
% can even change the chapter order by merely interchanging the order
% of the include statements (something I found helpful in my own
% dissertation).
%

\section{Introduction}
\label{section:Introduction}

Functional Reactive Programming (FRP) is a class of systems for describing
reactive programs. Reactive programs are programs which, rather than 
\new{batch-processing inputs to produce an output,}
% taking a single input and producing a single output, 
\new{instead map time-varying inputs onto outputs.}
%% must accept multiple inputs and 
%% alter temporal behavior, including the production of multiple outputs, based
%% on these inputs.

An FRP system provides a means of manipulating {\em behaviors} and
{\em events}. 
\rn{Switched to present tense.  Don't need to add additional future or past tense...}
Behaviors are often referred to as {\em signals} in FRP literature,
but the definition is the same.  Semantically, a behavior or signal is a
function from time to a value.  
\new{Whereas an event is a ordered sequence of discrete occurrences,
  carrying both values and timestamps.}
%% An event is a discrete, possibly infinite, and 
%% time-ordered sequence of occurrences, which are times paired with values.

\rn{Looks like you should standardize early on {\bf signal} and leave
  extra occurrences of {\bf behavior} out of this intro.}


FRP systems can generally be categorized as ``classic FRP,'' 
% which corresponds to the originally described FRP system 
in which behaviors
and events are manipulated directly as first-class values,
% in the FRP system,
or {\em signal-function FRP}. 
% in which behaviors  (generally termed signals in this approach) 
\new{In signal-function FRP, the programmer cannot directly manipulate
  signals, rather, functions from input-signal to output-signal are
  the central abstraction.  Thus}
%% and events are not first-class values, but signal
%% functions are first class values. 
signal functions are time-dependent,
reactive transformers of signals, events, or combinations of signals and events.
%
FRP {\em combines} behaviors and events through the use of {\em switching}, in which
a behavior (in classic FRP) or a signal function (in signal-function FRP) is
replaced by a new behavior or signal function carried by an event occurrence.

\todo{TODO: it would be good to give SOME sense for how an FRP
  evaluation proceeds (e.g. a strawman) before getting into the next para.}

Classic FRP was first described as a system for interactive animations~\cite{Elliott1997}.
Recent work on classic FRP has focused on efficient implementation \rn{CITATION?}.  One approach to
efficiency is to separate the evaluation of behaviors and events, since suitable 
strategies give best performance in each case. Push-based evaluation evaluates a
system only when input is available, and is thus suitable for discrete inputs
such as events. Pull-based evaluation evaluates the system as quickly as
possible, polling for input, and is preferable for behaviors and signals.
The initial implementations of FRP made use of pull-based evaluation for both
behaviors and events. ``Reactive''~\cite{Elliott2009}, as well as more recent systems such as
``reactive-banana''~\cite{Apfelmus}, make use of push-based evaluation for
events and pull-based evaluation for behaviors. This is known as ``push-pull''
evaluation.


All implementations of signal-function FRP to date~\cite{Courtney2001-1,Nilsson2002,Nilsson2005,Sculthorpe2011}
have used pull-based evaluation for both signals and events. This is due to
the ease of implementation of pull-based evaluation, and because the types used for
signal functions do not permit distinguishing signals and events, or
constructing only part of the input (for instance, one event occurrence).

\rn{What does it mean to construct one event occurrence?  As opposed to?}

A recent extension of signal-function FRP called N-Ary FRP~\cite{Sculthorpe2011}
describes a method of typing signal functions which, as we will show, enables
the push-based evaluation of events in a signal-function FRP system. The notion
of signal vectors allows the representation of signal function inputs and
outputs as combinations of signals and events, rather than a single signal which
may contain multiple values, including option values for events. Signal vectors
are uninhabited types, which can be used to describe partial or full representations
of the signal function inputs and outputs.

We present TimeFlies,\footnote{The sentence ``Time flies like an arrow.'' is a 
favorite quotation of one of the author's philosophy instructors, used to
demonstrate the ambiguity of language. The origin of the quotation is unknown.}
a push-pull signal-function FRP system. We hope to demonstrate the feasibility
of such an approach to FRP, and provide a basis for further research into
efficient implementation of signal-function FRP. We also describe a powerful
evaluation interface for TimeFlies, which permits us to use TimeFlies to
describe applications which make use of multiple and differing IO libraries.

\rn{Need a concrete contributions list.   ``The first system to apply
  push-pull implementation strategies to signal-function FRP'', etc.
  It doesn't hurt for it to be bulletted.}

Section~\ref{section:System_Design_and_Interface} describes design choices for the system,
and provides an overview of the interface. Section~\ref{section:Implementation}
describes how the system is implemented, and how the separation of evaluation
between events and signals is achieved. Section~\ref{section:Discussion} is a
discussion of the \textred{usefulness} \rn{more specific?} of our implementation. 
Section~\ref{section:Ongoing_and_Further_Work} describes the current and future
work on this system. Section~\ref{section:Related_Work} gives an overview of
related efforts. Section~\ref{section:Conclusion} concludes.


\chapter{Background}
\label{chapter:Background}

The key abstraction of functional programming is the function, which takes an
input and produces an output. Functions may produce functions as output and/or
take functions as input. We say that functions are {\em first-class values} in a
functional programming language.

In contrast, the more popular imperative model of programming takes statements,
or actions which modify the state of the world, as the primary building blocks
of programs. Such programs have sequential control flow, and require reasoning
about side-effects. They are thus inherently resistant to compositional
reasoning.

Even in functional languages, reactive programs are generally written in an imperative style, using low-level and non-composable abstractions including callbacks
or object-based event handlers, called by event loops. This ties the model of interactivity to low-level implementation details such as timing and event handling models. 

{\em Functional} Reactive Programming implies that a model should keep the characteristics of functional programming (i.e. that the basic constructs of the language
should remain first-class) while incorporating reactivity into the language model. In particular, functions should be lifted to operate on reactive values,
and functions themselves ought to be reactive.

The goal of FRP is to provide compositional and high-level abstractions for
creating reactive programs. The key abstractions of FRP are behaviors
\footnote{Behaviors are generally referred to as {\em signals} in
signal-function literature. This is unfortunate, since a {\em signal function}
may operate on events, behaviors, or some combination of the two.}, which are
time-dependent values defined at every point in continuous time, and events,
which are values defined at countably many points in time. An FRP system will
provide combinators to manipulate events and behaviors and to react to events by
replacing a portion of the running program in response to an event. Behaviors
and events, or some abstraction based on them, will be first class, in keeping
with the spirit of functional programming. Programs implemented in FRP languages
should of course be efficiently executable, but this has proven to be the main
challenge in implementing FRP.

The two general approaches to FRP are ``classic'' FRP, where behaviors and
events are first-class and reactive objects, and ``signal-function'' FRP, where
transformers of behaviors and events are first-class and reactive objects.

\section{Classic FRP}
\label{section:Background-Classic_FRP}

The earliest and still standard formulation of FRP provides two primitive type
constructors: {\tt Behavior} and {\tt Event}, together with combinators that
produce values of these types. The easiest semantic definition for these types
is given in Figure~\ref{figure:classic_frp_semantic_types}\footnote{The list type constructor {\tt []} should be considered to contain infinite as well as finite lists.}.

\begin{figure}
\begin{code}
type Event a = [(Time, a)]
type Behavior a = Time -> a
\end{code}
\hrule
\caption{Semantic types for Classic FRP.}
\label{figure:classic_frp_semantic_types}
\end{figure}


When these two type constructors are exposed directly, the system is known as a 
{\em Classic FRP} system. If the type aliases are taken as given in the semantic
definition, a simple, yet problematic, implementation is given in
Figure~\ref{figure:classic_frp_simple_implementation}.

\begin{figure}
\begin{code}
time :: Behavior Time
time = id

constant :: a -> Behavior a
constant = const

delayB :: a -> Time -> Behavior a -> Behavior a
delayB a td b te = if te <= td
                   then a
                   else b (te - td)          

instance Functor Behavior where
  fmap = (.)

instance Applicative Behavior where
  pure = constant
  bf <*> ba = (\t -> bf t (ba t))

never :: Event a
never = []

once :: a -> Event a
once a = [(0, a)]

delayE :: Time -> Event a -> Event a
delayE td = map (\(t, a) -> (t + td, a))

instance Functor Event where
  fmap f = map (\(t, a) -> (t, f a))

switcher :: Behavior a -> Event (Behavior a) -> Behavior a
switcher b [] t = b t
switcher b ((to, bo):_) t = if t < to
                            then b t
                            else bo t
\end{code}
\hrule
\caption{An obvious, yet inefficient and problematic, implementation of
Classic FRP.}
\label{figure:classic_frp_simple_implementation}
\end{figure}

There are several obvious problems with this implementation of course, but it
suffices to show the intuition behind the classic FRP model. Problems in this
implementation which are addressed by real implementations include the necessity
of waiting for event occurrences, the necessity of maintaining the full history
of a behavior in memory, and the lack of an enforced order for event
occurrences.

\subsection{History of Classic FRP}
\label{subsection:Background-Classic_FRP-History_of_Classic_FRP}
Classic FRP was originally described as the basis of Fran~\cite{Elliott1997}
(Functional Reactive ANimation), a framework for declaratively specifying
interactive animations. Fran represented behaviors as two sampling functions,
one from a time to a value and a new behavior (so that history may be discarded),
and the other from a time interval (a lower and upper time bound) to a value
interval and a new behavior. Events are represented as ``improving values'',
which, when sampled with a time, produce a lower bound on the time to the next
occurrence, or the next occurrence if it has indeed occurred.

The first implementation of FRP outside the Haskell language was Frapp\'{e}~\cite{Courtney2001-2},
implemented in the Java Beans framework. Frapp\'{e} built on the notion
of events\footnote{These are semantic objects to which callbacks may be added,
and not events in the FRP sense, though they are related.} and ``bound
properties'' in the Beans framework, providing abstract interfaces for FRP
events and behaviors, and combinators as concrete classes implementing these
interfaces. The evaluation of Frapp\'{e} used Bean components as sources and
sinks, and the implementation of Bean events and bound properties to propagate
changes to the network.

\subsection{Current Classic FRP Systems}
\label{subsection:Background-Classic_FRP-Current_Classic_FRP_Systems}

The FrTime\footnote{FrTime is available in all recent versions of Dr. Racket}~cite{Cooper2006}
language extends the Scheme evaluator with a mutable dependency graph, which is
constructed by program evaluation. This graph is then updated by signal changes.
FrTime does not provide a distinct concept of events, and chooses branches of
the dependency graph by conditional evaluation of behaviors, rather than by the
substitution of behaviors used by FRP systems.

The Reactive~\cite{Elliott2009} system is a push-pull FRP system with
first-class behaviors and events. The primary insight of Reactive is the
separation of reactivity (or changes in response to events whose occurrence time
could not be known beforehand) from time-dependence. This gives rise to {\em
reactive normal form}, which represents a behavior as a constant or simply
time-dependent value, together with an event stream carrying values which are also
behaviors in reactive normal form. Push-based evaluation is achieved by forking
a Haskell thread to evaluate the head behavior, while waiting on the evaluation
of the event stream. Upon an event occurrence, the current behavior thread is
killed and a new thread spawned to evaluate the new behavior. Unfortunately, the
implementation of Reactive uses a tenuous technique which depends on forking
threads to evaluate two Haskell values concurrently, in order to implement event
merging. This relies on the library author to ensure consistency when this
technique is used, and leads to thread leakage when one of the merged events is
itself the merging of other events.

A recent thesis~\cite{Czaplicki2012} described Elm, a stand-alone language for
reactivity. Elm provides combinators for manipulating discrete events, and
compiles to JavaScript, making it useful for client-side web programming.
However, Elm does not provide a notion of switching or continuous time
behaviors, though an approximation is given using discrete-time events which are
actuated at repeated intervals specified during the event definition. The thesis
asserts that Arrowized FRP (signal-function FRP, Section~\ref{section:Background-signal_function_frp})
can be embedded in Elm, but provides little support for this assertion\footnote{
A form of Arrowized FRP employing applicative functors is presented, and
justified by the assertion that applicative functors are just arrows with the
input type parameter fixed. While this is true, it ignores the compositional
benefits of the general Arrow framework, and it is not clear that it provides
the same enforcement of causality.}

The reactive-banana~\cite{Apfelmus} library is a push-based FRP system designed
for use with Haskell GUI frameworks. In particular, it features a monad for
the creation of behaviors and events which may then be composed and evaluated.
This monad includes constructs for binding GUI library constructs to primitive
events. It must be ``compiled'' to a Haskell IO action for evaluation to take
place. The implementation of reactive-banana is similar to FrTime, using a
dependency graph to update the network on event occurences. Reactive-banana
eschews generalized switching in favor of branching functions on behavior
values, similarly to FrTime, but maintains the distinction between behaviors and
events. Rather than a generalized switching combinator which allows the
replacement of arbitrary behaviors, reactive-banana provides a step combinator
which creates a stepwise behavior from the values of an event stream.

\section{Signal Function FRP}
\label{section:Background-signal_function_frp}

An alternative approach to FRP was first proposed in work on
Fruit~\cite{Courtney2001-1}, a library for declarative specification of GUIs.
This library utilized the concept of Arrows~\cite{Hughes2000} as an abstraction
for {\em signal functions}. Arrows are abstract type constructors with input and
output type parameters, together with a set of routing combinators (see
Figure~\ref{figure:arrow_combinators}). The concept of an Arrow in Haskell,
including the axioms the combinators of an arrow must satisfy, are derived from
the concept of arrows from category theory.

Signal functions are time-dependent and reactive transformers of events and
behaviors. Behaviors and events cannot be directly manipulated by the programmer.
This approach has two motivations: it increases modularity since both the input
and output of signal functions may be transformed (as opposed to behaviors or
events which may only be transformed in their output) and it avoids a large
class of time and space leaks which have emerged when implementing FRP with
first-class behaviors and events.

\begin{figure}
\begin{code}
(>>>)  :: (Arrow a) => a b c -> a c d -> a b d
arr    :: (Arrow a) => (b -> c) -> a b c
first  :: (Arrow a) => a b c -> a (b, d) (c, d)
second :: (Arrow a) => a b c -> a (d, b) (d, c)
(***)  :: (Arrow a) => a b c -> a b d -> a b (c, d)
loop   :: (Arrow a) => a (b, d) (c, d) => a b c
\end{code}
\hrule
\caption{Arrow combinators.}
\label{figure:arrow_combinators}
\end{figure}

Similarly to FrTime, the netwire~\cite{Soylemez} library eschews dynamic
switching, in this case in favor of {\em signal inhibition}. Netwire is written
as an arrow transformer, and permits the lifting of IO actions as sources and
sinks at arbitrary points in a signal function network. Signal inhibition is
accomplished by making the output of signal functions a monoid, and then
combining the outputs of signal functions. An inhibited signal function will
produce the monoid's zero as an output. Primitives have defined inhibition
behavior, and composed signal functions inhibit if their outputs combine to the
monoid's zero.

Yampa~\cite{Nilsson2005} is an optimization of the Arrowized FRP system first
utilized with Fruit (see above). The implementation of Yampa makes use of
Generalized Algebraic Datatypes to permit a much larger class of type-safe
datatypes for the signal function representation. This representation,
together with ``smart'' constructors and combinators, enables the construction
of a self-optimizing arrowized FRP system. Unfortunately, the primary
inefficiency, that of unnecessary evaluation steps due to pull-based evaluation,
remains. Further, the optimization is ad-hoc and each new optimization requires
the addition of new constructors, as well as the updating of every primitive
combinator to handle every combination of constructors. However, Yampa showed
noticeable performance gains over previous Arrowized FRP implementations.

A recent PhD thesis~\cite{Sculthorpe2011} introduced N-Ary FRP, a technique for
typing Arrowized FRP systems using dependent types. The bulk of the work
consisted in using the dependent type system to prove the correctness of the FRP
system discussed. This work introduced signal vectors, a typing construct
which permits the distinction of behavior and event types at the level of the FRP
system, instead of making events merely a special type of behavior.

\section{Outstanding Challenges}
\label{section:Background-outstanding_challenges}

At present, there are two key issues apparent with FRP. First, while
signal-function FRP is inherently safer and more modular than classic FRP, it
has yet to be efficiently implemented. Classic FRP programs are vulnerable to
time leaks and violations of causality due to the ability to directly manipulate
reactive values. Second, the interface between FRP programs and the many
different sources of inputs and sinks for outputs available to a modern
application writer remains ad-hoc and is in most cases limited by the
implementation.

One key exception to this is the reactive-banana system, which provides a monad
for constructing primitive events and behaviors from which an FRP program may
then be constructed. However, this approach is still inflexible as it requires
library support for the system which the FRP program will interact with.
Further, being a Classic FRP system, reactive-banana lacks the ability to
transform the inputs of behaviors and events, since all inputs are implicit.

\chapter{System Design and Interface}
\label{chapter:System_Design_and_Interface}

\section{Goals}
\label{section:System_Design_and_Interface-Goals}

The (not yet fully realized) goal of FRP is to provide an efficient, declarative
abstraction for creating reactive programs. Towards this overall goal, there are
three goals which this system is intended to meet.

\subsection{Efficient Evaluation}
\label{subsection:System_Design_and_Interface-Goals-Efficient_and_Push_based_Evaluation}

Efficient evaluation is the motivation for push-based evaluation of events.
Since FRP programs are expected to  interact with an outside world in real time,
efficiency cannot be measured by the runtime of a program. Thus, when speaking of efficiency,
we are expressing a desire that the system utilize as few system resources as possible
for the task at hand, while responding as quickly as possible to external inputs and
producing output at a consistently high sample rate.

\subsection{Composability}
\label{subsection:System_Design_and_Interface-Goals-Composability}

A composable abstraction is one in which values in that abstraction may be
combined in such a way that reasoning about their actions together involves
little more than reasoning about their actions separately. In a signal function
system, the only interaction between composed signal functions ought to be that
the output of one is the input of another. Composability permits a particularly
attractive form of software engineering in which successively larger systems are
created from by combining smaller systems, without having to reason about the 
components of the systems being combined.

\subsection{Ease of Integration}
\label{subsection:System_Design_and_Interface-Goals-Ease_of_Integration}

It is fine for a system to be composable with regards to itself, but an FRP
system must interact with the outside world. Since we cannot anticipate every
possible form of input and output that the system will be asked to interact
with, we must interface with Haskell's IO system. In particular, most libraries
for user interaction (e.g. GUI and graphics libraries such as GTK+ and GLUT) and
most libraries for time-dependent IO (e.g. audio and video systems) make use of
the event loop abstraction. In this abstraction, event handlers are registered
with the system, and then a command is issued to run a loop which detects events
and runs the handlers, and uses the results of the handlers to render the
appropriate output. 

We would like for the FRP system to be easy to integrate with such IO systems,
while being flexible enough to enable its use with other forms of IO systems,
such as simple imperative systems, threaded systems, or network servers.

\section{Semantics}
\label{section:System_Design_and_Interface-Semantics}

A rigorous and formal elucidation of the semantics of signal-function FRP remains
unattempted, but there is a sufficient change to the practical semantics of
signal-function FRP between previous signal-function systems and TimeFlies to warrant
some description.

In previous systems such as Yampa, events were understood (and typed) as option-valued
signals. This approach is undesirable for several reasons. The most pressing reason is
that it prohibits push-based evaluation of events, because events are embedded in the
samples of a signal and must be searched for.

Another concern is that this approach limits the event rate to the sampling rate.
The rate of sampling should, at some level, not matter to the FRP system. Events which
occur between sampling intervals are never observed by the system.

This concern drives the next concern. Events are not instantaneous in this formulation.
If a signal is option valued, the sampling instant must fall within the interval where
there is an event occurrence present for that event to be observed. If events are
instantaneous, the probability of observing an event occurrence is zero.

Therefore, TimeFlies employs the N-Ary FRP type formulation to represent signals and
events as distinct entities in the inputs and outputs of signal functions. This means
we are now free to choose our representation of events, and to separate it from the
representation and evaluation of signals.

This freedom yields the additional ability to make events independent of the sampling
interval altogether. The semantics of event handling in TimeFlies is that an event occurrence
is responded to immediately, and does not wait for the next sampling instant. This allows events
to be instantaneous, and further, allows multiple events to occur within a single sampling interval.

There are two tradeoffs intrinsic to this approach. The first is that events are only partially ordered
temporally. There is no way to guarantee the order of observation of event occurrences occurring in the
same sampling interval. Further, the precise time of an event occurrence cannot be observed, only the 
time of the last sample prior to the occurrence.

In return for giving up total ordering and precise observation of the timing of events, we obtain the
ability to employ push-based evaluation for event occurrences, and the ability to non-deterministically
merge event occurrences. When events being input to a non-deterministic merge have simultaneous occurrences,
we simply select one arbitrarly to occur first. This does not violate any guarantee about time values, since
they will both have the same time value in either case, and does not violate any guarantee about ordering,
since no guarantee of their order is given.

A formal semantic description of signal function FRP would clarify the consequences of this decision somewhat,
but is outside the scope of this thesis.

\section{Types}
\label{section:System_Design_and_Interface-Types}

In a strongly and statically typed functional language, types are a key part of
an interface. Types provide a mechanism for describing and ensuring properties
of the interface's components and about the systems created with these
components. 

\subsection{Signal Vectors}
\label{subsection:System_Design_and_Interface-Types-Signal_Vectors}

In order to type signal functions, we must be able to describe their input and
output. In most signal function systems, a signal function takes exactly one
input and produces exactly one output. Multiple inputs or outputs are handled
by making the output a tuple, and combinators which combine or split the inputs
or outputs of a signal assume this. Events are represented at the type level
as a particular type of signal, and at the value level as an option, either an
event occurrence or not.

This method of typing excludes push-based evaluation at the outset.
It is not possible to construct a "partial tuple" nor in general is it possible
to construct only part of any type of value. Push-based evaluation depends on
evaluating only that part of the system which is updated, which means evaluating
only that part of the input which is updated.

In order to permit the construction of partial inputs and outputs, we make use
of signal vectors. Signal vectors are uninhabited types which describe the input
and output of a signal function. Singleton vectors are parameterized over the
type carried by the signal or by event occurrences. The definition of the signal
vector type is shown in Figure~\ref{figure:signal_vector_types}. 

Having an uninhabited signal vector type allows us to construct representations
of inputs and outputs which are hidden from the user of the system, and are
designed for partial representations.

\begin{figure}
\begin{code}
data SVEmpty    -- An empty signal vector component,
                -- neither event nor signal
data SVSignal a -- A signal, carrying values of type a
data SVEvent a  -- An event, whose occurrences carry values of type a
data SVAppend svLeft svRight -- The combination of the signal vectors
                             -- svLeft and svRight
\end{code}
\hrule
\caption{Signal vector types}
\label{figure:signal_vector_types}
\end{figure}

\subsection{Signal Functions}
\label{subsection:System_Design_and_Interface-Types-Signal_Functions}

The type constructor for signal functions is shown in
Figure~\ref{figure:signal_function_types}. For the {\tt init} parameter, only
one possible instantiation is shown. The usefulness of this type parameter,
along with another instantation which is hidden from users of the library,
is discussed in the section on implementation of signal functions
(Section~\ref{section:Implementation-Signal_Functions}).

Signal functions with signal vectors as input and output types form a
Haskell {\tt GArrow}~\cite{Megacz2011}. Specifically, the signal function
type constructor (with the initialization parameter fixed) forms the arrow
type, the {\tt SVAppend} type constructor forms the product type, and the
{\tt SVEmpty} type constructor forms the unit type.

The representation of signal functions is discussed in
Section~\ref{section:Implementation-Signal_Functions}. Here it suffices to say
that the use of GADTs permits the construction of values which make use of
uninhabited types as instantiations of type parameters.

The type synonyms {\tt :\textasciitilde>} and {\tt :\textasciicircum:} are included for readability and are
not crucial to the FRP system.

\begin{figure}
\begin{code}
-- Signal functions
-- init: The initialization type for 
-- the signal function, always NonInitialized
-- for exported signal functions
-- svIn: The input signal vector
-- svOut: The output signal vector
data SF init svIn svOut

data NonInitialized

type svIn :~> svOut = SF NonInitialized svIn svOut
type svLeft :^: svRight = SVAppend svLeft svRight
\end{code}
\hrule
\caption{Signal function types.}
\label{figure:signal_function_types}
\end{figure}

\subsection{Evaluation Monad}
\label{section:System_Design_and_Interface-Types-Evaluation_Monad}

A {\em monad} is a standard, composable abstraction for writing functions with
a context, used in Haskell for IO~\cite{PeytonJones1993,PeytonJones2001} among
other tasks. A monad is simply a 1-arity type constructor together with two
functions. The first function, {\tt return}, takes a value of type {\tt a} and
produces a value of type {\tt m a}, where {\tt m} is the type constructor. The
second, called {\tt bind} and stylized in the Haskell standard library as the
infix operator {\tt (>>=)}, takes a value of type {\tt m a} and a function
from {\tt a} to {\tt m b} and produces a value of type {\tt m b}. This allows
a value to be operated on out of the context and a new context to be assigned.

A monad can have other primitives which manipulate the context in some way. For 
instance, the primtives in Haskell's {\tt IO} monad produce actions which, when
interpreted as part of the {\tt main} action, produce some side-effect. The
{\tt State} monad provides {\tt get} and {\tt put} operations to work with a 
state value stored in the context.

Monad transformers~\cite{Jones1995} provide a means to combine the functionality
of multiple monads. A monad transformer is a monad with an extra type parameter.
This type parameter is instantiated with the type constructor of the underlying
monad, and an extra operation {\tt lift} is provided which converts values in
the underlying monad to values in the monad transformer.

The evaluation monad is a monad transformer. This permits it to be used in
conjunction with the {\tt IO} monad (or any other monad) to describe how input
is to be obtained for the signal function being evaluated, and how outputs are
to be handled.

The evaluation monad, in addition to the standard monad operators, provides a
means of {\em initializing} a signal function, and a means of translating the
monadic value describing evaluation to a value in the underlying monad. This
means, for instance, that we can obtain an action in the {\tt IO} monad to
evaluate a signal function.

The type of the evaluation monad must track the input type of the signal
function. The monad's context stores a mapping from outputs to handling actions.
An existential type can thus be used to ``hide'' the output type of the signal
function. However, inputs must come from external values, so the input type
cannot be hidden. There are thus three type parameters to the monad's type
constructor: the input signal vector, the type of the underlying monad, and the
monadic type parameter. The type is shown in Figure~\ref{figure:evaluation_monad_types}.

\begin{figure}
\begin{code}
-- A signal function's evaluations state
data SFEvalState svIn m
-- The evaluation monad
data SFEvalT svIn m a
\end{code}
\hrule
\caption{Evaluation monad types.}
\label{figure:evaluation_monad_types}
\end{figure}

\section{Combinators}
\label{section:System_Design_and_Interface-Combinators}

Signal functions are constructed from combinators, which are primitive signal
functions and operations to combine these primitives. These combinators are
grouped as basic signal functions, lifting operations for pure functions,
routing, reactivity, feedback, and time dependence.

\subsection{Basic Signal Functions}
\label{subsection:System_Design_and_Interface-Combinators-Basic_Signal_Functions}

The basic signal functions (Figure~\ref{figure:basic_signal_functions})
provide very simple operations. The {\tt identity} signal function, as expected,
simply copies its input to its output. The {\tt constant} signal function
produces the provided value as a signal at all times. The {\tt never} signal
function has an event output which never produces occurrences. The {\tt asap}
function produces an event occurrence with the given value at the first time
step after it is switched into the network. The {\tt after} function waits for
the specified amount of time before producing the event occurrence.

With the exception of {\tt identity}, all of the basic signal functions have
empty inputs. This allows these signal functions to be used to insert values
into the network which are known when the signal function is created, without
having to route those values from an input.

\begin{figure}
\begin{code}
-- Pass the input unmodified to the output
identity :: sv :~> sv

-- Produce a signal which is at all times the supplied value
constant :: a -> SVEmpty :~> SVSignal a

-- An event with no occurrences
never    :: SVEmpty :~> SVEvent a

-- An event with one occurrence, as soon as possible after
-- the signal function is initialized
asap     :: a -> SVEmpty :~> SVEvent a

-- An event after the specified amount of time has elapsed.
after    :: Double -> a -> SVEmpty :~> SVEvent a
\end{code}
\hrule
\caption{Basic signal functions.}
\label{figure:basic_signal_functions}
\end{figure}

\subsection{Lifting Pure Functions}
\label{subsection:System_Design_and_Interface-Combinators-Lifting_Pure_Functions}

Two combinators are provided to lift pure functions to signal functions (Figure~\ref{figure:lifting_pure_functions}).
The {\tt pureSignal} combinator applies the pure function to a signal at every
sample point. The {\tt pureEvent} combinator applies the function to each
occurrence of an input event.

\begin{figure}
\begin{code}
-- Apply the given function to a signal at all points in time
pureSignalTransformer :: (a -> b) -> SVSignal a :~> SVSignal b

-- Apply the given function to each event occurrence
pureEventTransformer  :: (a -> b) -> SVEvent a :~> SVEvent b
\end{code}
\hrule
\caption{Lifting pure functions.}
\label{figure:lifting_pure_functions}
\end{figure}

\subsection{Routing}
\label{subsection:System_Design_and_Interface-Combinators-Routing}

The routing combinators are used to combine signal functions, and
to re-arrange signal vectors in order to connect signal functions.
The routing combinators are shown in Figure~\ref{figure:routing_combinators}.

Only those combinators which modify or combine signal functions
({\tt (>>>)}, {\tt first}, {\tt second}) are reactive, and then
only if they inherit their reactivity from the signal function(s)
they modify. The rest do not react to or modify the input in any
way, except to re-arrange it, copy it, or discard it altogether.

\begin{figure}
\begin{code}
-- Use the output of one signal function as the input for another
(>>>) :: (svIn :~> svBetween) -> (svBetween :~> svOut) -> svIn :~> svOut

-- Pass through the right side of the input unchanged
first :: (svIn :~> svOut) -> (svIn :^: sv) :~> (svOut :^: sv)

-- Pass through the left side of the input unchanged
second :: (svIn :~> svOut) -> (sv :^: svIn) :~> (sv :^: svOut)

-- Swap the left and right sides
swap :: (svLeft :^: svRight) :~> (svRight :^: svLeft)

-- Duplicate the input
copy :: sv :~> (sv :^: sv)

-- Ignore the input
ignore :: sv :~> svEmpty

-- Remove an empty vector on the left
cancelLeft :: (SVEmpty :^: sv) :~> sv

-- Add an empty vector on the left
uncancelLeft :: sv :~> (SVEmpty :^: sv)

-- Remove an empty vector on the right
cancelRight :: (sv :^: SVEmpty) :~> sv

-- Add an empty vector on the right
uncancelRight :: sv :~> (sv :^: SVEmpty)

-- Make right-associative
associate :: ((sv1 :^: sv2) :^: sv3) :~> (sv1 :^: (sv2 :^: sv3))

-- Make left-associative
unassociate :: (sv1 :^: (sv2 :^: sv3)) :~> ((sv1 :^: sv2) :^: sv3)
\end{code}
\hrule
\caption{Routing combinators.}
\label{figure:routing_combinators}
\end{figure}

\subsection{Reactivity}
\label{subsection:System_Design_and_Interface-Combinators-Reactivity}

Reactivity is introduced by means of the {\tt switch} combinator
(Figure~\ref{figure:switch_combinator}). The design of this combinator
allows modular packaging of reactivity. A signal function can determine
autonomously when to replace itself, based only on its input and state,
by emitting an event occurrence carrying its replacement. The combinator
consumes and hides the event carrying the replacement signal function, 
so the reactivity is not exposed by the resulting reactive signal function.

\begin{figure}
\begin{code}
switch ::    (svIn :~> (svOut :^: SVEvent (svIn :~> svOut)))
          -> svIn :~> svOut
\end{code}
\hrule
\caption{Combinator for reactivity.}
\label{figure:switch_combinator}
\end{figure}

There are other formulations of a reactive combinator which may be implemented
using the one supplied. These are shown in Figure~\ref{figure:alternate_switching_combinators}
and may be provided in a future version of the TimeFlies library.

\begin{figure}
\begin{code}
-- Alternate version of switch,
-- implemented in terms of supplied version
switch_gen ::    (svIn :~> (svOut :^: SVEvent a))
              -> (a -> svIn :~> svOut)
              -> svIn :~> svOut
switch_gen sf f =
  switch (sf >>> second (pureEventTransformer f))

-- Supplied version in terms of alternate version
switch ::    (svIn :~> (svOut :^: SVEvent (svIn :~> svOut)))
          -> svIn :~> svOut
switch sf = switch_gen sf id

-- Repeated switch, which takes replacement signal functions
-- externally.
rswitch ::    (svIn :~> svOut)
           -> (svIn :^: SVEvent (svIn :~> svOut)) :~> svOut
rswitch sf =
  switch (first sf >>> second (pureEventTransformer rswitch))
\end{code}
\hrule
\caption{Alternate combinators for reactivity.}
\label{figure:alternate_switching_combinators}
\end{figure}

\subsection{Feedback}
\label{subsection:System_Design_and_Interface-Combinators-Feedback}

It is often useful for a signal function to receive a portion of its
own output as input. This is especially useful when we have two
signal functions which we would like to mutually interact. We cannot
just serially compose them, we must also bring the output of the second
back around to the first. Many signal-processing algorithms also depend
on feedback. The combinator which provides this ability is shown
in Figure~\ref{figure:feedback_combinator}. 

\begin{figure}
\begin{code}
loop ::    ((svIn :^: svLoop) :~> (svOut :^: svLoop))
        -> svIn :~> svOut
\end{code}
\hrule
\caption{Feedback combinator.}
\label{figure:feedback_combinator}
\end{figure}

This combinator provides decoupling for signals
(the input signal is the output signal at the previous time-step)
but not events (event occurrences are supplied to the combinator immediately).
This means that the programmer has the responsibility to ensure that feedback
does not generate an infinite sequence of events in a single time-step.

\subsection{Event-Specific Combinators}
\label{subsection:System_Design_and_Interface-Combinators-Event_specific_combinators}

Several combinators are provided for manipulating, suppressing, and generating events.
Output events, in these combinators, are triggered by input events.

Each of the combinators has an option variant and a list variant. The option variant
produces an output event occurrence whenever the application of the supplied function
to the input event produces a value. The list version produces an event occurrence for
each of the elements of the output list. The combinators are shown in Figure~\ref{figure:event_specific_combinators}.

\begin{figure}
\begin{code}
-- Apply the function to each input occurrence,
-- and produce an occurrence for each Just.
filter :: (a -> Maybe b) -> SVEvent a :~> SVEvent b

-- Apply the function to each input occurrence,
-- and produce an occurrence for each list element
filterList :: (a -> [b]) -> SVEvent a :~> SVEvent b

-- Apply the function to the stored accumulator
-- and the event occurrence, replacing the accumulator
-- and possibly outputting an occurrence
accumulate ::    (a -> b -> (Maybe c, a))
              -> a
              -> SVEvent b :~> SVEvent c

-- Apply the function to the stored accumulator
-- and the event occurrence, replacing the
-- accumulator and outputting an event occurrence
-- for each element of the list
accumulateList ::    (a -> b -> ([c]))
                  -> a
                  -> SVEvent b :~> SVEvent c
\end{code}
\hrule
\caption{Event-specific combinators.}
\label{figure:event_specific_combinators}
\end{figure}

The filter combinators are stateless, and thus apply the function to only the new
input value. They are useful for suppressing events, as well as for extracting one
of multiple cases of a datatype. For instance, a splitter for events carrying
{\tt Either}-valued occurrences could be written as:

\begin{code}
getLeft :: Either a b -> Maybe a
getLeft (Left x) = Just x
getLeft _ = Nothing

getRight :: Either a b -> Maybe b
getRight (Right x) = Just x
getRight _ = Nothing

split :: SVEvent (Either a b) :~> (SVEvent a :^: SVEvent b)
split = copy >>> first (filter getLeft) >>> second (filter getRight)
\end{code}

The accumulate combinators are stateful, applying the supplied function
to both the input value and an accumulator. This function has two results:
the option or list of output event occurrence values, and the new value
for the accumulator.

The accumulator is useful when responses to multiple event occurrences
(from one or more sources) must be coordinated. For instance, in the
benchmark application (see Chapter~\ref{chapter:Evaluation_and_Comparisons},
a table is maintained that allows knowledge from previous event occurrences
(packets from a network switch) to be used in deciding where the present
packet ought to go.

\subsection{Joining}
\label{subsection:System_Design_and_Interface-Combinators-Joining}

The joining combinators provide the ability to combine two event
streams, two signals, or a signal and an event stream. These
combinators are shown in Figure~\ref{figure:joining_combinators}

The {\tt union} combinator is a non-deterministic merge of event
streams. Any event which occurs on either input will occur
on the output. For simultaneous event occurrences, the order of occurrence
is not guaranteed, but the occurrence itself is. This construct
is also guaranteed to respect the relationship of event occurrences to sampling
intervals.

The {\tt combineSignals} combinator applies a binary function pointwise to two
signals, and produces the result of this application as a third signal.

The {\tt capture} combinator adds the last-sampled value of a signal at the time
of an event occurrence to that event occurrence.

These three combinators together provide the ability to combine elements of
signal vectors. By combining these combinators, arbitrary signal vectors can
be reduced.

\begin{figure}
\begin{code}
union          :: (SVEvent a :^: SVEvent a) :~> SVEvent a
combineSignals :: (a -> b -> c) -> (SVEvent a :^: SVEvent b) :~> SVEvent c
capture        :: (SVSignal a :^: SVEvent b) :~> SVEvent (b, a)
\end{code}
\hrule
\caption{Joining combinators.}
\label{figure:joining_combinators}
\end{figure}

\section{Evaluation}
\label{section:System_Design_and_Interface-Evaluator}

The evaluation interface provides a modified state monad which holds a signal
function, together with some additional information, as its state (shown in Figure~\ref{figure:evaluation_state}.
Rather than monadic instructions to put and get the state, the monad provides instructions
to trigger an input event, update an input signal, and trigger sampling of
signals in the signal function. Additional state includes the current set of
modifications to the input signals (since the last sample) and a set of
handlers which actuate effects based on output events or changes to the output
signal.

\begin{figure}
\begin{code}
-- A vector of handlers for outputs
data SVHandler out sv

-- A dummy handler for an empty output
emptyHandler    :: SVHandler out SVEmpty

-- A handler for an updated signal sample
signalHandler   :: (a -> out) -> SVHandler out (SVSignal a)

-- A handler for an event occurrence
eventHandler    :: (a -> out) -> SVHandler out (SVEvent a)

-- Combine handlers for a vector
combineHandlers ::    SVHandler out svLeft
                   -> SVHandler out svRight
                   -> SVHandler out (svLeft :^: svRight)

-- The state maintained when evaluating a signal function
data SFEvalState m svIn svOut

-- Create the initial state for evaluating a signal function
initSFEval ::    SVHandler (m ()) svOut
              -> SVSample svIn
              -> Double
              -> (svIn :~> svOut)
              -> SFEvalState m svIn svOut
\end{code}
\hrule
\caption{State maintained when evaluating a signal function}
\label{figure:evaluation_state}
\end{figure}

In order to initialize the state, the user must supply a set of handlers, the
signal function to evaluate, and initial values for all of the signal inputs
(Figure~\ref{figure:initial_input}).

\begin{figure}
\begin{code}
-- A sample for all leaves of a signal vector
data SVSample sv

-- Create a sample for a signal leaf
sample          :: a -> SVSample (SVSignal a)

-- A dummy sample for an event leaf
sampleEvt       :: SVSample (SVEvent a)

-- A dummy sample for an empty leaf
sampleNothing   :: SVSample SVEmpty

-- Combine two samples
combineSamples  ::    SVSample svLeft
                   -> SVSample svRight
                   -> SVSample (svLeft :^: svRight)
\end{code}
\hrule
\caption{Data type for initial input}
\label{figure:initial_input}
\end{figure}

This state can then be passed to a monadic action which will supply input to
the signal function. Inputs are constructed using a simple interface with
functions to construct sample updates and event occurrences, and to specify
their place in the vector (Figure~\ref{figure:ongoing_input}).

\begin{figure}
\begin{code}
-- Class to overload left and right functions
class SVRoutable r where
  svLeft          :: r svLeft -> r (svLeft :^: svRight)
  svRight         :: r svRight -> r (svLeft :^: svRight)

-- An input event occurrence
data SVEventInput sv
instance SVRoutable SVEventInput sv

-- An updated sample for a signal
data SVSignalUpdate sv
instance SVRoutable SVSignalUpdate sv

-- Create an event occurrence
svOcc           :: a -> SVEventInput (SVEvent a)

-- Create an updated sample
svSig           :: a -> SVSignalUpdate (SVSignal a)
\end{code}
\hrule
\caption{Data types for ongoing input.}
\label{figure:ongoing_input}
\end{figure}

The {\tt SFEvalT} monad is actually a monad transformer, that is, it is
parameterized over an underlying monad whose actions may be lifted to
{\tt SFEvalT}. In the usual case, this will be the {\tt IO} monad.

{\tt SFEvalT} actions are constructed using combinators to push events,
update inputs, and step time, as well as actions lifted from the underlying
monad (used to obtain these inputs). An action in the underlying monad
which produces the result and a new state is obtained with the {\tt runSFEvalT}
function. These combinators are shown in Figure~\ref{figure:evaluation_combinators}.

\begin{figure}
\begin{code}
-- The evaluation monad
data SFEvalT svIn svOut m a
instance MonadTrans (SFEvalT svIn svOut)
instance (Monad m) => Monad (SFEvalT svIn svOut m)
instance (Functor m) => Functor (SFEvalT svIn svOut m)
instance (Monad m, Functor m) => Applicative (SFEvalT svIn svOut m)
instance (MonadIO m) => MonadIO (SVEvalT svIn svOut m)

-- Obtain an action in the underlying monad
-- from an SFEvalT and a new state.
runSFEvalT ::    SFEvalT svIn svOut m a
              -> SFEvalState m svIn svOut
              -> m (a, SFEvalState m svIn svOut)

-- Push an event occurrence.
push :: (Monad m) => SVEventInput svIn -> SFEvalT svIn svOut m ()

-- Update the value of an input signal sample
-- (not immediately observed)
update :: (Monad m) => SVEventInput svIn -> SFEvalT svIn svOut m ()

-- Step forward in time, observing the updated signal values
step :: (Monad m) => Double -> SFEvalT svIn svOut m ()
\end{code}
\hrule
\caption{Evaluation combinators}
\label{figure:evaluation_combinators}
\end{figure}


\chapter{Implementation}
\label{chapter:Implemenation}

Having established our method of modeling reactivity, we proceed to a purely 
functional implementation of this method. For this implementation, we will make
significant use of certain advanced extensions to the Haskell type system.
\chapter{Example Application}
\label{chapter:Example_Application}

TimeFlies is a library for Functional Reactive Programming, which is a paradigm
for creating interactive and time-dependent applications. This chapter presents
the design of one such application, and its implementation using TimeFlies. The
application is an OpenFlow controller which implements a learning switch. In
short, it is a re-implementation of the standard kind of switch used in local
area networks, using ``software-defined networking.'' This
is the application which is benchmarked for performance comparisons in
Chapter~\ref{chapter:Evaluation_and_Comparisons}.

\section{OpenFlow}
\label{section:Example_Application-OpenFlow}

The OpenFlow protocol~\cite{OpenflowSpec} is a protocol for software-defined
networking applications. In particular, it defines the communication between
switches (devices which quickly route packets from input ports to output ports
based on learned rules) and {\em controllers}, which are generally devices with
large computational resources, such as servers. OpenFlow allows switches to
report packets for which no rule exists to a controller, and provides a means
for the controller to install new rules on a switch either preemptively, or in
response to a reported packet.

One of the simplests tasks which may be implemented as a OpenFlow controller is
a ``learning switch''. Such a switch uses the source and destination hardware
addresses in network packets to make routing decisions. A table is kept which
records the ports where source addresses are observed on incoming packets. This
table thus contains knowledge of which hardware address(es) can be reached on
which port. Using this table, rules are constructed to route packets with
particular source-destination pairs to the correct port. When a packet is seen
for which no rule exists, it is reported to the controller, which updates the
table and rules based on the new knowledge, and broadcasts the packet so that
it can reach its intended destination and receive a response.

Our example application is a controller for a ``learning switch.'' We describe
its implementation as a TimeFlies signal function, and two approaches to running
this signal function using the TimeFlies evaluation interface.

\section{Implementation}
\label{section:Example_Application-Implementation}

The first component for our learning switch is the table which maps addresses
to ports. This is a stateful data structure which will be updated by input
events and possibly produce output events. We employ the {\tt accumulateList}
combinator to produce the signal function shown in Figure~\ref{figure:switch_table_sf}.

\begin{figure}
\begin{code}
-- | Function type to modify a table and produce messages to the switch
type TableAccumulator = 
     SwitchTable 
  -> ([(SwitchHandle EthernetFrame,
        TransactionID,
        CSMessage)],
      SwitchTable)

-- | Empty map
M.empty :: M.Map k v

-- | Reverse application
rapp :: a -> (a -> b) -> b
rapp x f = f x

-- | Accumulate a switch table, producing output messages
--   based on the incoming functions
switchTable ::    SVEvent TableAccumulator
              :~> SVEvent (SwitchHandle EthernetFrame,
                           TransactionID,
                           CSMessage)
switchTable = accumulateList rapp M.empty
\end{code}
\hrule
\caption{Signal function for switch table.}
\label{figure:switch_table_sf}
\end{figure}

Note that the input events are closures which expect the table as input and
produce an updated table as output. This enables us to write several different
event sources whose final events require state (the table) and produce
messages to the switch.

The first source of such events is packets forwarded to the controller by
switches. The closures carried by these events implement the response to
packet inputs. The signal function from input switch message events to events
from the accumulator is shown in Figure~\ref{figure:packet_in_sf}, but the
details of the switch routing algorithm are elided.

\begin{figure}
\begin{code}
getPacketIn ::     SVEvent (SwitchHandle EthernetFrame,
                            TransactionID,
                            SCMessage EthernetFrame)
               :~> SVEvent (SwitchHandle EthernetFrame,
                            TransactionID,
                            PacketInfo EthernetFrame)

-- | Handle a packet in event
handlePacketIn ::    ((SwitchHandle EthernetFrame,
                       TransactionID,
                       PacketInfo EthernetFrame), Double)
                  -> TableAccumulator

packetIn ::     SVEvent ((SwitchHandle EthernetFrame,
                          TransactionID,
                          PacketInfo EthernetFrame),
                         Double)
            :~> SVEvent TableAccumulator
packetIn = pureEventTransformer handlePacketIn

-- | Capture the occurrence time of an event.
captureTime :: SVEvent a :~> SVEvent (a, Double)
captureTime = uncancelLeft >>>
              first time >>>
              capture

-- | Build TableAccumulator events from input packets.
handleSCMessage ::     SVEvent (SwitchHandle EthernetFrame,
                                TransactionID,
                                SCMessage EthernetFrame)
                   :~> SVEvent TableAccumulator
handleSCMessage = getPacketIn >>> captureTime >>> packetIn

\end{code}
\hrule
\caption{Signal function handling incoming packets.}
\label{figure:packet_in_sf}
\end{figure}

The {\tt captureTime} signal function attaches the time to incoming packets. The
{\tt getPacketIn} signal function extracts packet-in messages from the variety
of message events that a switch may send to the controller. These two signal
functions feed into the {\tt packetIn} signal function, which builds the
closures for the accumulator using the {\tt handlePacketIn} pure function.

The other task is to periodically eliminate all rules which are older than a
specified threshold. For this purpose, we construct another signal function
which produces {\tt TableAccumulator} events. This signal function runs a timer
(the {\tt every} combinator, which is implemented using {\tt switch} and {\tt after}
and produces an event repeatedly on a given interval) and produces events carrying closures which scan
the table for old rules, delete them, and produces messages to inform the
switches of their deletion. This signal function is shown in Figure~\ref{figure:clean_sf},
again eliding the implementation of the closures.

\begin{figure}
\begin{code}
-- | Take a function-value pair and return the application.
pairApp :: (a -> b, a) -> b
pairApp (f, x) = f x

-- | Alter a table and generate messages to remove expired entries and rules
cleanTable :: Double -> Double -> TableAccumulator


-- | Produce an event with the rule-cleaning accumulator.
cleanRules :: Double -> SVEmpty :~> SVEvent TableAccumulator
cleanRules t = copy >>>
               second (every t cleanRulesAcc) >>>
               first time >>>
               capture >>>
               pureEventTransformer pairApp
  where 
    cleanRulesAcc :: Double -> TableAccumulator
    cleanRulesAcc nowT table = cleanTable t nowT table
\end{code}
\hrule
\caption{Table-cleaning signal function.}
\label{figure:clean_sf}
\end{figure}

We now have all of the pieces for a signal function implementing a learning
switch, as shown in Figure~\ref{figure:learn_sf}.

\begin{figure}
\begin{code}
-- | A constant, for how long rules may survive
tableTime :: Double

-- | Make IO actions to send each packet.
sendCSMessage ::     SVEvent (SwitchHandle EthernetFrame,
                              TransactionID,
                              CSMessage)
                 :~> SVEvent (IO ())

-- | The learning switch
learn ::     SVEvent (SwitchHandle EthernetFrame,
                      TransactionID,
                      SCMessage EthernetFrame)
         :~> SVEvent (IO ())
learn = uncancelLeft >>> 
        first (cleanRules tableTime) >>>
        second handleSCMessage >>>
        union >>>
        switchTable >>>
        sendCSMessage
\end{code}
\hrule
\caption{The signal function for a learning switch.}
\label{figure:learn_sf}
\end{figure}

The learning switch composes the {\tt cleanRules} and {\tt handleSCMessage}
signal functions in parallel, but routes input only to {\tt handleSCMessage},
by using the {\tt uncancelLeft} routing combinator. The event streams are
merged and fed to the {\tt switchTable} signal function, and the resulting
packets are fed to the {\tt sendCSMessage} signal function which constructs
{\tt IO ()} events.

Note that this approach enables a modular design where each task is implemented
in its own signal function. Additional behaviors could be coded as additional
signal functions, whose events were also routed to the table accumulator.

Finally, we show how the signal function is interfaced with IO code using the
evaluation interface. First, we need a few extra definitions. For starters,
a controller must also handle switches when they first connect. We create a
separate signal function for this purpose, and compose it in parallel with our
learning switch signal function. We also need a definition of which port to
listen on, and an IO action for reading the time. These are shown in
Figure~\ref{figure:example_io_prelim}.

\begin{figure}
\begin{code}

-- | The port for the server to listen on.
port :: Word16

-- | Get the current time from the monotonic timer
getMonoTimeAsDouble :: IO Double

-- | Handle the connection of a new switch          
switchConnect ::     SVEvent (SwitchHandle EthernetFrame)
                 :~> SVEvent (IO ())
switchConnect = pureEventTransformer (void . handshake)

-- | Run switch connection and learning concurrently
learningSwitch ::     (SVEvent (SwitchHandle EthernetFrame) :^:
                       SVEvent (SwitchHandle EthernetFrame,
                                TransactionID,
                                SCMessage EthernetFrame))
                  :~> SVEvent (IO ())
learningSwitch = first switchConnect >>> second learn >>> union
\end{code}
\hrule
\caption{Evaluation preliminaries.}
\label{figure:example_io_prelim}
\end{figure}

Now we can create the main procedure for our learning switch. Note that
the code for placing an event on the input of a signal function is a simple
monadic sequence, sandwich between {\tt IORef} reads and writes of an opaque
value (Figure~\ref{figure:example_io}).

\begin{figure}
\begin{code}
-- | Entry point
main :: IO ()
main = do (argS:_) <- getArgs
          let sampleTime = read argS
          time <- getMonoTimeAsDouble
          reactiveRef <- newIORef $ initSFEval 
                                      (eventHandler id)
                                      (combineSamples sampleEvt sampleEvt)
                                      time
                                      learningSwitch
          ofEventManager <- 
            openFlowEventManager
              Nothing
              port 
              -- Switch addition inputs
              (\handle -> do rState <- readIORef reactiveRef
                             ((), rState') <- runSFEvalT 
                                                (push $ svLeft $ svOcc handle)
                                                rState
                             writeIORef reactiveRef rState')
              -- Switch message inputs:
              (\((tid, msg), handle) -> 
                 do rState <- readIORef reactiveRef
                    ((), rState') <- runSFEvalT 
                                       (push $ svRight $
                                        svOcc (handle, tid, msg))
                                       rState
                    writeIORef reactiveRef rState')
          let evtMgr = getEventManager ofEventManager
              -- Sampling:
              sample = void $ registerTimeout evtMgr sampleTime $ 
                         do rState <- readIORef reactiveRef
                            time <- getMonoTimeAsDouble
                            ((), rState') <- runSFEvalT (step time) rState
                            writeIORef reactiveRef rState'
                            sample
          sample 
          loop evtMgr      
\end{code}
\hrule
\caption{Using the evaluation interface for IO.}
\label{figure:example_io}
\end{figure}

\section{Comparison of Implementations}
\label{section:comparison_of_implementations}
\chapter{Conclusions and Further Work}
\label{chapter:Conclusions_and_Further_Work}
I have presented TimeFlies, a push-based signal-function FRP system. I have
demonstrated that TimeFlies does realize the theoretical benefits of a
push-based signal-function system.

\section{Conclusions}
\label{section:Conclusions_and_Further_Work-Conclusions}
The TimeFlies system is a fully-implemented and -documented FRP library which
may be extended with utility functions and further optimized. Its performance in
responding to events is demonstrated to be superior to that of the predominant
pull-based signal-function FRP library.

Further, the model of events used by TimeFlies subverts problematic semantic
questions about the evaluation of events in an FRP system. By using the N-Ary
FRP type model and separating the evaluation of events from the time steps used
for signals, TimeFlies fully supports non-deterministic merging of events,
and provides a semantic guarantee that events are not ''lost'' during evaluation.

TimeFlies includes a fully-documented evaluation interface which clarifies and
simplifies the task of integrating the TimeFlies system with the many IO systems
available to Haskell programmers.

\section{Further Work}
\label{section:Conclusions_and_Further_Work-Further_Work}
The TimeFlies system would benefit from attentive microbenchmarking and performance
tuning, as well as optimizations to avoid evaluating irrelevant parts of the network
during the the evaluation of time steps. A formal semantic justification for the
formulation of event evaluation (which would require a full formal semantics of FRP)
would enable a far more robust correctness argument, as well as providing a basis for
semantic extensions to signal-function FRP.

FRP is not yet mature, and has not been the subject of focused application development.
Thus, there is a dearth of design patterns for FRP applications. Such design patterns
would yield necessary feedback as to which generalizations and restrictions of FRP would
be appropriate and useful, and clarify the necessity of various utility combinators to
be included in the standard libraries of FRP systems.

In order to improve the performance of FRP yet further, it may be productive to attempt
to introduce parallel evaluation into FRP, taking advantage of the functional purity in
the implementation of the signal function combinators. This may involve, for instance,
evaluating several time-steps at once in a data-parallel manner,task parallelism
between different branches of a signal function, or speculative evaluation of switch
combinators.

Many classes of reactive application would benefit from a ``dynamic collections'' combinator
similar to {\tt pSwitch} in Yampa. Such a combinator allows a collection of signal functions
to be evaluated as one signal function, with addition or removal of signal functions instead
of the total replacement given by the {\tt switch} combinator. This is useful, for instance,
when simulating objects for games or computer models, as the behavior of each object can be
modeled as a signal function, and these signal functions can be added to and removed from the
collection.

The TimeFlies system provides a principled and performant system for future experimentation
on FRP, as well as implementation of FRP applications.



%%%%%%%%%%%%%%%%%%%%%%%%%%%%%%%%%%%%%%%%%%%%%%%%%%%%%%%%%%%%%%%%%%%%%%
% Appendix/Appendices                                                %
%%%%%%%%%%%%%%%%%%%%%%%%%%%%%%%%%%%%%%%%%%%%%%%%%%%%%%%%%%%%%%%%%%%%%%
%
% If you have only one appendix, use the command \appendix instead
% of \appendices.
%
\appendix
\index{Appendices@\emph{Appendices}}%

\chapter{Haskell Concepts}
\label{chapter:Haskell_Concepts}

One of the primary attractions of the Haskell language, and the reason for its
use throughout this work, is its advanced yet practical and usable type system.
This type system enables the use of compositional software design that would
be rendered infeasible without a type system to both inform and verify
composition and implementation. This appendix gives an overview of Haskell
concepts, design patterns, and idioms which are used in this thesis.

\section{Datatypes and Pattern Matching}
\label{section:Haskell_Concepts-Datatypes_and_Pattern_Matching}

In Haskell, new type constructors are introduced by defining Algebraic Datatypes.
An ADT declaration can take one of two forms. The first is a {\tt data}
declaration, e.g.:

\begin{code}
data Bool = True | False
data Maybe a = Just a | Nothing
\end{code}

In this form the identifier(s) preceding the {\tt =} character are a type
constructor followed by zero or more type variables. Following the {\tt =} 
character, and separated by {\tt |} characters, are data constructors. Each data
constructor is an identifier followed by zero or more types, which are the types
of its arguments. Data constructors can be used in expressions to construct a
value of this newly-declared type constructor, and in pattern matching to
``take apart'' the value and observe its components (the arguments to the data
constructor.

The second form is the {\tt newtype} declaration. This form is more restricted.
It is limited to one data constructor with exactly one argument. It introduces
a new type without introducing a new runtime restriction, though the Haskell
code must still use the data constructor in pattern matches and expressions
to explicitly coerce between the new type and the type of its data constructor's
parameter. This behavior is most often used to hide implementation types without
introducing the runtime overhead of value construction and pattern-matching, as
these are erased for constructors declared using {\tt newtype} once type-checking
is complete.

Once a type constructor has been introduced, it can be used in a type, with
its arguments replaced by any valid Haskell type. For instance:

\begin{code}
not      :: Bool -> Bool
cbool    :: Bool -> Int
maybeInt :: Int -> Maybe Int -> Int
isJust   :: Maybe a -> Bool
mapMaybe :: (a -> b) -> Maybe a -> Maybe b
\end{code}

Its data constructors can be used in the patterns of functions, and in their
expressions. For instance:

\begin{code}
not :: Bool -> Bool
not True  = False
not False = True

isJust :: Maybe a -> Bool
isJust (Just _) = True
isJust Nothing  = False

mapMaybe :: (a -> b) -> Maybe a -> Maybe b
mapMaybe f (Just x) = Just (f x)
mapMaybe _ Nothing  = Nothing
\end{code}

\subsection{Generalized Algebraic Datatypes}
\label{subsection:Haskell_Concepts-Datatypes_and_Pattern_Matching-Generalized_Algebraic_Datatypes}

GADTs~\cite{Cheney2003,Xi2003} permit us to specify what types fill in the type
parameters for specific constructors. For instance, if we wish to build a tiny
expression language, we can use a standard ADT:

\begin{code}
data Exp a = Const a 
           | Plus (Exp a) (Exp a)
           | LessThan (Exp a) (Exp a)
           | If (Exp a) (Exp a) (Exp a)
\end{code}

Let us assume for the moment that Haskell exports functions\footnote{The types
for addition and comparison actually involve a typeclass constraint, but the
point is that the functions' types are not parametric.}:

\begin{code}
(+)  :: Int -> Int -> Int
(<)  :: Int -> Int -> Bool
\end{code}

An attempt to write an evaluation function for our expression type is:

\begin{code}
eval :: Exp a -> a
eval (Const x)  = x
eval (Plus x y) = eval x + eval y
eval (LessThan x y) = eval x < eval y
eval (If p c a) = if (eval p)
                  then eval c
                  else eval a
\end{code}

But this function will not typecheck. In the {\tt Plus} case, we cannot
assume that the argument to the type constructor {\tt Exp} typing {\tt x} or {\tt y}
is {\tt Int}, and similarly for the {\tt LessThan} case. Again in the {\tt If}
case, we cannot assume that the predicate is of type {\tt Exp Bool}, and if we
could, that would force our results to be of type {\tt Exp Bool} as well.

Let's try a slightly modified ADT:

\begin{code}
data Exp a = Const a
           | Plus (Exp Int) (Exp Int)
           | LessThan (Exp Int) (Exp Int)
           | IfThenElse (Exp Bool) (Exp a) (Exp a)
\end{code}

Here the motivation for a type parameter to our type constructor becomes clear.
We can introduce both {\tt Bool} and {\tt Int} constants (as well as others, but
we cannot do anything with them unless we extend the language). Further, we
can constrain the types of the input expressions to each of our constructors to
be of the appropriate type.

The code for our our evaluator is the same. But now note that the in the
{\tt Plus} and {\tt LessThan} cases, even though the input types are compatible
with the functions used, the output type expected from our function is not.
{\tt Plus x y} has type {\tt Exp a} in the pattern match, so our output is
expected to be of type {\tt a} for any {\tt a} argument to type of an 
input {\tt Exp}.

Here is our expression type as a GADT:
\begin{code}
data Exp a where
  Const    :: a                               -> Exp a
  Plus     :: Exp Int  -> Exp Int             -> Exp Int
  LessThan :: Exp Int  -> Exp Int             -> Exp Bool
  If       :: Exp Bool -> Exp a   -> Exp a    -> Exp a
\end{code}

Now our evaluation function can typecheck. Each constructor is able to constrain
the type parameter for output, not just its arguments. So when pattern matching
on the {\tt Plus} case, we know that each of our inputs will be of type
{\tt Exp Int}, and that the type of the expression we are pattern matching has
type {\tt Exp Int}, so the output from our function can be constrained to type
{\tt Int}, similarly for {\tt LessThan}. The type argument to {\tt Exp} in the
type of the {\tt If} is constrainted to be the same as that as the argument
to the types of the consequent and alternate to the conditional. This permits
our {\tt If} statement to be parametric while still allowing our evaluation to
typecheck.

This capacity to constrain the output types of data constructors, and thus, to
constrain the types of expressions in the scope of pattern matches of these data
constructors, is called {\em type refinement}. We will make use of this ability
to parameterize concrete datatypes over abstract type structures, rather than to
permit typechecking in specific cases, but the principle is the same.

\section{Typeclasses}
\label{section:Haskell_Concepts-Typeclasses}

Typeclasses in Haskell provide a means to implement functions that are openly
polymorphic while not being parametric. A typeclass is declared as follows:

\begin{code}
class Show t where
  show :: t -> String
\end{code}

A typeclass has an identifier and a single type parameter. This type parameter
is used in the type of one or more functions which are members of the class.

The class can then be instantiated:

\begin{code}
instance Show Bool where
  show True  = "True"
  show False = "False"

instance (Show a) => Show (Maybe a) where
  show (Just x) = "Just " ++ show x
  show Nothing  = Nothing
\end{code}

Functions can now be written polymorphically over the types instantiating the
typeclass, by including the typeclass as a constraint:

\begin{code}
repL :: (Show t) => t -> Int
repL x = length (show x)
\end{code}

Typeclasses are used in Haskell to provide common interfaces or functionality
across types. The {\tt Show} class used as an example is exported, along with
instances for most of the basic Haskell types, from Haskell's Prelude (the
standard module imported into every Haskell module). 

\section{Monads and Monad Transformers}
\label{section:Haskell_Concepts-Monads_and_Monad_Transformers}

One of the primary concepts employed in Haskell programs is that of the
monad~\cite{PeytonJones1993,PeytonJones2001}. The concept of the monad is
borrowed from category theory, but it is quite simple when understood within
Haskell. A monad is a type constructor with a single parameter, and two
associated functions. In Haskell's {\tt Monad} typeclass, these functions are
denoted {\tt return} and {tt (>>=)}.

\begin{code}
class Monad m where
  return :: a -> m a
  (>>=)  :: m a -> (a -> m b) -> m b
\end{code}

A monad must obey the following axioms:
\begin{itemize}
\item Left identity: {\tt return x >>= f = f x}
\item Right identity: {\tt m >>= return = m}
\item Associativity: {\tt (a >>= b) >>= c) = (a >>= (\\x -> b x >>= c))}
\end{itemize}

Several standard Haskell type constructor are monads in interesting ways, but
the most well-known is Haskell's {\tt IO} type constructor. This is the basis of
Haskell's input/output system. The entry point to a Haskell program is the
{\tt main} function, of type {\tt IO ()}. This function can be constructed by
using the monadic functions to sequence an arbitrary number of other functions
whose output type is {\tt IO a}. Since the sequencing operator takes an
arbitrary function, this allows the full power of Haskell functions, including
first-class and higher-order functions, to be employed in defining a program's
input and output. A convenience function is commonly used when the result is not
necessary as part of the sequencing:

\begin{code}
(>>) :: Monad m => m a -> m b -> m b
(>>) m1 m2 = m1 >>= (const m2)

\subsection{Do-notation}
\label{subsection:Haskell_Concepts-Monads_and_Monad_Transformers-Do_notation}

Because monads are such a pervasive concept in Haskell, the language includes
special syntax for writing monadic expressions. Do-notation is expression syntax
which begins with the keyword {\tt do} and is followed by lines of two forms:

\begin{code}
do
  x <- m1
  m2
\end{code}

The first form is a binding expression: it binds the variable {\tt x} to the
output of the monadic value {\tt m2}. The second form simply sequences the monad
value {\tt m2} into the monadic value being built. Do-notation has a syntax-driven
translation to desugared Haskell expression:

\begin{code}
desugar {
do x <- m1
   ...} = 
m1 >>= \x -> desugar {do ...}

desugar {
do m1
   ...} =
m1 >> desugar {do ...}
\end{code}

\subsection{Monad Transformers}
\label{subsection:Haskell_Concepts-Monads_and_Monad_Transformers-Monad_Transformers}
A monad transformer is a type constructor with two parameters. The first
parameter parameterizes over a one-parameter type constructor, rather than a
type. The second is the monadic type parameter. A type constructor {T} is a monad
transformer if it has the following instance of the Monad typeclass:

\begin{code}
instance Monad m => Monad T m
\end{code}

and is also an instance of the class

\begin{code}
class MonadTrans t where
  lift :: (Monad m) => m a -> t m a
\end{code}

The axioms for the lift function are:
\begin{itemize}
\item {\tt lift . return = return}
\item {\tt lift (m >>= f) = lift m >>= (lift . f)}
\end{itemize}

Restated, {\tt lift} does not modify return and distributes over monadic
sequencing.

As an example of a monad transformer, we can consider the {\tt StateT} type,
which is employed in the implementation of this thesis.

The type is declared:
\begin{code}
newtype StateT s m a = S { runStateT :: s -> m (s, a) }
\end{code}

Its monad instance, for any {\tt s}, is

\begin{code}
instance Monad m => Monad (StateT s m) where
  return x  = S (\s -> return (x, s))
  (>>=) (S f) mf = S (\s -> f s >>= (\ (x, s') -> let (S f') = mf x in f' s ))
\end{code}

The return and sequencing functions carry the state through the underlying monad.

The {\tt MonadTrans} instance is:
\begin{code}
instance MonadTrans (StateT s) where
  lift m = S (\s -> m >>= (\x -> return (x, s))
\end{code}

Finally, there are two functions provided to access and set the state:

\begin{code}
put :: s -> StateT s m ()
put = S (\_ -> return ((), s'))

get :: StateT s m s
get = S (\s -> return (s, s))
\end{code}

If we use {\tt StateT} as a wrapper around the IO monad, we might employ it as
a way to generate a unique line number for each "putStrLn" we call.

\begin{code}
putStrLnN :: String -> StateT Int IO ()
putStrLnN s = do i <- get
                 put (i + 1)
                 lift (putStrLn (show i ++ " " ++ s))

main = runStateT mainSt 1
  where mainSt = do g <- lift getLine
                    putStrLnN g
                    mainSt
\end{code}

\chapter{Glossary of Type Terminology}
\label{chapter:Haskell_Concepts-Glossary_of_Type_Terminology}
\begin{description}
\item[ADT] See {\em Algebraic Datatype}.

\item[Algebraic Datatype] An Algebraic Datatype or ADT is a type whose terms
are {\em data constructors}. An ADT is defined by naming a
{\em type constructor} and its parameters (zero or more)
(as {\em type variables}), together with one or more data constructors and the
types of their members. Each data constructor takes a fixed number
(zero or more) data members, whose types are given following the constructor
name. These types are defined in terms of the type variables named as parameters
of the type constructor and any type constructors (including the type
constructor associated with the ADT) in scope in the module.

\item[Data Constructor] A Data Constructor is a component of an
Algebraic~Datatype which, when applied to values of the appropriate type, 
creates a value typed with the ADT. Data constructors are the primary element
which may be pattern matched in languages such as Haskell.

\item[GADT] See {\em Generalized Algebraic Datatype}.

\item[Generalized Algebraic Datatype] Similar to an {\em Algebraic Datatype},
but {\em type variables} in the {\em type constructor} declaration serve merely
to denote the number of type parameters (and thus may be replaced by a
{\em kind signature}) and types are given for each {\em data constructor}. These
types must have the type constructor as their top-level term, but may fill in
the parameters of the type constructor with arbitrary types. Variables which
appear in the data member types but on in the data constructor type are
{\em existentially quantified}, and types appearing in the data constructor
type but not the data member types may be instantiated arbitrarily.

\item[Kind] A ``type of types.'' Kinds are used to verify that types are
consistent during typechecking. The kind of types which contain values is
{\tt *}, and the kind of single-parameter type constructors which take a
type is {\tt * -> *}. Other kinds may also be introduced. For instance,
signal vectors should be their own separate kind, but the Haskell type mechanism
was not mature enough to support this at the time of this writing..

\item[Kind Signature] A means of specifying the number and kind of types which
may instantiate type variables. Type variables in Haskell are not restricted to
types, but may be instantiated by type constructors as well. The kind of a
variable restricts what it may be instantiated with. A kind signature gives
kinds to a type constructor, and thus to its parameters. Specifying the kind
of a type constructor perfectly constrains the number of parameters.

\item[Type Constructor] A type constructor is a type level term which, when
applied to the proper number of types, produces a type. Type constructors,
together with {\em type variables}, form the basis of polymorphism in Haskell
and similar languages.

\item[Type Variable] A type variable is a type-level term which may be 
instantiated (by the typechecker via inference, or by the user via annotation)
with any type at the point where the value so typed is used. 
Together with {\em type constructors}, type variables form the basis of
polymorphism in Haskell and similar languages.
\end{description}


%\include{chapter-appendix2}

%\include{chapter-appendix3}


%%%%%%%%%%%%%%%%%%%%%%%%%%%%%%%%%%%%%%%%%%%%%%%%%%%%%%%%%%%%%%%%%%%%%%
% Generate the bibliography.                         %
%%%%%%%%%%%%%%%%%%%%%%%%%%%%%%%%%%%%%%%%%%%%%%%%%%%%%%%%%%%%%%%%%%%%%%
%                                    %
% NOTE: For master's theses and reports, NOTHING is permitted to     %
%   come between the bibliography and the vita. The command      %
%   to generate the index (if used) MUST be moved to before      %
%   this section.                            %
%                                    %
%
%\nocite{*}      % This command causes all items in the           %
                % bibliographic database to be added to          %
                % the bibliography, even if they are not         %
                % explicitly cited in the text.              %
        %                            %
  % Here the bibliography           %
        % is inserted.                %

\index{Bibliography@\emph{Bibliography}}%
\bibliographystyle{plain}
\bibliography{thesis}
%%%%%%%%%%%%%%%%%%%%%%%%%%%%%%%%%%%%%%%%%%%%%%%%%%%%%%%%%%%%%%%%%%%%%%


%%%%%%%%%%%%%%%%%%%%%%%%%%%%%%%%%%%%%%%%%%%%%%%%%%%%%%%%%%%%%%%%%%%%%%
% Generate the index.                            %
%%%%%%%%%%%%%%%%%%%%%%%%%%%%%%%%%%%%%%%%%%%%%%%%%%%%%%%%%%%%%%%%%%%%%%
%                                    %
% NOTE: For master's theses and reports, NOTHING is permitted to     %
%   come between the bibliography and the vita. This section     %
%   to generate the index (if used) MUST be moved to before      %
%   the bibliography section.                    %
%                                    %
%\printindex%    % Include the index here. Comment out this line      %
%       % with a percent sign if you do not want an index.   %
%%%%%%%%%%%%%%%%%%%%%%%%%%%%%%%%%%%%%%%%%%%%%%%%%%%%%%%%%%%%%%%%%%%%%%


%%%%%%%%%%%%%%%%%%%%%%%%%%%%%%%%%%%%%%%%%%%%%%%%%%%%%%%%%%%%%%%%%%%%%%
% Vita page.                                 %
%%%%%%%%%%%%%%%%%%%%%%%%%%%%%%%%%%%%%%%%%%%%%%%%%%%%%%%%%%%%%%%%%%%%%%
\begin{vita}
Edward Amsden was born in Dayton, Ohio in the year 1990, to Andrew and
Vivian Amsden. He is pursuing concurrent B.S. and M.S. degrees at the Rochester
Institute of Technology. Once his M.S. is completed, he plans to begin his
Ph.~D. at Indiana University. His research interests include functional
programming languages, concurrency and parallelism, computer graphics, and
computer audio. 
\end{vita}
\end{document}
