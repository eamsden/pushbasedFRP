\chapter{System Design and Interface}
\label{chapter:System_Design_and_Interface}

\section{Goals}
\label{section:System_Design_and_Interface-Goals}

The goal of FRP is to provide an efficient, declarative abstraction for creating
reactive programs. Towards this overall goal, there are three goals which this
system is intended to meet.

\subsection{Efficient Evaluation}
\label{subsection:System_Design_and_Interface-Goals-Efficient_and_Push_based_Evaluation}

Efficient evaluation is the motivation for push-based evaluation of events.
Since FRP programs are expected to  interact with an outside world in real time,
efficiency cannot simply be measured by the runtime of a program. Thus, when speaking of efficiency,
we are expressing a desire that the system utilize as few system resources as possible
for the task at hand, while responding as quickly as possible to external inputs and
producing output at a consistently high sample rate.

\subsection{Composability}
\label{subsection:System_Design_and_Interface-Goals-Composability}

A composable abstraction is one in which values in that abstraction may be
combined in such a way that reasoning about their combined actions involves
little more than reasoning about their individual actions. In a signal function
system, the only interaction between composed signal functions ought to be that
the output of one is the input of another. Composability permits a particularly
attractive form of software engineering in which successively larger systems are
created from by combining smaller systems, without having to reason about the 
implementation of the components of the systems being combined.

\subsection{Simple Integration}
\label{subsection:System_Design_and_Interface-Goals-Ease_of_Integration}

It is fine for a system to be composable with regards to itself, but an FRP
system must interact with the outside world. Since we cannot anticipate every
possible form of input and output that the system will be asked to interact
with, we must interface with Haskell's IO system. In particular, most libraries
for user interaction (e.g. GUI and graphics libraries such as GTK+ and GLUT) and
most libraries for time-dependent IO (e.g. audio and video systems) make use of
the event loop abstraction. In this abstraction, event handlers are registered
with the system, and then a command is issued to run a loop which detects events
and runs the handlers, and uses the results of the handlers to render the
appropriate output. 

We would like for the FRP system to be easy to integrate with such IO systems,
while being flexible enough to enable its use with other forms of IO systems,
such as simple imperative systems, threaded systems, or network servers.

\section{Semantics}
\label{section:System_Design_and_Interface-Semantics}

A rigorous and formal elucidation of the semantics of signal-function FRP remains
unattempted, but there is a sufficient change to the practical semantics of
signal-function FRP between previous signal-function systems and TimeFlies to warrant
some description.

In previous systems such as Yampa, events were understood (and implemented) as
option-valued signals.\footnote{{\tt type Event a = Signal (Maybe a)}} This
approach is undesirable for several reasons. The most pressing reason is that it
prohibits push-based evaluation of events, because events are embedded in the
samples of a signal and must be searched for when sampling.

Another concern is that this approach limits the event rate to the sampling rate.
The rate of sampling should, at some level, not matter to the FRP system. Events
which occur between sampling intervals are never observed by the system. Since
the signal function is never evaluated except at the sampling intervals, no
events can otherwise be observed.

This concern drives the next concern. Events are not instantaneous in this formulation.
If a signal is option valued, the sampling instant must fall within the interval where
there is an event occurrence present for that event to be observed. If events are
instantaneous, the probability of observing an event occurrence is zero.

Therefore, TimeFlies employs the N-Ary FRP type formulation to represent signals and
events as distinct entities in the inputs and outputs of signal functions. This means
we are now free to choose our representation of events, and to separate it from the
representation and evaluation of signals.

This freedom yields the additional ability to make events independent of the sampling
interval altogether. The semantics of event handling in TimeFlies is that an event occurrence
is responded to immediately, and does not wait for the next sampling instant. This allows events
to be instantaneous, and further, allows multiple events to occur within a single sampling interval.

There are two tradeoffs intrinsic to this approach. The first is that events are only partially ordered
temporally. There is no way to guarantee the order of observation of event occurrences occurring in the
same sampling interval. Further, the precise time of an event occurrence cannot be observed, only the 
time of the last sample prior to the occurrence.

In return for giving up total ordering and precise observation of the timing of events, we obtain the
ability to employ push-based evaluation for event occurrences, and the ability to non-deterministically
merge event occurrences. When events being input to a non-deterministic merge have simultaneous occurrences,
we arbitrarily select one to occur first. This does not violate any guarantee about time values, since
they will both have the same time value in either case, and does not violate any guarantee about ordering,
since no guarantee of their order is given.

A formal semantic description of signal function FRP would clarify the consequences of this decision somewhat,
but is outside the scope of this thesis.

\section{Types}
\label{section:System_Design_and_Interface-Types}

In a strongly and statically typed functional language, types are a key part of
an interface. Types provide a mechanism for describing and ensuring properties
of the interface's components and about the systems created with these
components. 

\subsection{Signal Vectors}
\label{subsection:System_Design_and_Interface-Types-Signal_Vectors}

In order to type signal functions, we must be able to describe their input and
output. In most signal function systems, a signal function takes exactly one
input and produces exactly one output. Multiple inputs or outputs are handled
by making the input or output a tuple, and combinators which combine or split
the inputs or outputs of a signal assume this. Events are represented at the
type level as a particular type of signal, and at the value level as an option,
either an event occurrence or not.

This method of typing excludes push-based evaluation at the outset.
It is not possible to construct a ``partial tuple'' nor in general is it
possible to construct only part of any type of value. Push-based evaluation
depends on evaluating only that part of the system which is updated, which means
evaluating only that part of the input which is updated.

In order to permit the construction of partial inputs and outputs, we make use
of signal vectors. Signal vectors are uninhabited types which describe the input
and output of a signal function. Singleton vectors are parameterized over the
type carried by the signal or by event occurrences. The definition of the signal
vector type is shown in Figure~\ref{figure:signal_vector_types}. 

Having an uninhabited signal vector type allows us to construct representations
of inputs and outputs which are hidden from the user of the system, and are
designed for partial representations.

\begin{figure}
\begin{code}
data SVEmpty    -- An empty signal vector component,
                -- neither event nor signal
data SVSignal a -- A signal, carrying values of type a
data SVEvent a  -- An event, whose occurrences carry values of type a
data SVAppend svLeft svRight -- The combination of the signal vectors
                             -- svLeft and svRight
\end{code}
\hrule
\caption{Signal vector types.}
\label{figure:signal_vector_types}
\end{figure}

\subsection{Signal Functions}
\label{subsection:System_Design_and_Interface-Types-Signal_Functions}

The type constructor for signal functions is shown in
Figure~\ref{figure:signal_function_types}. For the {\tt init} parameter, only
one possible instantiation is shown. The usefulness of this type parameter,
along with another instantation which is hidden from users of the library,
is discussed in the section on implementation of signal functions
(Section~\ref{section:Implementation-Signal_Functions}).

Signal functions with signal vectors as input and output types form a
Haskell {\tt GArrow}~\cite{Megacz2011}. Specifically, the signal function
type constructor (with the initialization parameter fixed) forms the arrow
type, the {\tt SVAppend} type constructor forms the product type, and the
{\tt SVEmpty} type constructor forms the unit type.

The representation of signal functions is discussed in
Section~\ref{section:Implementation-Signal_Functions}. The type synonyms
{\tt :\textasciitilde>} and {\tt :\textasciicircum:} are included for
readability and are not crucial to the FRP system.

\begin{figure}
\begin{code}
-- Signal functions
-- init: The initialization type for 
--   the signal function, always NonInitialized
--   for exported signal functions
-- svIn: The input signal vector
-- svOut: The output signal vector
data SF init svIn svOut

data NonInitialized

type svIn :~> svOut = SF NonInitialized svIn svOut
type svLeft :^: svRight = SVAppend svLeft svRight
\end{code}
\hrule
\caption{Signal function types.}
\label{figure:signal_function_types}
\end{figure}

\subsection{Evaluation Monad Transformer}
\label{section:System_Design_and_Interface-Types-Evaluation_Monad_Transformer}

TimeFlies provides a monad transformer for evaluating signal functions. Thi
monad transformer may be used in conjunction with the {\tt IO} monad (or any
other monad) to describe how input is to be obtained for the signal function
being evaluated, and how outputs are to be handled.

The evaluation monad, in addition to the standard monad operators, provides a
means of {\em initializing} a signal function, and a means of translating the
monadic value describing evaluation to a value in the underlying monad. This
means, for instance, that we can obtain an action in the {\tt IO} monad to
evaluate a signal function.

The type of the evaluation monad must track the input and output type of the
signal function. The monad is a state monad, and its state is a record (the
{\tt SFEvalState} type) with fields for a mapping from outputs to handling
actions, the current state of the signal function, the last time the signal
function was sampled, and the currently-unsampled input. There are thus four
type parameters to the monad's type constructor: the input signal vector (to
type the signal function and unsampled input), the output signal vector (to type
the signal function and output handlers), the type of the underlying monad, and
the monadic type parameter. The type is shown in
Figure~\ref{figure:evaluation_monad_types}.

\begin{figure}
\begin{code}
-- A signal function's evaluations state
data SFEvalState svIn svOut m
-- The evaluation monad
data SFEvalT svIn svOut m a

instance Monad m => Monad (SFEvalT svIn svOut m)
\end{code}
\hrule
\caption{Evaluation monad types.}
\label{figure:evaluation_monad_types}
\end{figure}

\section{Combinators}
\label{section:System_Design_and_Interface-Combinators}

Signal functions are constructed from combinators, which are primitive signal
functions and operations to combine these primitives. These combinators are
grouped as basic signal functions, lifting operations for pure functions,
routing, reactivity, feedback, event processing, joining, and time dependence.

\subsection{Basic Signal Functions}
\label{subsection:System_Design_and_Interface-Combinators-Basic_Signal_Functions}

The basic signal functions (Figure~\ref{figure:basic_signal_functions})
provide very simple operations. The {\tt identity} signal function, as expected,
simply copies its input to its output. The {\tt constant} signal function
produces the provided value as a signal at all times. The {\tt never} signal
function has an event output which never produces occurrences. The {\tt asap}
function produces an event occurrence with the given value at the first time
step after it is switched into the network. The {\tt after} function waits for
the specified amount of time before producing an event occurrence.

With the exception of {\tt identity}, all of the basic signal functions have
empty inputs. This allows these signal functions to be used to insert values
into the network which are known when the signal function is created, without
having to route those values from an input.

\begin{figure}
\begin{code}
-- Pass the input unmodified to the output
identity :: sv :~> sv

-- Produce a signal which is at all times the supplied value
constant :: a -> SVEmpty :~> SVSignal a

-- An event with no occurrences
never    :: SVEmpty :~> SVEvent a

-- An event with one occurrence, as soon as possible after
-- the signal function is initialized
asap     :: a -> SVEmpty :~> SVEvent a

-- An event with one occurrence
-- after the specified amount of time has elapsed.
after    :: Double -> a -> SVEmpty :~> SVEvent a
\end{code}
\hrule
\caption{Basic signal function combinators.}
\label{figure:basic_signal_functions}
\end{figure}

\subsection{Lifting Pure Functions}
\label{subsection:System_Design_and_Interface-Combinators-Lifting_Pure_Functions}

Two combinators are provided to lift pure functions to signal functions
(Figure~\ref{figure:lifting_pure_functions}). The {\tt pureSignalTransformer}
combinator applies the pure function to a signal at every sample point. The
{\tt pureEventTransformer} combinator applies the function to each occurrence of
an input event.

\begin{figure}
\begin{code}
-- Apply the given function to a signal at all points in time
pureSignalTransformer :: (a -> b) -> SVSignal a :~> SVSignal b

-- Apply the given function to each event occurrence
pureEventTransformer  :: (a -> b) -> SVEvent a :~> SVEvent b
\end{code}
\hrule
\caption{Lifting pure functions.}
\label{figure:lifting_pure_functions}
\end{figure}

\subsection{Routing}
\label{subsection:System_Design_and_Interface-Combinators-Routing}

The routing combinators are used to combine signal functions, and
to re-arrange signal vectors in order to connect signal functions.
The routing combinators are shown in Figure~\ref{figure:routing_combinators}.

Only those combinators which modify or combine signal functions
({\tt (>>>)}, {\tt first}, {\tt second}) are reactive (may replace themselves in
response to events), and then only if they inherit their reactivity from the
signal function(s) which are inputs to combinator. The rest do not
react to or modify the input in any way, except to re-arrange it, copy it, or
discard it altogether.

\begin{figure}
\begin{code}
-- Use the output of one signal function as the input for another
(>>>) :: (svIn :~> svBetween) -> (svBetween :~> svOut) -> svIn :~> svOut

-- Pass through the right side of the input unchanged
first :: (svIn :~> svOut) -> (svIn :^: sv) :~> (svOut :^: sv)

-- Pass through the left side of the input unchanged
second :: (svIn :~> svOut) -> (sv :^: svIn) :~> (sv :^: svOut)

-- Swap the left and right sides
swap :: (svLeft :^: svRight) :~> (svRight :^: svLeft)

-- Duplicate the input
copy :: sv :~> (sv :^: sv)

-- Ignore the input
ignore :: sv :~> svEmpty

-- Remove an empty vector on the left
cancelLeft :: (SVEmpty :^: sv) :~> sv

-- Add an empty vector on the left
uncancelLeft :: sv :~> (SVEmpty :^: sv)

-- Remove an empty vector on the right
cancelRight :: (sv :^: SVEmpty) :~> sv

-- Add an empty vector on the right
uncancelRight :: sv :~> (sv :^: SVEmpty)

-- Make right-associative
associate :: ((sv1 :^: sv2) :^: sv3) :~> (sv1 :^: (sv2 :^: sv3))

-- Make left-associative
unassociate :: (sv1 :^: (sv2 :^: sv3)) :~> ((sv1 :^: sv2) :^: sv3)
\end{code}
\hrule
\caption{Routing combinators.}
\label{figure:routing_combinators}
\end{figure}

\subsection{Reactivity}
\label{subsection:System_Design_and_Interface-Combinators-Reactivity}

Reactivity is introduced by means of the {\tt switch} combinator
(Figure~\ref{figure:switch_combinator}). The design of this combinator
allows modular packaging of reactivity. A signal function can determine
autonomously when to replace itself, based only on its input and state,
by emitting an event occurrence carrying its replacement. The combinator
consumes and hides the event carrying the replacement signal function, 
so the reactivity is not exposed by the resulting reactive signal function.

\begin{figure}
\begin{code}
switch ::    (svIn :~> (svOut :^: SVEvent (svIn :~> svOut)))
          -> svIn :~> svOut
\end{code}
\hrule
\caption{Reactivity combinator.}
\label{figure:switch_combinator}
\end{figure}

There are other formulations of a reactive combinator found in FRP literature
and libraries, which may be implemented using the one supplied. Some of these
are shown in Figure~\ref{figure:alternate_switching_combinators} and may be
provided in a future version of the TimeFlies library.

\begin{figure}
\begin{code}
-- Alternate version of switch,
-- implemented in terms of supplied version
switch_gen ::    (svIn :~> (svOut :^: SVEvent a))
              -> (a -> svIn :~> svOut)
              -> svIn :~> svOut
switch_gen sf f =
  switch (sf >>> second (pureEventTransformer f))

-- Repeated switch, which takes replacement signal functions
-- externally.
rswitch ::    (svIn :~> svOut)
           -> (svIn :^: SVEvent (svIn :~> svOut)) :~> svOut
rswitch sf =
  switch (first sf >>> second (pureEventTransformer rswitch))
\end{code}
\hrule
\caption{Alternate reactivity combinators.}
\label{figure:alternate_switching_combinators}
\end{figure}

\subsection{Feedback}
\label{subsection:System_Design_and_Interface-Combinators-Feedback}

It is often useful for a signal function to receive a portion of its
own output as input. This is especially useful when we have two
signal functions which we would like to mutually interact. We cannot
just serially compose them, we must also bring the output of the second
back around to the first. Many signal-processing algorithms also depend
on feedback. The combinator which provides this ability is shown
in Figure~\ref{figure:feedback_combinator}. 

\begin{figure}
\begin{code}
loop ::    ((svIn :^: svLoop) :~> (svOut :^: svLoop))
        -> svIn :~> svOut
\end{code}
\hrule
\caption{Feedback combinator.}
\label{figure:feedback_combinator}
\end{figure}

This combinator provides decoupling for signals
(the input signal is the output signal at the previous time-step)
but not events (event occurrences are supplied to the combinator immediately).
This means that the programmer has the responsibility to ensure that feedback
does not generate an infinite sequence of events in a single time-step.

\subsection{Event-Specific Combinators}
\label{subsection:System_Design_and_Interface-Combinators-Event_specific_combinators}

Several combinators are provided for manipulating, suppressing, and generating events.
Each of the combinators has an option variant and a list variant. The option variant
produces an output event occurrence whenever the application of the supplied function
to the input event produces a value. The list version produces an event occurrence for
each of the elements of the output list. The combinators are shown in Figure~\ref{figure:event_specific_combinators}.

\begin{figure}
\begin{code}
-- Apply the function to each input occurrence,
-- and produce an occurrence for each Just.
filter :: (a -> Maybe b) -> SVEvent a :~> SVEvent b

-- Apply the function to each input occurrence,
-- and produce an occurrence for each list element
filterList :: (a -> [b]) -> SVEvent a :~> SVEvent b

-- Apply the function to the stored accumulator
-- and the event occurrence, replacing the accumulator
-- and possibly outputting an occurrence
accumulate ::    (a -> b -> (Maybe c, a))
              -> a
              -> SVEvent b :~> SVEvent c

-- Apply the function to the stored accumulator
-- and the event occurrence, replacing the
-- accumulator and outputting an event occurrence
-- for each element of the list
accumulateList ::    (a -> b -> ([c], a))
                  -> a
                  -> SVEvent b :~> SVEvent c
\end{code}
\hrule
\caption{Event-specific combinators.}
\label{figure:event_specific_combinators}
\end{figure}

The filter combinators are stateless, and thus apply the function to only the new
input value. They are useful for suppressing events, as well as for extracting one
of multiple cases of a datatype. For instance, a splitter for events carrying
{\tt Either}-valued occurrences could be written as:

\begin{code}
getLeft :: Either a b -> Maybe a
getLeft (Left x) = Just x
getLeft _ = Nothing

getRight :: Either a b -> Maybe b
getRight (Right x) = Just x
getRight _ = Nothing

split :: SVEvent (Either a b) :~> (SVEvent a :^: SVEvent b)
split = copy >>> first (filter getLeft) >>> second (filter getRight)
\end{code}

The accumulate combinators are stateful, applying the supplied function
to both the input value and an accumulator. This function has two results:
the option or list of output event occurrence values, and the new value
for the accumulator.

The accumulator is useful when responses to multiple event occurrences
(from one or more sources) must be coordinated. For instance, in the
benchmark application (see Chapter~\ref{chapter:Evaluation_and_Comparisons}),
a table is maintained that allows knowledge from previous event occurrences
(packets from a network switch) to be used in deciding where the present
packet ought to go.

\subsection{Joining}
\label{subsection:System_Design_and_Interface-Combinators-Joining}

The joining combinators provide the ability to combine two event
streams, two signals, or a signal and an event stream. These
combinators are shown in Figure~\ref{figure:joining_combinators}

The {\tt union} combinator is a non-deterministic merge of event
streams. Any event which occurs on either input will occur
on the output. For simultaneous event occurrences, the order of occurrence
is not guaranteed, but the occurrence itself is. This construct
is also guaranteed to respect the relationship of event occurrences to sampling
intervals.

The {\tt combineSignals} combinator applies a binary function pointwise to two
signals, and produces the result of this application as a third signal. The
combining function is necessary because we will have two input samples at each
time step, and must produce exactly one output sample.

The {\tt capture} combinator adds the last-sampled value of a signal at the time
of an event occurrence to that event occurrence.

These three combinators together provide the ability to combine elements of
signal vectors. By combining these combinators, along with the {\tt cancelLeft}
and {\tt cancelRight} routing combinators, arbitrary signal vectors can be
reduced.

\begin{figure}
\begin{code}
union          :: (SVEvent a :^: SVEvent a) :~> SVEvent a
combineSignals :: (a -> b -> c) -> (SVSignal a :^: SVSignal b) :~> SVSignal c
capture        :: (SVSignal a :^: SVEvent b) :~> SVEvent (b, a)
\end{code}
\hrule
\caption{Joining combinators.}
\label{figure:joining_combinators}
\end{figure}

\subsection{Time Dependence}
\label{subsection:System_Design_and_Interface-Signal_Functions-Time_Dependence}

A set of combinators are provided for making use of time-dependence in a signal
function. These combinators allow the modification of signals and events with
respect to time, and the observation of the current time value. The combinators
are shown in Figure~\ref{figure:time_combinators}

\begin{figure}
\begin{code}
time      :: SVEmpty :~> SVSignal Double
delay     :: Double -> (SVEvent a :^: SVEvent Double) :~> SVEvent a
integrate :: TimeIntegrate i => SVSignal i :~> SVSignal i
\end{code}
\hrule
\caption{Time-dependent combinators.}
\label{figure:time_combinators}
\end{figure}

The simplest time-dependent combinator is {\tt time}, which simply outputs
the time since it began to be evaluated. This does not necessarily correspond to
the time since the global signal function began to be evaluated, since the
signal function in which the {\tt time} combinator is used may have been
introduced through {\tt switch}.

The {\tt delay} signal function allows events to be delayed. An initial delay
time is given, and event occurrences on the right input can carry a new delay
time. Event occurrences on the left input are stored and output when their delay
time has passed. Changing the delay time does not affect occurrences already
waiting.

The {\tt integrate} combinator outputs the rectangle-rule approximation of the
integral of its input signal with respect to time.

\section{Evaluation}
\label{section:System_Design_and_Interface-Evaluator}

The goal of the evaluation interface is to provide a means to evaluate
declaratively specified signal functions in the context of a monadic
input/output system, such as Haskell's IO monad. The means providing a simple
monadic interface for taking actions on a signal function in response to
inputs, and for actuating the outputs of a signal function.

The evaluation interface provides a modified state monad which holds a signal
function, together with some additional information, as its state (shown in
Figure~\ref{figure:evaluation_state}).
Rather than monadic instructions to put and get the state, the monad provides instructions
to trigger an input event, update an input signal, and trigger sampling of
signals in the signal function. Additional state includes the current set of
modifications to the input signals (since the last sample) and a set of
handlers which actuate effects based on output events or changes to the output
signal.

\begin{figure}
\begin{code}
-- A vector of handlers for outputs
data SVHandler out sv

-- A dummy handler for an empty output
emptyHandler    :: SVHandler out SVEmpty

-- A handler for an updated signal sample
signalHandler   :: (a -> out) -> SVHandler out (SVSignal a)

-- A handler for an event occurrence
eventHandler    :: (a -> out) -> SVHandler out (SVEvent a)

-- Combine handlers for a vector
combineHandlers ::    SVHandler out svLeft
                   -> SVHandler out svRight
                   -> SVHandler out (svLeft :^: svRight)

-- The state maintained when evaluating a signal function
data SFEvalState m svIn svOut

-- Create the initial state for evaluating a signal function
initSFEval ::    SVHandler (m ()) svOut   -- Output handlers

              -> SVSample svIn            -- Initial input samples
              
              -> Double                   -- Initial external time,
                                          -- corresponding to time 0 for
                                          -- the signal function
                                          
              -> (svIn :~> svOut)         -- Signal function to evaluate
              -> SFEvalState m svIn svOut
\end{code}
\hrule
\caption{State maintained when evaluating a signal function.}
\label{figure:evaluation_state}
\end{figure}

Sets which correspond to signal vectors are built with ``smart'' constructors.
For instance, to construct a set of handlers, individual handling functions are
lifted to handler sets with the {\tt signalHandler} and {\tt eventHandler}
functions, and then combined with each other and {\tt emptyHandler} leaves
using the {\tt combineHandlers} function.

Building the initial input sample is similar, but {\tt sampleEvt} leaves do
not carry a value.

In order to initialize the state, the user must supply a set of handlers, the
signal function to evaluate, and initial values for all of the signal inputs
(Figure~\ref{figure:initial_input}).

\begin{figure}
\begin{code}
-- A sample for all leaves of a signal vector
data SVSample sv

-- Create a sample for a signal leaf
sample          :: a -> SVSample (SVSignal a)

-- A dummy sample for an event leaf
sampleEvt       :: SVSample (SVEvent a)

-- A dummy sample for an empty leaf
sampleNothing   :: SVSample SVEmpty

-- Combine two samples
combineSamples  ::    SVSample svLeft
                   -> SVSample svRight
                   -> SVSample (svLeft :^: svRight)
\end{code}
\hrule
\caption{Data type for initial input}
\label{figure:initial_input}
\end{figure}

This state can then be passed to a monadic action which will supply input to
the signal function. Inputs are constructed using a simple interface with
functions to construct sample updates and event occurrences, and to specify
their place in the vector (Figure~\ref{figure:ongoing_input}).

\begin{figure}
\begin{code}
-- Class to overload left and right functions
class SVRoutable r where
  svLeft          :: r svLeft -> r (svLeft :^: svRight)
  svRight         :: r svRight -> r (svLeft :^: svRight)

-- An input event occurrence
data SVEventInput sv
instance SVRoutable SVEventInput sv

-- An updated sample for a signal
data SVSignalUpdate sv
instance SVRoutable SVSignalUpdate sv

-- Create an event occurrence
svOcc           :: a -> SVEventInput (SVEvent a)

-- Create an updated sample
svSig           :: a -> SVSignalUpdate (SVSignal a)
\end{code}
\hrule
\caption{Data types for ongoing input.}
\label{figure:ongoing_input}
\end{figure}

The {\tt SFEvalT} monad is actually a monad transformer, that is, it is
parameterized over an underlying monad whose actions may be lifted to
{\tt SFEvalT}. In the usual case, this will be the {\tt IO} monad.

{\tt SFEvalT} actions are constructed using combinators to push events,
update inputs, and step time, as well as actions lifted from the underlying
monad (used to obtain these inputs). An action in the underlying monad
which produces the result and a new state is obtained with the {\tt runSFEvalT}
function. These combinators are shown in Figure~\ref{figure:evaluation_combinators}.

\begin{figure}
\begin{code}
-- The evaluation monad
data SFEvalT svIn svOut m a
instance MonadTrans (SFEvalT svIn svOut)
instance (Monad m) => Monad (SFEvalT svIn svOut m)
instance (Functor m) => Functor (SFEvalT svIn svOut m)
instance (Monad m, Functor m) => Applicative (SFEvalT svIn svOut m)
instance (MonadIO m) => MonadIO (SVEvalT svIn svOut m)

-- Obtain an action in the underlying monad
-- from an SFEvalT and a new state.
runSFEvalT ::    SFEvalT svIn svOut m a
              -> SFEvalState m svIn svOut
              -> m (a, SFEvalState m svIn svOut)

-- Push an event occurrence.
push :: (Monad m) => SVEventInput svIn -> SFEvalT svIn svOut m ()

-- Update the value of an input signal sample
-- (not immediately observed)
update :: (Monad m) => SVEventInput svIn -> SFEvalT svIn svOut m ()

-- Step forward in time, observing the updated signal values
step :: (Monad m) => Double -> SFEvalT svIn svOut m ()
\end{code}
\hrule
\caption{Evaluation combinators}
\label{figure:evaluation_combinators}
\end{figure}
