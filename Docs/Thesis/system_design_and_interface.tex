\chapter{System Design and Interface}
\label{chapter:System_Design_and_Interface}

\section{Goals}
\label{section:System_Design_and_Interface-Goals}

The (not yet fully realized) goal of FRP is to provide an efficient, declarative
abstraction for creating reactive programs. Towards this overall goal, there are
three goals which this system is intended to meet.

\subsection{Efficient Evaluation}
\label{subsection:System_Design_and_Interface-Goals-Efficient_and_Push_based_Evaluation}

Efficient evaluation is the motivation for push-based evaluation of events.
Since FRP programs are expected to  interact with an outside world in real time,
efficiency cannot be measured by the runtime of a program. Thus, when speaking of efficiency,
we are expressing a desire that the system utilize as few system resources as possible
for the task at hand, while responding as quickly as possible to external inputs and
producing output at a consistently high sample rate.

\subsection{Composability}
\label{subsection:System_Design_and_Interface-Goals-Composability}

A composable abstraction is one in which values in that abstraction may be
combined in such a way that reasoning about their actions together involves
little more than reasoning about their actions separately. In a signal function
system, the only interaction between composed signal functions ought to be that
the output of one is the input of another. Composability permits a particularly
attractive form of software engineering in which successively larger systems are
created from by combining smaller systems, without having to reason about the 
components of the systems being combined.

\subsection{Ease of Integration}
\label{subsection:System_Design_and_Interface-Goals-Ease_of_Integration}

It is fine for a system to be composable with regards to itself, but an FRP
system must interact with the outside world. Since we cannot anticipate every
possible form of input and output that the system will be asked to interact
with, we must interface with Haskell's IO system. In particular, most libraries
for user interaction (e.g. GUI and graphics libraries such as GTK+ and GLUT) and
most libraries for time-dependent IO (e.g. audio and video systems) make use of
the event loop abstraction. In this abstraction, event handlers are registered
with the system, and then a command is issued to run a loop which detects events
and runs the handlers, and uses the results of the handlers to render the
appropriate output. 

We would like for the FRP system to be easy to integrate with such IO systems,
while being flexible enough to enable its use with other forms of IO systems,
such as simple imperative systems, threaded systems, or network servers.

\section{Semantics}
\label{section:System_Design_and_Interface-Semantics}

A rigorous and formal elucidation of the semantics of signal-function FRP remains
unattempted, but there is a sufficient change to the practical semantics of
signal-function FRP between previous signal-function systems and TimeFlies to warrant
some description.

In previous systems such as Yampa, events were understood (and typed) as option-valued
signals. This approach is undesirable for several reasons. The most pressing reason is
that it prohibits push-based evaluation of events, because events are embedded in the
samples of a signal and must be searched for.

Another concern is that this approach limits the event rate to the sampling rate.
The rate of sampling should, at some level, not matter to the FRP system. Events which
occur between sampling intervals are never observed by the system.

This concern drives the next concern. Events are not instantaneous in this formulation.
If a signal is option valued, the sampling instant must fall within the interval where
there is an event occurrence present for that event to be observed. If events are
instantaneous, the probability of observing an event occurrence is zero.

Therefore, TimeFlies employs the N-Ary FRP type formulation to represent signals and
events as distinct entities in the inputs and outputs of signal functions. This means
we are now free to choose our representation of events, and to separate it from the
representation and evaluation of signals.

This freedom yields the additional ability to make events independent of the sampling
interval altogether. The semantics of event handling in TimeFlies is that an event occurrence
is responded to immediately, and does not wait for the next sampling instant. This allows events
to be instantaneous, and further, allows multiple events to occur within a single sampling interval.

There are two tradeoffs intrinsic to this approach. The first is that events are only partially ordered
temporally. There is no way to guarantee the order of observation of event occurrences occurring in the
same sampling interval. Further, the precise time of an event occurrence cannot be observed, only the 
time of the last sample prior to the occurrence.

In return for giving up total ordering and precise observation of the timing of events, we obtain the
ability to employ push-based evaluation for event occurrences, and the ability to non-deterministically
merge event occurrences. When events being input to a non-deterministic merge have simultaneous occurrences,
we simply select one arbitrarly to occur first. This does not violate any guarantee about time values, since
they will both have the same time value in either case, and does not violate any guarantee about ordering,
since no guarantee of their order is given.

A formal semantic description of signal function FRP would clarify the consequences of this decision somewhat,
but is outside the scope of this thesis.

\section{Types}
\label{section:System_Design_and_Interface-Types}

In a strongly and statically typed functional language, types are a key part of
an interface. Types provide a mechanism for describing and ensuring properties
of the interface's components and about the systems created with these
components. 

\subsection{Signal Vectors}
\label{subsection:System_Design_and_Interface-Types-Signal_Vectors}

In order to type signal functions, we must be able to describe their input and
output. In most signal function systems, a signal function takes exactly one
input and produces exactly one output. Multiple inputs or outputs are handled
by making the output a tuple, and combinators which combine or split the inputs
or outputs of a signal assume this. Events are represented at the type level
as a particular type of signal, and at the value level as an option, either an
event occurrence or not.

This method of typing excludes push-based evaluation at the outset.
It is not possible to construct a "partial tuple" nor in general is it possible
to construct only part of any type of value. Push-based evaluation depends on
evaluating only that part of the system which is updated, which means evaluating
only that part of the input which is updated.

In order to permit the construction of partial inputs and outputs, we make use
of signal vectors. Signal vectors are uninhabited types which describe the input
and output of a signal function. Singleton vectors are parameterized over the
type carried by the signal or by event occurrences. The definition of the signal
vector type is shown in Figure~\ref{figure:signal_vector_types}. 

Having an uninhabited signal vector type allows us to construct representations
of inputs and outputs which are hidden from the user of the system, and are
designed for partial representations.

\begin{figure}
\begin{code}
data SVEmpty    -- An empty signal vector component,
                -- neither event nor signal
data SVSignal a -- A signal, carrying values of type a
data SVEvent a  -- An event, whose occurrences carry values of type a
data SVAppend svLeft svRight -- The combination of the signal vectors
                             -- svLeft and svRight
\end{code}
\hrule
\caption{Signal vector types}
\label{figure:signal_vector_types}
\end{figure}

\subsection{Signal Functions}
\label{subsection:System_Design_and_Interface-Types-Signal_Functions}

The type constructor for signal functions is shown in
Figure~\ref{figure:signal_function_types}. For the {\tt init} parameter, only
one possible instantiation is shown. The usefulness of this type parameter,
along with another instantation which is hidden from users of the library,
is discussed in the section on implementation of signal functions
(Section~\ref{section:Implementation-Signal_Functions}).

Signal functions with signal vectors as input and output types form a
Haskell {\tt GArrow}~\cite{Megacz2011}. Specifically, the signal function
type constructor (with the initialization parameter fixed) forms the arrow
type, the {\tt SVAppend} type constructor forms the product type, and the
{\tt SVEmpty} type constructor forms the unit type.

The representation of signal functions is discussed in
Section~\ref{section:Implementation-Signal_Functions}. Here it suffices to say
that the use of GADTs permits the construction of values which make use of
uninhabited types as instantiations of type parameters.

The type synonyms {\tt :\textasciitilde>} and {\tt :\textasciicircum:} are included for readability and are
not crucial to the FRP system.

\begin{figure}
\begin{code}
-- Signal functions
-- init: The initialization type for 
-- the signal function, always NonInitialized
-- for exported signal functions
-- svIn: The input signal vector
-- svOut: The output signal vector
data SF init svIn svOut

data NonInitialized

type svIn :~> svOut = SF NonInitialized svIn svOut
type svLeft :^: svRight = SVAppend svLeft svRight
\end{code}
\hrule
\caption{Signal function types.}
\label{figure:signal_function_types}
\end{figure}

\subsection{Evaluation Monad}
\label{section:System_Design_and_Interface-Types-Evaluation_Monad}

A {\em monad} is a standard, composable abstraction for writing functions with
a context, used in Haskell for IO~\cite{PeytonJones1993,PeytonJones2001} among
other tasks. A monad is simply a 1-arity type constructor together with two
functions. The first function, {\tt return}, takes a value of type {\tt a} and
produces a value of type {\tt m a}, where {\tt m} is the type constructor. The
second, called {\tt bind} and stylized in the Haskell standard library as the
infix operator {\tt (>>=)}, takes a value of type {\tt m a} and a function
from {\tt a} to {\tt m b} and produces a value of type {\tt m b}. This allows
a value to be operated on out of the context and a new context to be assigned.

A monad can have other primitives which manipulate the context in some way. For 
instance, the primtives in Haskell's {\tt IO} monad produce actions which, when
interpreted as part of the {\tt main} action, produce some side-effect. The
{\tt State} monad provides {\tt get} and {\tt put} operations to work with a 
state value stored in the context.

Monad transformers~\cite{Jones1995} provide a means to combine the functionality
of multiple monads. A monad transformer is a monad with an extra type parameter.
This type parameter is instantiated with the type constructor of the underlying
monad, and an extra operation {\tt lift} is provided which converts values in
the underlying monad to values in the monad transformer.

The evaluation monad is a monad transformer. This permits it to be used in
conjunction with the {\tt IO} monad (or any other monad) to describe how input
is to be obtained for the signal function being evaluated, and how outputs are
to be handled.

The evaluation monad, in addition to the standard monad operators, provides a
means of {\em initializing} a signal function, and a means of translating the
monadic value describing evaluation to a value in the underlying monad. This
means, for instance, that we can obtain an action in the {\tt IO} monad to
evaluate a signal function.

The type of the evaluation monad must track the input type of the signal
function. The monad's context stores a mapping from outputs to handling actions.
An existential type can thus be used to ``hide'' the output type of the signal
function. However, inputs must come from external values, so the input type
cannot be hidden. There are thus three type parameters to the monad's type
constructor: the input signal vector, the type of the underlying monad, and the
monadic type parameter. The type is shown in Figure~\ref{figure:evaluation_monad_types}.

\begin{figure}
\begin{code}
-- A signal function's evaluations state
data SFEvalState svIn m
-- The evaluation monad
data SFEvalT svIn m a
\end{code}
\hrule
\caption{Evaluation monad types.}
\label{figure:evaluation_monad_types}
\end{figure}

\section{Combinators}
\label{section:System_Design_and_Interface-Combinators}

Signal functions are constructed from combinators, which are primitive signal
functions and operations to combine these primitives. These combinators are
grouped as basic signal functions, lifting operations for pure functions,
routing, reactivity, feedback, and time dependence.

\subsection{Basic Signal Functions}
\label{subsection:System_Design_and_Interface-Combinators-Basic_Signal_Functions}

The basic signal functions (Figure~\ref{figure:basic_signal_functions})
provide very simple operations. The {\tt identity} signal function, as expected,
simply copies its input to its output. The {\tt constant} signal function
produces the provided value as a signal at all times. The {\tt never} signal
function has an event output which never produces occurrences. The {\tt asap}
function produces an event occurrence with the given value at the first time
step after it is switched into the network. The {\tt after} function waits for
the specified amount of time before producing the event occurrence.

With the exception of {\tt identity}, all of the basic signal functions have
empty inputs. This allows these signal functions to be used to insert values
into the network which are known when the signal function is created, without
having to route those values from an input.

\begin{figure}
\begin{code}
-- Pass the input unmodified to the output
identity :: sv :~> sv

-- Produce a signal which is at all times the supplied value
constant :: a -> SVEmpty :~> SVSignal a

-- An event with no occurrences
never    :: SVEmpty :~> SVEvent a

-- An event with one occurrence, as soon as possible after
-- the signal function is initialized
asap     :: a -> SVEmpty :~> SVEvent a

-- An event after the specified amount of time has elapsed.
after    :: Double -> a -> SVEmpty :~> SVEvent a
\end{code}
\hrule
\caption{Basic signal functions.}
\label{figure:basic_signal_functions}
\end{figure}

\subsection{Lifting Pure Functions}
\label{subsection:System_Design_and_Interface-Combinators-Lifting_Pure_Functions}

Two combinators are provided to lift pure functions to signal functions (Figure~\ref{figure:lifting_pure_functions}).
The {\tt pureSignal} combinator applies the pure function to a signal at every
sample point. The {\tt pureEvent} combinator applies the function to each
occurrence of an input event.

\begin{figure}
\begin{code}
-- Apply the given function to a signal at all points in time
pureSignalTransformer :: (a -> b) -> SVSignal a :~> SVSignal b

-- Apply the given function to each event occurrence
pureEventTransformer  :: (a -> b) -> SVEvent a :~> SVEvent b
\end{code}
\hrule
\caption{Lifting pure functions.}
\label{figure:lifting_pure_functions}
\end{figure}

\subsection{Routing}
\label{subsection:System_Design_and_Interface-Combinators-Routing}

The routing combinators are used to combine signal functions, and
to re-arrange signal vectors in order to connect signal functions.
The routing combinators are shown in Figure~\ref{figure:routing_combinators}.

Only those combinators which modify or combine signal functions
({\tt (>>>)}, {\tt first}, {\tt second}) are reactive, and then
only if they inherit their reactivity from the signal function(s)
they modify. The rest do not react to or modify the input in any
way, except to re-arrange it, copy it, or discard it altogether.

\begin{figure}
\begin{code}
-- Use the output of one signal function as the input for another
(>>>) :: (svIn :~> svBetween) -> (svBetween :~> svOut) -> svIn :~> svOut

-- Pass through the right side of the input unchanged
first :: (svIn :~> svOut) -> (svIn :^: sv) :~> (svOut :^: sv)

-- Pass through the left side of the input unchanged
second :: (svIn :~> svOut) -> (sv :^: svIn) :~> (sv :^: svOut)

-- Swap the left and right sides
swap :: (svLeft :^: svRight) :~> (svRight :^: svLeft)

-- Duplicate the input
copy :: sv :~> (sv :^: sv)

-- Ignore the input
ignore :: sv :~> svEmpty

-- Remove an empty vector on the left
cancelLeft :: (SVEmpty :^: sv) :~> sv

-- Add an empty vector on the left
uncancelLeft :: sv :~> (SVEmpty :^: sv)

-- Remove an empty vector on the right
cancelRight :: (sv :^: SVEmpty) :~> sv

-- Add an empty vector on the right
uncancelRight :: sv :~> (sv :^: SVEmpty)

-- Make right-associative
associate :: ((sv1 :^: sv2) :^: sv3) :~> (sv1 :^: (sv2 :^: sv3))

-- Make left-associative
unassociate :: (sv1 :^: (sv2 :^: sv3)) :~> ((sv1 :^: sv2) :^: sv3)
\end{code}
\hrule
\caption{Routing combinators.}
\label{figure:routing_combinators}
\end{figure}

\subsection{Reactivity}
\label{subsection:System_Design_and_Interface-Combinators-Reactivity}

Reactivity is introduced by means of the {\tt switch} combinator
(Figure~\ref{figure:switch_combinator}). The design of this combinator
allows modular packaging of reactivity. A signal function can determine
autonomously when to replace itself, based only on its input and state,
by emitting an event occurrence carrying its replacement. The combinator
consumes and hides the event carrying the replacement signal function, 
so the reactivity is not exposed by the resulting reactive signal function.

\begin{figure}
\begin{code}
switch ::    (svIn :~> (svOut :^: SVEvent (svIn :~> svOut)))
          -> svIn :~> svOut
\end{code}
\hrule
\caption{Combinator for reactivity.}
\label{figure:switch_combinator}
\end{figure}

There are other formulations of a reactive combinator which may be implemented
using the one supplied. These are shown in Figure~\ref{figure:alternate_switching_combinators}
and may be provided in a future version of the TimeFlies library.

\begin{figure}
\begin{code}
-- Alternate version of switch,
-- implemented in terms of supplied version
switch_gen ::    (svIn :~> (svOut :^: SVEvent a))
              -> (a -> svIn :~> svOut)
              -> svIn :~> svOut
switch_gen sf f =
  switch (sf >>> second (pureEventTransformer f))

-- Supplied version in terms of alternate version
switch ::    (svIn :~> (svOut :^: SVEvent (svIn :~> svOut)))
          -> svIn :~> svOut
switch sf = switch_gen sf id

-- Repeated switch, which takes replacement signal functions
-- externally.
rswitch ::    (svIn :~> svOut)
           -> (svIn :^: SVEvent (svIn :~> svOut)) :~> svOut
rswitch sf =
  switch (first sf >>> second (pureEventTransformer rswitch))
\end{code}
\hrule
\caption{Alternate combinators for reactivity.}
\label{figure:alternate_switching_combinators}
\end{figure}

\section{Evaluation}
\label{section:System_Design_and_Interface-Evaluator}

The evaluation interface provides a modified state monad which holds a signal
function, together with some additional information, as its state (shown in Figure~\ref{figure:evaluation_state}.
Rather than monadic instructions to put and get the state, the monad provides instructions
to trigger an input event, update an input signal, and trigger sampling of
signals in the signal function. Additional state includes the current set of
modifications to the input signals (since the last sample) and a set of
handlers which actuate effects based on output events or changes to the output
signal.

\begin{figure}
\begin{code}
-- A vector of handlers for outputs
data SVHandler out sv

-- A dummy handler for an empty output
emptyHandler    :: SVHandler out SVEmpty

-- A handler for an updated signal sample
signalHandler   :: (a -> out) -> SVHandler out (SVSignal a)

-- A handler for an event occurrence
eventHandler    :: (a -> out) -> SVHandler out (SVEvent a)

-- Combine handlers for a vector
combineHandlers ::    SVHandler out svLeft
                   -> SVHandler out svRight
                   -> SVHandler out (svLeft :^: svRight)

-- The state maintained when evaluating a signal function
data SFEvalState m svIn svOut

-- Create the initial state for evaluating a signal function
initSFEval ::    SVHandler (m ()) svOut
              -> SVSample svIn
              -> Double
              -> (svIn :~> svOut)
              -> SFEvalState m svIn svOut
\end{code}
\hrule
\caption{State maintained when evaluating a signal function}
\label{figure:evaluation_state}
\end{figure}

In order to initialize the state, the user must supply a set of handlers, the
signal function to evaluate, and initial values for all of the signal inputs
(Figure~\ref{figure:initial_input}).

\begin{figure}
\begin{code}
-- A sample for all leaves of a signal vector
data SVSample sv

-- Create a sample for a signal leaf
sample          :: a -> SVSample (SVSignal a)

-- A dummy sample for an event leaf
sampleEvt       :: SVSample (SVEvent a)

-- A dummy sample for an empty leaf
sampleNothing   :: SVSample SVEmpty

-- Combine two samples
combineSamples  ::    SVSample svLeft
                   -> SVSample svRight
                   -> SVSample (svLeft :^: svRight)
\end{code}
\hrule
\caption{Data type for initial input}
\label{figure:initial_input}
\end{figure}

This state can then be passed to a monadic action which will supply input to
the signal function. Inputs are constructed using a simple interface with
functions to construct sample updates and event occurrences, and to specify
their place in the vector (Figure~\ref{figure:ongoing_input}).

\begin{figure}
\begin{code}
-- Class to overload left and right functions
class SVRoutable r where
  svLeft          :: r svLeft -> r (svLeft :^: svRight)
  svRight         :: r svRight -> r (svLeft :^: svRight)

-- An input event occurrence
data SVEventInput sv
instance SVRoutable SVEventInput sv

-- An updated sample for a signal
data SVSignalUpdate sv
instance SVRoutable SVSignalUpdate sv

-- Create an event occurrence
svOcc           :: a -> SVEventInput (SVEvent a)

-- Create an updated sample
svSig           :: a -> SVSignalUpdate (SVSignal a)
\end{code}
\hrule
\caption{Data types for ongoing input.}
\label{figure:ongoing_input}
\end{figure}

The {\tt SFEvalT} monad is actually a monad transformer, that is, it is
parameterized over an underlying monad whose actions may be lifted to
{\tt SFEvalT}. In the usual case, this will be the {\tt IO} monad.

{\tt SFEvalT} actions are constructed using combinators to push events,
update inputs, and step time, as well as actions lifted from the underlying
monad (used to obtain these inputs). An action in the underlying monad
which produces the result and a new state is obtained with the {\tt runSFEvalT}
function. These combinators are shown in Figure~\ref{figure:evaluation_combinators}.

\begin{figure}
\begin{code}
-- The evaluation monad
data SFEvalT svIn svOut m a
instance MonadTrans (SFEvalT svIn svOut)
instance (Monad m) => Monad (SFEvalT svIn svOut m)
instance (Functor m) => Functor (SFEvalT svIn svOut m)
instance (Monad m, Functor m) => Applicative (SFEvalT svIn svOut m)
instance (MonadIO m) => MonadIO (SVEvalT svIn svOut m)

-- Obtain an action in the underlying monad
-- from an SFEvalT and a new state.
runSFEvalT ::    SFEvalT svIn svOut m a
              -> SFEvalState m svIn svOut
              -> m (a, SFEvalState m svIn svOut)

-- Push an event occurrence.
push :: (Monad m) => SVEventInput svIn -> SFEvalT svIn svOut m ()

-- Update the value of an input signal sample
-- (not immediately observed)
update :: (Monad m) => SVEventInput svIn -> SFEvalT svIn svOut m ()

-- Step forward in time, observing the updated signal values
step :: (Monad m) => Double -> SFEvalT svIn svOut m ()
\end{code}
\hrule
\caption{Evaluation combinators}
\label{figure:evaluation_combinators}
\end{figure}

