\chapter{System Design and Interface}
\label{chapter:System_Design_and_Interface}

\section{Goals}
\label{section:System_Design_and_Interface-Goals}

The (not yet fully realized) goal of FRP is to provide an efficient, declarative
abstraction for creating reactive programs. Towards this overall goal, there are
three goals which this system is intended to meet.

\subsection{Efficient and Push-Based Evaluation}
\label{subsection:System_Design_and_Interface-Goals-Efficient_and_Push_based_Evaluation}

Efficient evaluation is the for push-based evaluation. Since FRP programs are
expected to  interact with an outside world in real time, efficiency cannot be
measured by runtime. Thus, when speaking of efficiency, we are expressing a
desire that the system utilize as few system resources as possible for the task
at hand, while responding as quickly as possible to external inputs and
producing output at a consistently high sample rate.

\subsection{Composability}
\label{subsection:System_Design_and_Interface-Goals-Composability}

A composable abstraction is one in which values in that abstraction may be
combined in such a way that reasoning about their actions together involves
little more than reasoning about their actions separately. In a signal function
system, the only interaction between composed signal functions ought to be that
the output of one is the input of another. Composability permits a particularly
attractive form of software engineering in which successively larger systems are
created from by combining smaller systems, without having to reason about the 
components of the systems being combined.

\subsection{Ease of Integration}
\label{subsection:System_Design_and_Interface-Goals-Ease_of_Integration}

It is fine for a system to be composable with regards to itself, but an FRP
system must interact with the outside world. Since we cannot anticipate every
possible form of input and output that the system will be asked to interact
with, we must interface with Haskell's IO system. In particular, most libraries
for user interaction (e.g. GUI and graphics libraries such as GTK+ and GLUT) and
most libraries for time-dependent IO (e.g. audio and video systems) make use of
the event loop abstraction. In this abstraction, event handlers are registered
with the system, and then a command is issued to run a loop which detects events
and runs the handlers, and uses the results of the handlers to render the
appropriate output. 

We would like for the FRP system to be easily to integrate with such IO systems,
while being flexible enough to enable its use with other forms of IO systems,
such as simple imperative systems, threaded systems, or network servers.

\section{Types}
\label{section:System_Design_and_Interface-Types}

In a strongly and statically typed functional language, types are a key part of
an interface. Types provide a mechanism for describing and ensuring properties
of the interface's components and about the systems created with these
components. 

\subsection{Signal Vectors}
\label{subsection:System_Design_and_Interface-Types-Signal_Vectors}

In order to type signal functions, we must be able to describe their input and
output. In most signal function systems, a signal function takes exactly one
input and produces exactly one output. Multiple inputs or outputs are handled
by making the output a tuple, and combinators which combine or split the inputs
or outputs of a signal assume this. Events are represented at the type level
as a particular type of signal, and at the value level as an option, either an
event occurrence or not.

This method of typing excludes push-based evaluation at the outset.
It is not possible to construct a "partial tuple" nor in general is it possible
to construct only part of any type of value. Push-based evaluation depends on
evaluating only that part of the system which is updated, which means evaluating
only that part of the input which is updated.

In order to permit the construction of partial inputs and outputs, we make use
of signal vectors. Signal vectors are uninhabited types which describe the input
and output of a signal function. Singleton vectors are parameterized over the
type carried by the signal or by event occurrences. The definition of the signal
vector type is shown in Figure~\ref{figure:signal_vector_types}. 

Having an uninhabited signal vector type allows us to construct representations
of inputs and outputs which are hidden from the user of the system, and are
designed for partial representations.

\begin{figure}
\begin{code}
data SVEmpty    -- An empty signal vector component,
                -- neither event nor signal
data SVSignal a -- A signal, carrying values of type a
data SVEvent a  -- An event, whose occurrences carry values of type a
data SVAppend svLeft svRight -- The combination of the signal vectors
                             -- svLeft and svRight
\end{code}
\hrule
\caption{Signal vector types}
\label{figure:signal_vector_types}
\end{figure}

\subsection{Signal Functions}
\label{subsection:System_Design_and_Interface-Types-Signal_Functions}

The type constructor for signal functions is shown in
Figure~\ref{figure:signal_function_types}. For the {\tt init} parameter, only
one possible instantiation is shown. The usefulness of this type parameter,
along with another instantation which is hidden from users of the library,
is discussed in the section on implementation of signal functions
(Section~\ref{section:Implementation-Signal_Functions}).

The representation of signal functions is also discussed in
Section~\ref{section:Implementation-Signal_Functions}. Here it suffices to say
that the use of GADTs permits the construction of values which make use of
uninhabited types as instantiations of type parameters.

The type synonyms {\tt :~>} and {\tt :\textasciicircum:} are included for readability and are
not crucial to the FRP system.

\begin{figure}
\begin{code}
-- Signal functions
-- init: The initialization type for 
-- the signal function, always NonInitialized
-- for exported signal functions
-- svIn: The input signal vector
-- svOut: The output signal vector
data SF init svIn svOut

data NonInitialized

type svIn :~> svOut = SF NonInitialized svIn svOut
type svLeft :^: svRight = SVAppend svLeft svRight
\end{code}
\hrule
\caption{Signal function types.}
\label{figure:signal_function_types}
\end{figure}

\subsection{Evaluation Monad}
\label{section:System_Design_and_Interface-Types-Evaluation_Monad}

A {\em monad} is a standard, composable abstraction for writing functions with
a context, used in Haskell for IO~\cite{PeytonJones1993,PeytonJones2001} among
other tasks. A monad is simply a 1-arity type constructor together with two
functions. The first function, {\tt return}, takes a value of type {\tt a} and
produces a value of type {\tt m a}, where {\tt m} is the type constructor. The
second, called {\tt bind} and stylized in the Haskell standard library as the
infix operator {\tt (>>=)}, takes a value of type {\tt m a} and a function
from {\tt a} to {\tt m b} and produces a value of type {\tt m b}. This allows
a value to be operated on out of the context and a new context to be assigned.

A monad can have other primitives which manipulate the context in some way. For 
instance, the primtives in Haskell's {\tt IO} monad produce actions which, when
interpreted as part of the {\tt main} action, produce some side-effect. The
{\tt State} monad provides {\tt get} and {\tt put} operations to work with a 
state value stored in the context.

Monad transformers~\cite{Jones1995} provide a means to combine the functionality
of multiple monads. A monad transformer is a monad with an extra type parameter.
This type parameter is instantiated with the type constructor of the underlying
monad, and an extra operation {\tt lift} is provided which converts values in
the underlying monad to values in the monad transformer.

The evaluation monad is a monad transformer. This permits it to be used in
conjunction with the {\tt IO} monad (or any other monad) to describe how input
is to be obtained for the signal function being evaluated, and how outputs are
to be handled.

The evaluation monad, in addition to the standard monad operators, provides a
means of {\em initializing} a signal function, and a means of translating the
monadic value describing evaluation to a value in the underlying monad. This
means, for instance, that we can obtain an action in the {\tt IO} monad to
evaluate a signal function.

The type of the evaluation monad must track the input type of the signal
function. The monad's context stores a mapping from outputs to handling actions.
An existential type can thus be used to ``hide'' the output type of the signal
function. However, inputs must come from external values, so the input type
cannot be hidden. There are thus three type parameters to the monad's type
constructor: the input signal vector, the type of the underlying monad, and the
monadic type parameter. The type is shown in Figure~\ref{figure:evaluation_monad_types}.

\begin{figure}
\begin{code}
-- A signal function's evaluations state
data SFEvalState svIn m
-- The evaluation monad
data SFEvalT svIn m a
\end{code}
\hrule
\caption{Evaluation monad types.}
\label{figure:evaluation_monad_types}
\end{figure}

\section{Combinators}
\label{section:System_Design_and_Interface-Combinators}

Signal functions are constructed from combinators, which are primitive signal
functions and operations to combine these primitives. These combinators are
grouped as basic signal functions, lifting operations for pure functions,
routing, reactivity, feedback, and time dependence.

\subsection{Basic Signal Functions}
\label{subsection:System_Design_and_Interface-Combinators-Basic_Signal_Functions}

The basic signal functions (Figure~\ref{figure:basic_signal_functions})
provide very simple operations. The {\tt identity} signal function, as expected,
simply copies its input to its output. The {\tt constant} signal function
produces the provided value as a signal at all times. The {\tt never} signal
function has an event output which never produces occurrences. The {\tt asap}
function produces an event occurrence with the given value at the first time
step after it is switched into the network. The {\tt after} function waits for
the specified amount of time before producing the event occurrence.

With the exception of {\tt identity}, all of the basic signal functions have
empty inputs. This allows these signal functions to be used to insert values
into the network which are known when the signal function is created, without
having to route those values from an input.

\begin{figure}
\begin{code}
identity :: SF NonInitialized sv sv
constant :: a -> SF NonInitialized SVEmpty (SVSignal a)
never    :: SF NonInitialized SVEmpty (SVEvent a)
asap     :: a -> SF NonInitialized SVEmpty (SVEvent a)
after    :: Double -> a -> SF NonInitialized SVEmpty (SVEvent a)
\end{code}
\hrule
\caption{Basic signal functions.}
\label{figure:basic_signal_functions}
\end{figure}

\subsection{Lifting Pure Functions}
\label{subsection:System_Design_and_Interface-Combinators-Lifting_Pure_Functions}

Two combinators are provided to lift pure functions to signal functions (Figure~\ref{figure:lifting_pure_functions}).
The {\tt pureSignal} combinator applies the pure function to a signal at every
sample point. The {\tt pureEvent} combinator applies the function to each
occurrence of an input event.

\begin{figure}
\begin{code}
pureSignal ::    (a -> b) 
              -> SF NonInitialized (SVSignal a) (SVSignal b)
pureEvent  ::    (a -> b)
              -> SF NonInitialized (SVSignal a) (SVSignal b)
\end{code}
\hrule
\caption{Lifting pure functions.}
\label{figure:lifting_pure_functions}
\end{figure}

\subsection{Routing}
\label{subsection:System_Design_and_Interface-Combinators-Routing}

\begin{figure}
\begin{code}
(>>>)       ::    SF NonInitialized svIn svBetween
               -> SF NonInitialized svBetween svOut
               -> SF NonInitialized svIn svOut
first       ::    SF NonInitialized svIn svOut
               -> SF NonInitialized (SVAppend svIn sv)
                                    (SVAppend svOut sv)
second      ::    SF NonInitialized svIn svOut
               -> SF NonInitialized (SVAppend sv svIn)
                                    (SVAppend sv svOut)
swap        ::    SF NonInitialized (SVAppend svLeft svRight) 
                                    (SVAppend svRight svLeft)
\end{code}
\hrule
\caption{Routing combinators.}
\label{figure:routing_combinators}
\end{figure}

\section{Evaluation}
\label{section:System_Design_and_Interface-Evaluator}

\subsection{Initialization and Running}

\subsection{Input and Output}

\subsection{Monad Lifting}
