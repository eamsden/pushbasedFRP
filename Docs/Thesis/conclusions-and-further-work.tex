\chapter{Conclusions and Further Work}
\label{chapter:Conclusions_and_Further_Work}
I have presented TimeFlies, a push-based signal-function FRP system. I have
demonstrated that TimeFlies does realize the theoretical benefits of a
push-based signal-function system.

\section{Conclusions}
\label{section:Conclusions_and_Further_Work-Conclusions}
The TimeFlies system is a fully-implemented and -documented FRP library which
may be extended with utility functions and further optimized. Its performance in
responding to events is demonstrated to be superior to that of the predominant
pull-based signal-function FRP library.

Further, the model of events used by TimeFlies subverts problematic semantic
questions about the evaluation of events in an FRP system. By using the N-Ary
FRP type model and separating the evaluation of events from the time steps used
for signals, TimeFlies fully supports non-deterministic merging of events,
and provides a semantic guarantee that events are not ''lost'' during evaluation.

TimeFlies includes a fully-documented evaluation interface which clarifies and
simplifies the task of integrating the TimeFlies system with the many IO systems
available to Haskell programmers.

\section{Further Work}
\label{section:Conclusions_and_Further_Work-Further_Work}
The TimeFlies system would benefit from attentive microbenchmarking and performance
tuning, as well as optimizations to avoid evaluating irrelevant parts of the network
during the the evaluation of time steps. A formal semantic justification for the
formulation of event evaluation (which would require a full formal semantics of FRP)
would enable a far more robust correctness argument, as well as providing a basis for
semantic extensions to signal-function FRP.

FRP is not yet mature, and has not been the subject of focused application development.
Thus, there is a dearth of design patterns for FRP applications. Such design patterns
would yield necessary feedback as to which generalizations and restrictions of FRP would
be appropriate and useful, and clarify the necessity of various utility combinators to
be included in the standard libraries of FRP systems.

In order to improve the performance of FRP yet further, it may be productive to attempt
to introduce parallel evaluation into FRP, taking advantage of the functional purity in
the implementation of the signal function combinators. This may involve, for instance,
evaluating several time-steps at once in a data-parallel manner,task parallelism
between different branches of a signal function, or speculative evaluation of switch
combinators.

Many classes of reactive application would benefit from a ``dynamic collections'' combinator
similar to {\tt pSwitch} in Yampa. Such a combinator allows a collection of signal functions
to be evaluated as one signal function, with addition or removal of signal functions instead
of the total replacement given by the {\tt switch} combinator. This is useful, for instance,
when simulating objects for games or computer models, as the behavior of each object can be
modeled as a signal function, and these signal functions can be added to and removed from the
collection.

The TimeFlies system provides a principled and performant system for future experimentation
on FRP, as well as implementation of FRP applications.
