\chapter{Haskell Concepts}
\label{chapter:Haskell_Concepts}

One of the primary attractions of the Haskell language, and the reason for its
use throughout this work, is its advanced yet practical and usable type system.
This type system enables the use of compositional software design that would
be rendered infeasible without a type system to both inform and verify
composition and implementation. This appendix gives an overview of Haskell
concepts, design patterns, and idioms which are used in this thesis.

\section{Datatypes and Pattern Matching}
\label{section:Haskell_Concepts-Datatypes_and_Pattern_Matching}

In Haskell, new type constructors are introduced by defining Algebraic Datatypes.
An ADT declaration can take one of two forms. The first is a {\tt data}
declaration, e.g.:

\begin{code}
data Bool = True | False
data Maybe a = Just a | Nothing
\end{code}

In this form the identifier(s) preceding the {\tt =} character are a type
constructor followed by zero or more type variables. Following the {\tt =} 
character, and separated by {\tt |} characters, are data constructors. Each data
constructor is an identifier followed by zero or more types, which are the types
of its arguments. Data constructors can be used in expressions to construct a
value of this newly-declared type constructor, and in pattern matching to
``take apart'' the value and observe its components (the arguments to the data
constructor.

The second form is the {\tt newtype} declaration. This form is more restricted.
It is limited to one data constructor with exactly one argument. It introduces
a new type without introducing a new runtime restriction, though the Haskell
code must still use the data constructor in pattern matches and expressions
to explicitly coerce between the new type and the type of its data constructor's
parameter. This behavior is most often used to hide implementation types without
introducing the runtime overhead of value construction and pattern-matching, as
these are erased for constructors declared using {\tt newtype} once type-checking
is complete.

Once a type constructor has been introduced, it can be used in a type, with
its arguments replaced by any valid Haskell type. For instance:

\begin{code}
not      :: Bool -> Bool
cbool    :: Bool -> Int
maybeInt :: Int -> Maybe Int -> Int
isJust   :: Maybe a -> Bool
mapMaybe :: (a -> b) -> Maybe a -> Maybe b
\end{code}

Its data constructors can be used in the patterns of functions, and in their
expressions. For instance:

\begin{code}
not :: Bool -> Bool
not True  = False
not False = True

isJust :: Maybe a -> Bool
isJust (Just _) = True
isJust Nothing  = False

mapMaybe :: (a -> b) -> Maybe a -> Maybe b
mapMaybe f (Just x) = Just (f x)
mapMaybe _ Nothing  = Nothing
\end{code}

\subsection{Generalized Algebraic Datatypes}
\label{subsection:Haskell_Concepts-Datatypes_and_Pattern_Matching-Generalized_Algebraic_Datatypes}

GADTs~\cite{Cheney2003,Xi2003} permit us to specify what types fill in the type
parameters for specific constructors. For instance, if we wish to build a tiny
expression language, we can use a standard ADT:

\begin{code}
data Exp a = Const a 
           | Plus (Exp a) (Exp a)
           | LessThan (Exp a) (Exp a)
           | If (Exp a) (Exp a) (Exp a)
\end{code}

Let us assume for the moment that Haskell exports functions\footnote{The types
for addition and comparison actually involve a typeclass constraint, but the
point is that the functions' types are not parametric.}:

\begin{code}
(+)  :: Int -> Int -> Int
(<)  :: Int -> Int -> Bool
\end{code}

An attempt to write an evaluation function for our expression type is:

\begin{code}
eval :: Exp a -> a
eval (Const x)  = x
eval (Plus x y) = eval x + eval y
eval (LessThan x y) = eval x < eval y
eval (If p c a) = if (eval p)
                  then eval c
                  else eval a
\end{code}

But this function will not typecheck. In the {\tt Plus} case, we cannot
assume that the argument to the type constructor {\tt Exp} typing {\tt x} or {\tt y}
is {\tt Int}, and similarly for the {\tt LessThan} case. Again in the {\tt If}
case, we cannot assume that the predicate is of type {\tt Exp Bool}, and if we
could, that would force our results to be of type {\tt Exp Bool} as well.

Let's try a slightly modified ADT:

\begin{code}
data Exp a = Const a
           | Plus (Exp Int) (Exp Int)
           | LessThan (Exp Int) (Exp Int)
           | IfThenElse (Exp Bool) (Exp a) (Exp a)
\end{code}

Here the motivation for a type parameter to our type constructor becomes clear.
We can introduce both {\tt Bool} and {\tt Int} constants (as well as others, but
we cannot do anything with them unless we extend the language). Further, we
can constrain the types of the input expressions to each of our constructors to
be of the appropriate type.

The code for our our evaluator is the same. But now note that the in the
{\tt Plus} and {\tt LessThan} cases, even though the input types are compatible
with the functions used, the output type expected from our function is not.
{\tt Plus x y} has type {\tt Exp a} in the pattern match, so our output is
expected to be of type {\tt a} for any {\tt a} argument to type of an 
input {\tt Exp}.

Here is our expression type as a GADT:
\begin{code}
data Exp a where
  Const    :: a                               -> Exp a
  Plus     :: Exp Int  -> Exp Int             -> Exp Int
  LessThan :: Exp Int  -> Exp Int             -> Exp Bool
  If       :: Exp Bool -> Exp a   -> Exp a    -> Exp a
\end{code}

Now our evaluation function can typecheck. Each constructor is able to constrain
the type parameter for output, not just its arguments. So when pattern matching
on the {\tt Plus} case, we know that each of our inputs will be of type
{\tt Exp Int}, and that the type of the expression we are pattern matching has
type {\tt Exp Int}, so the output from our function can be constrained to type
{\tt Int}, similarly for {\tt LessThan}. The type argument to {\tt Exp} in the
type of the {\tt If} is constrainted to be the same as that as the argument
to the types of the consequent and alternate to the conditional. This permits
our {\tt If} statement to be parametric while still allowing our evaluation to
typecheck.

This capacity to constrain the output types of data constructors, and thus, to
constrain the types of expressions in the scope of pattern matches of these data
constructors, is called {\em type refinement}. We will make use of this ability
to parameterize concrete datatypes over abstract type structures, rather than to
permit typechecking in specific cases, but the principle is the same.

\section{Typeclasses}
\label{section:Haskell_Concepts-Typeclasses}

Typeclasses in Haskell provide a means to implement functions that are openly
polymorphic while not being parametric. A typeclass is declared as follows:

\begin{code}
class Show t where
  show :: t -> String
\end{code}

A typeclass has an identifier and a single type parameter. This type parameter
is used in the type of one or more functions which are members of the class.

The class can then be instantiated:

\begin{code}
instance Show Bool where
  show True  = "True"
  show False = "False"

instance (Show a) => Show (Maybe a) where
  show (Just x) = "Just " ++ show x
  show Nothing  = Nothing
\end{code}

Functions can now be written polymorphically over the types instantiating the
typeclass, by including the typeclass as a constraint:

\begin{code}
repL :: (Show t) => t -> Int
repL x = length (show x)
\end{code}

Typeclasses are used in Haskell to provide common interfaces or functionality
across types. The {\tt Show} class used as an example is exported, along with
instances for most of the basic Haskell types, from Haskell's Prelude (the
standard module imported into every Haskell module). 

\section{Monads and Monad Transformers}
\label{section:Haskell_Concepts-Monads_and_Monad_Transformers}

One of the primary concepts employed in Haskell programs is that of the
monad~\cite{PeytonJones1993,PeytonJones2001}. The concept of the monad is
borrowed from category theory, but it is quite simple when understood within
Haskell. A monad is a type constructor with a single parameter, and two
associated functions. In Haskell's {\tt Monad} typeclass, these functions are
denoted {\tt return} and {tt (>>=)}.

\begin{code}
class Monad m where
  return :: a -> m a
  (>>=)  :: m a -> (a -> m b) -> m b
\end{code}

A monad must obey the following axioms:
\begin{itemize}
\item Left identity: {\tt return x >>= f = f x}
\item Right identity: {\tt m >>= return = m}
\item Associativity: {\tt (a >>= b) >>= c) = (a >>= (\\x -> b x >>= c))}
\end{itemize}

Several standard Haskell type constructor are monads in interesting ways, but
the most well-known is Haskell's {\tt IO} type constructor. This is the basis of
Haskell's input/output system. The entry point to a Haskell program is the
{\tt main} function, of type {\tt IO ()}. This function can be constructed by
using the monadic functions to sequence an arbitrary number of other functions
whose output type is {\tt IO a}. Since the sequencing operator takes an
arbitrary function, this allows the full power of Haskell functions, including
first-class and higher-order functions, to be employed in defining a program's
input and output. A convenience function is commonly used when the result is not
necessary as part of the sequencing:

\begin{code}
(>>) :: Monad m => m a -> m b -> m b
(>>) m1 m2 = m1 >>= (const m2)

\subsection{Do-notation}
\label{subsection:Haskell_Concepts-Monads_and_Monad_Transformers-Do_notation}

Because monads are such a pervasive concept in Haskell, the language includes
special syntax for writing monadic expressions. Do-notation is expression syntax
which begins with the keyword {\tt do} and is followed by lines of two forms:

\begin{code}
do
  x <- m1
  m2
\end{code}

The first form is a binding expression: it binds the variable {\tt x} to the
output of the monadic value {\tt m2}. The second form simply sequences the monad
value {\tt m2} into the monadic value being built. Do-notation has a syntax-driven
translation to desugared Haskell expression:

\begin{code}
desugar {
do x <- m1
   ...} = 
m1 >>= \x -> desugar {do ...}

desugar {
do m1
   ...} =
m1 >> desugar {do ...}
\end{code}

\subsection{Monad Transformers}
\label{subsection:Haskell_Concepts-Monads_and_Monad_Transformers-Monad_Transformers}
A monad transformer is a type constructor with two parameters. The first
parameter parameterizes over a one-parameter type constructor, rather than a
type. The second is the monadic type parameter. A type constructor {T} is a monad
transformer if it has the following instance of the Monad typeclass:

\begin{code}
instance Monad m => Monad T m
\end{code}

and is also an instance of the class

\begin{code}
class MonadTrans t where
  lift :: (Monad m) => m a -> t m a
\end{code}

The axioms for the lift function are:
\begin{itemize}
\item {\tt lift . return = return}
\item {\tt lift (m >>= f) = lift m >>= (lift . f)}
\end{itemize}

Restated, {\tt lift} does not modify return and distributes over monadic
sequencing.

As an example of a monad transformer, we can consider the {\tt StateT} type,
which is employed in the implementation of this thesis.

The type is declared:
\begin{code}
newtype StateT s m a = S { runStateT :: s -> m (s, a) }
\end{code}

Its monad instance, for any {\tt s}, is

\begin{code}
instance Monad m => Monad (StateT s m) where
  return x  = S (\s -> return (x, s))
  (>>=) (S f) mf = S (\s -> f s >>= (\ (x, s') -> let (S f') = mf x in f' s ))
\end{code}

The return and sequencing functions carry the state through the underlying monad.

The {\tt MonadTrans} instance is:
\begin{code}
instance MonadTrans (StateT s) where
  lift m = S (\s -> m >>= (\x -> return (x, s))
\end{code}

Finally, there are two functions provided to access and set the state:

\begin{code}
put :: s -> StateT s m ()
put = S (\_ -> return ((), s'))

get :: StateT s m s
get = S (\s -> return (s, s))
\end{code}

If we use {\tt StateT} as a wrapper around the IO monad, we might employ it as
a way to generate a unique line number for each "putStrLn" we call.

\begin{code}
putStrLnN :: String -> StateT Int IO ()
putStrLnN s = do i <- get
                 put (i + 1)
                 lift (putStrLn (show i ++ " " ++ s))

main = runStateT mainSt 1
  where mainSt = do g <- lift getLine
                    putStrLnN g
                    mainSt
\end{code}

\section{Glossary of Type Terminology}
\label{section:Haskell_Concepts-Glossary_of_Type_Terminology}
\begin{description}
\item[ADT] See {\em Algebraic Datatype}.

\item[Algebraic Datatype] An Algebraic Datatype or ADT is a type whose terms
are {\em data constructors}. An ADT is defined by naming a
{\em type constructor} and its parameters (zero or more)
(as {\em type variables}), together with one or more data constructors and the
types of their members. Each data constructor takes a fixed number
(zero or more) data members, whose types are given following the constructor
name. These types are defined in terms of the type variables named as parameters
of the type constructor and any type constructors (including the type
constructor associated with the ADT) in scope in the module.

\item[Data Constructor] A Data Constructor is a component of an
Algebraic~Datatype which, when applied to values of the appropriate type, 
creates a value typed with the ADT. Data constructors are the primary element
which may be pattern matched in languages such as Haskell.

\item[GADT] See {\em Generalized Algebraic Datatype}.

\item[Generalized Algebraic Datatype] Similar to an {\em Algebraic Datatype},
but {\em type variables} in the {\em type constructor} declaration serve merely
to denote the number of type parameters (and thus may be replaced by a
{\em kind signature}) and types are given for each {\em data constructor}. These
types must have the type constructor as their top-level term, but may fill in
the parameters of the type constructor with arbitrary types. Variables which
appear in the data member types but on in the data constructor type are
{\em existentially quantified}, and types appearing in the data constructor
type but not the data member types may be instantiated arbitrarily.

\item[Kind] A ``type of types.'' Kinds are used to verify that types are
consistent during typechecking. The kind of types which contain values is
{\tt *}, and the kind of single-parameter type constructors which take a
type is {\tt * -> *}. Other kinds may also be introduced. For instance,
signal vectors should be their own separate kind, but the Haskell type mechanism
was not mature enough to support this at the time of this writing..

\item[Kind Signature] A means of specifying the number and kind of types which
may instantiate type variables. Type variables in Haskell are not restricted to
types, but may be instantiated by type constructors as well. The kind of a
variable restricts what it may be instantiated with. A kind signature gives
kinds to a type constructor, and thus to its parameters. Specifying the kind
of a type constructor perfectly constrains the number of parameters.

\item[Type Constructor] A type constructor is a type level term which, when
applied to the proper number of types, produces a type. Type constructors,
together with {\em type variables}, form the basis of polymorphism in Haskell
and similar languages.

\item[Type Variable] A type variable is a type-level term which may be 
instantiated (by the typechecker via inference, or by the user via annotation)
with any type at the point where the value so typed is used. 
Together with {\em type constructors}, type variables form the basis of
polymorphism in Haskell and similar languages.
\end{description}
