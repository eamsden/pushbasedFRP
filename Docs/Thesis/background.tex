\chapter{Background}
\label{chapter:Background}

The key abstraction of functional programming is the function, which takes an input and produces an output. Functions may produce functions
as output and/or take functions as input. We say that functions are {\em first-class values} in a language.

Even in functional languages, reactive programs are generally written in an imperative style, using low-level and non-composable abstractions including callbacks
or object-based event handlers, called by event loops. This ties the model of interactivity to low-level implementation details such as timing and event handling models. 

{\em Functional} Reactive Programming implies that a model should keep the characteristics of functional programming (i.e. that the basic constructs of the language
should remain first-class) while incorporating reactivity into the language model. In particular, functions should be lifted to operate on reactive values,
and functions themselves ought to be reactive.

The goal of FRP is to provide compositional and high-level abstractions for creating reactive programs. The key
abstractions of FRP are behaviors or signals\footnote{Behaviors are generally referred to as {\em signals} in signal function literature. This is unfortunate, since a signal
function may operate on events, signals, or some combination of the two.}, which are time-dependent values defined at every point in continuous time, and events, which are 
values defined at countably many points in time. An FRP system will provide functions to manipulate events and signals and to react
to events by replacing a portion of the running program in response to an event. Behaviors and events, or some abstraction
based on them, will be first class, in keeping with the spirit of functional programming. Programs implemented in FRP languages
should of course be efficiently executable. This has proven to be the main challenge in implementing FRP.

The two general approaches to FRP are ``classic'' FRP, where behaviors and signals are first-class and reactive objects, and ``signal-function'' FRP,
where transformers of signals and events are first-class and reactive objects.

\section{Classic FRP}
\label{section:Background-Classic_FRP}

The earliest and still standard formulation of FRP provides two primitive type
constructors: {\tt Behavior} and {\tt Event}, together with combinators that
produce values of these types. The easiest semantic definition for these types
is given in Figure~\ref{figure:classic_frp_semantic_types}\footnote{The list constructor {\tt []} should be considered to contain infinite as well as finite lists.}.

\begin{figure}
\begin{code}
type Event a = [(Time, a)]
type Behavior a = Time -> a
\end{code}
\hrule
\caption{Semantic types for Classic FRP.}
\label{figure:classic_frp_semantic_types}
\end{figure}


When these two type constructors are exposed directly, the system is known as a 
{\em Classic FRP} system. If the type aliases are taken as given in the semantic
definition, a simple, yet problematic, implementation is given in
Figure~\ref{figure:classic_frp_simple_implementation}.

\begin{figure}
\begin{code}
time :: Behavior Time
time = id

constant :: a -> Behavior a
constant = const

delayB :: a -> Time -> Behavior a -> Behavior a
delayB a td b te = if te <= td
                   then a
                   else b (te - td)          

instance Functor Behavior where
  fmap = (.)

instance Applicative Behavior where
  pure = constant
  bf <*> ba = (\t -> bf t (ba t))

never :: Event a
never = []

once :: a -> Event a
once a = [(0, a)]

delayE :: Time -> Event a -> Event a
delayE td = map (\(t, a) -> (t + td, a))

instance Functor Event where
  fmap f = map (\(t, a) -> (t, f a))

switcher :: Behavior a -> Event (Behavior a) -> Behavior a
switcher b [] t = b t
switcher b [(to, bo)] t = if t < to
                          then b t
                          else bo t
\end{code}
\hrule
\caption{An obvious, yet inefficient and problematic, implementation of
Classic FRP.}
\label{figure:classic_frp_simple_implementation}
\end{figure}

Obvious problems include the lack of ordering enforcement on event occurrences,
the necessity of waiting for the first event occurrence in a switch, and the
necessity of maintaining the full history of a behavior for evaluation.

\subsection{History of Classic FRP}
\label{subsection:Background-Classic_FRP-History_of_Classic_FRP}
Classic FRP was originally described as the basis of Fran~\cite{Elliott1997}
(Functional Reactive ANimation), a framework for declaratively specifying
interactive animations. Fran represented behaviors as two sampling functions,
one from a time to a value and a new behavior (so that history may be discared),
and the other from a time interval (a lower and upper time bound) to a value
interval and a new behavior. Events are represented as ``improving values'',
which, when sampled with a time, produce a lower bound on the time to the next
occurrence, or the next occurrence if it has indeed occurred.

The first implementation of FRP outside the Haskell language was Frapp\'{e}, an
FRP implementation for the Java Beans framework. Frapp\'{e} built on the notion
of events and ``bound properties'' in the Beans framework, providing abstract
interfaces for FRP events and behaviors, and combinators as concrete classes
implementing these interfaces. The evaluation of Frapp\'{e} used Bean components
as sources and sinks, and the implementation of Bean events and bound properties
to propagate changes to the network~\cite{Courtney2001-2}. 

\subsection{Current Classic FRP Systems}
\label{subsection:Background-Classic_FRP-Current_Classic_FRP_Systems}

The FrTime\footnote{FrTime is available in the current version of Dr Racket (5.2.1).}~\cite{Cooper2006}
language extends the Scheme evaluator with a mutable dependency graph, which is
constructed by program evaluation. This graph is then updated by signal changes.
FrTime does not provide a distinct concept of events, and chooses branches of
the dependency graph by conditional evaluation of signals, rather than by the
substitution of signals used by FRP systems.

The Reactive~\cite{Elliott2009} system is a push-pull FRP system with
first-class behaviors and events. The primary insight of Reactive is the
separation of reactivity (or changes in response to events whose occurrence time
could not be known beforehand) from time-dependence. This gives rise to {\em
reactive normal form}, which represents a behavior as a constant or simple
time-varying value, together with an event stream carrying values which are also
behaviors in reactive normal form. Push-based evaluation is achieved by forking
a Haskell thread to evaluate the head behavior, while waiting on the evaluation
of the event stream. Upon an event occurrence, the current behavior thread is
killed and a new thread spawned to evaluate the new behavior. Unfortunately, the
implmentation of Reactive uses a tenuous technique which depends on also forking
threads to evaluate two haskell values concurrently in order to implement event
merging. This relies on the library author to ensure consistency when this
technique is used, and leads to thread leakage when one of the merged events is
itself the merging of other events.

A recent thesis~\cite{Czaplicki2012} described Elm, a stand-alone language for
reactivity. Elm provides combinators for manipulating discrete events, and
compiles to Javascript, making it useful for client-side web programming.
However, Elm does not provide a notion of switching or of continuous time
behaviors, though an approximation is given using discrete-time events which are
actuated at repeated intervals specified during the event definition. The thesis
asserts that Arrowized FRP (signal-function FRP, Section~\ref{section:Background-signal_function_frp})
can be embedded in Elm, but provides no support for this assertion.

The reactive-banana~\cite{Apfelmus} library is a push-based FRP system designed
for use with Haskell GUI frameworks. In particular, it features a monad for
the creation of behaviors and events which may then be composed and evaluated.
This monad includes constructs for binding GUI library constructs to primitive
events. It must be ``compiled'' to a Haskell IO action for evaluation to take
place. The implementation of reactive-banana is similar to FrTime, using a
dependency graph to update the network on event occurences. Reactive-banana
eschews generalized switching in favor of branching functions on behavior
values, similarly to FrTime, but maintains the distinction between behaviors and
events. Rather than a generalized switching combinator which allows the
replacement of arbitrary behaviors, reactive-banana provides a step combinator
which creates a stepwise behavior from the values of an event stream.

\section{Signal Function FRP}
\label{section:Background-signal_function_frp}

An alternative approach to FRP was first proposed in work on
Fruit~\cite{Courtney2001-1}, a library for declarative specification of GUIs.
This library utilized the abstraction of Arrows~\cite{Hughes2000} to structure
{\em signal functions}. Arrows are abstract type constructors with input and
output type parameters, together with a set of routing combinators (see Figure~\ref{figure:arrow_combinators}).
Signal functions are the first-class
abstraction in this approach, they represent time-dependent and reactive
transformers of events and signals,  which are themselves not first class,
since such values cannot be directly manipulated by the programmer. This
approach has two motivations: it increases modularity since both the input and
output of signal functions may be transformed, as opposed to signals or events
which may only be transformed in their output, and it avoids a large class of
time and space leaks which have emerged when implementing FRP with first-class
signals and events.

\begin{figure}
\begin{code}
(>>>)  :: (Arrow a) => a b c -> a c d -> a b d
arr    :: (Arrow a) => (b -> c) -> a b c
first  :: (Arrow a) => a b c -> a (b, d) (c, d)
second :: (Arrow a) => a b c -> a (d, b) (d, c)
(***)  :: (Arrow a) => a b c -> a b d -> a b (c, d)
loop   :: (Arrow a) => a (b, d) (c, d) => a b c
\end{code}
\hrule
\caption{Arrow combinators.}
\label{figure:arrow_combinators}
\end{figure}

Similarly to FrTime, the netwire~\cite{Soylemez} library eschews dynamic
switching, in this case in favor of {\em signal inhibition}. Netwire is written
as an arrow transformer, which, together with the Kliesli arrow instance for the
IO monad\footnote{Any monad forms a Kliesli arrow.}, permits it to lift IO
actions as sources and sinks at arbitrary points in a signal function network.
Signal inhibition is accomplished by making the output of signal functions a
monoid, and then combining the outputs of signal functions. An inhibited signal
function will produce the monoid's zero as an output. Primitives have defined
inhibition behavior, and composed signal functions inhibit if their outputs
combine to the monoid's zero.

Yampa~\cite{Nilsson2005} is an optimization of the Arrowized FRP system first
utilized with Fruit (see above). The implementation of Yampa makes use of
Generalized Algebraic Datatypes to permit a much larger class of type-safe
datatypes for the signal function representation. This representation,
together with ``smart'' constructors and combinators, enables the construction
of a self-optimizing arrowized FRP system. Unfortunately, the primary
inefficiency, that of unnecessary evaluation steps due to pull-based evaluation,
remains. Further, the optimization is ad-hoc and each new optimization requires
the addition of new constructors, as well as the updating of every primitive
combinator to handle every combination of constructors. However, Yampa showed
noticeable performance gains over previous Arrowized FRP implementations.

A recent PhD thesis~\cite{Sculthorpe2011} introduced N-Ary FRP, a technique for
typing Arrowized FRP systems using dependent types. The bulk of the work
consisted in using the dependent type system to prove the correctness of the FRP
system discussed. This work introduced the typing construct of signal vectors,
which permit the distinction of signal and event types at the level of the FRP
system, instead of making events merely a special type of signal.

\section{Outstanding Challenges}
\label{section:Background-outstanding_challenges}

At present, there are two key issues apparent with FRP. First, while
signal-function FRP is inherently safer and more modular than classic FRP, it
has yet to be efficiently implemented. Classic FRP programs are vulnerable to
time leaks and violations of causality due to the ability to directly manipulate
reactive values. Second, the interface between FRP programs and the many
different sources of inputs and sinks for outputs available to a modern
application writer remains ad-hoc and is in most cases limited by the
implementation.

One key exception to this is the reactive-banana system, which provides a monad
for constructing primitive events and behaviors from which an FRP program may
then be constructed. However, this approach is still inflexible as it requires
library support for the system which the FRP program will interact with.
Further, being a Classic FRP system, reactive-banana lacks the ability to
transform the inputs of behaviors and events, since all inputs are implicit.
