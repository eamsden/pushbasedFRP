\chapter{Background}
\label{chapter:Background}

The key abstraction of functional programming is the function, which takes an input and produces an output. Functions may produces functions
as output and/or take functions as input. We say that functions are {\em first-class values} in a language.

Most reactive programs are written in an imperative style, using low-level and non-composable abstractions including callbacks
or object-based event handlers, called by event loops. This ties the model of interactivity to low-level implementation details such as timing and event handling models. 

FRP implies that a model should keep the characteristics of functional programming (i.e. that the basic constructs of the language
should remain first-class) while incorporating reactivity into the language model. In particular, functions should be lifted to operate on reactive values,
and functions themselves ought to be reactive.

The goal of FRP is to provide compositional and high-level abstractions for creating reactive programs. The key
abstractions of FRP are behaviors or signals\footnote{Behaviors are generally referred to as {\em signals} in signal function literature. This is unfortunate, since a signal
function may operate on events, signals, or some combination of the two.}, which are time-dependent values defined at every point in continuous time, and events, which are 
values defined at countably many points in time. An FRP system will provide functions to manipulate events and signals and to react
to events by replacing a portion of the running program in response to an event. Behaviors and events, or some abstraction
based on them, will be first class, in keeping with the spirit of functional programming. Programs implemented in functional reactive
models should of course be efficiently executable. This has proven to be the main challenge in implementing Functional Reactive Programming.

The two general approaches to FRP are ``classic'' FRP, where behaviors and signals are first-class and reactive objects, and ``signal-function'' FRP,
where transformers of signals and events are first-class and reactive objects.

\section{Classic FRP}
\label{section:Background-classic_frp}

The first FRP system, Fran~\cite{Elliott1997}, was introduced as a system for compositionally describing interactive animations.
Though the term ``Functional Reactive Programming'' was not used in this work, it was the first to introduce the concept of first-class
behaviors and events as an abstract model for reactive programs. The key motivation was to separate the implementation of reactivity
from the modelling of reactivity, so that a programmer could be freed from details such as sampling time-varying values, time-slicing
to simulate updating values in parallel, and sampling of inputs.

The first implementation of FRP outside the Haskell language was Frapp\'{e}, an FRP implementation for the Java Beans framework. Frapp\'{e} built on
the notion of events and {\em bound properties} in the Beans framework, providing abstract interfaces for FRP events and behaviors, and combinators
as concrete classes implementing these interfaces. The evaluation of Frapp\'{e} used Bean components as sources and sinks, and the implementation of
Bean events and bound properties to propagate changes to the network~\cite{Courtney2001-2}.


The FrTime\footnote{FrTime is available in the current version of Dr Racket (5.2.1).}~\cite{Cooper2006} language extends the Scheme evaluator with a mutable
dependency graph, which is constructed by program evaluation. This graph is then updated by signal changes. FrTime does not provide a distinct concept of
events, and chooses branches of the dependency graph by conditional evaluation of signals, rather than by the substitution of signals used by FRP systems.

The Reactive~\cite{Elliott2009} system is a push-based FRP system with first-class behaviors and events. The primary insight of Reactive is
the separation of reactivity (or changes in response to events whose occurrence time could not be know beforehand) from time-dependence. This
gives rise to {\em reactive normal form}, which represents a behavior as a constant or simple time-varying value, with an event stream carrying values
which are also behaviors in reactive normal form. Push-based evaluation is achieved by forking a Haskell thread to evaluate the head behavior,
while waiting on the evaluation of the event stream. Upon an event occurrence, the current behavior thread is killed and a new thread
spawned to evaluate the new behavior. Unfortunately, the implmentation of Reactive uses a tenuous technique which depends on also forking threads to evaluate
two haskell values concurrently in order to implement event merging. This relies on the library author to ensure consistency when this technique is
used, and leads to thread leakage when one of the merged events is itself the merging of other events.

The reactive-banana~\cite{Apfelmus} library is a push-based FRP system designed for use with Haskell GUI frameworks. In particular, it features a monad for
the creation of behaviors and events which may then be composed and evaluated. This monad includes constructs for binding GUI library constructs to primitive events.
It must be ``compiled'' to a Haskell IO action for evaluation to take place. The implementation of reactive-banana is similar to FrTime, using a dependency graph to update the network on event occurences. Reactive-banana eschews generalized switching in favor of branching functions on behavior values, similarly to FrTime, but
maintains the distinction between behaviors and events. Rather than a generalized switching combinator which allows the replacement of arbitrary behaviors,
reactive-banana provides a step combinator which creates a stepwise behavior from the values of an event stream.

A recent thesis~\cite{Czaplicki2012} described Elm, a stand-alone language for reactivity. Elm provides combinators for manipulating discrete events, and
compiles to Javascript, making it useful for client-side web programming. However, Elm does not provide a notion of switching or of continuous time behaviors,
though an approximation is given using discrete-time events which are actuated at repeated intervals specified during the event definition. The thesis asserts
that Arrowized FRP (signal-function FRP, Section~\ref{section:Background-signal_function_frp}) can be embedded in Elm, but provides no support for this assertion.

\section{Signal Function FRP}
\label{section:Background-signal_function_frp}

An alternative approach to FRP was first proposed in work on Fruit~\cite{Courtney2001-1}, a library for declarative specification of GUIs. This library
utilized the abstraction of Arrows~\cite{Hughes2000} to structure {\em signal functions}. Arrows are abstract type constructors with input and output type
parameters, together with sequential ({\tt >>>}) and parallel ({\tt first} and {\tt second}) composition, branching ({\tt ***}), and lifting ({\tt arr}) functions
to the Arrow type. Signal functions are the first-class abstraction in this approach, they represent time-dependent and reactive transformers of events and signals, 
which are themselves not first class, since such values cannot be directly manipulated by the programmer.
This approach has two motivations: it increases modularity since both the input and output of signal functions may be transformed,
as opposed to signals or events which may only be transformed in their output, and it avoids a large class of time and space leaks which have emerged when
implementing FRP with first-class signals and events.

Similarly to FrTime, the netwire~\cite{Soylemez} library eschews dynamic switching, in this case in favor of {\em signal inhibition}. Netwire is written as an arrow
transformer, which, together with the Kliesli arrow instance for the IO monad\footnote{Any monad forms a Kliesli arrow.}, permits it to lift IO actions as sources and
sinks at arbitrary points in a signal function network. Signal inhibition is accomplished by making the output of signal functions a monoid, and then combining the
outputs of signal functions. An inhibited signal function will produce the monoid's zero as an output. Primitives have defined inhibition behavior, and composed signal
functions inhibit if their outputs combine to the monoid's zero.

Yampa~\cite{Nilsson2005} is an optimization of the Arrowized FRP system first utilized with Fruit (see above). The implementation of Yampa makes use of Generalized Algebraic Datatypes to permit a much larger class of type-safe datatypes for the signal function representation. This representation, together with ``smart''
constructors and combinators, enables the construction of a self-optimizing arrowized FRP system. Unfortunately, the primary inefficiency, that of unnecessary evaluation
steps due to pull-based evaluation, remains. Further, the optimization is ad-hoc and each new optimization requires the addition of new constructors, as well
as the updating of every primitive combinator to handle every combination of constructors. However, Yampa showed noticeable performance gains over previous Arrowized FRP implementations.

A recent PhD thesis~\cite{Sculthorpe2011} introduced N-Ary FRP, a technique for typing Arrowized FRP systems using dependent types. The bulk of the work consisted
in using the dependent type system to prove the correctness of the FRP system discussed. This work introduced the typing construct of signal vectors,
which permit the distinction of signal and event types at the level of the FRP system, instead of making events merely a special type of signal.

\section{Outstanding Challenges}
\label{section:Background-outstanding_challenges}

At present, there are two key issues apparent with FRP. First, while signal-function FRP is inherently safer and more modular than classic FRP, it has yet to be
efficiently implemented. Second, the interface between FRP programs and the many different sources of inputs and sinks for outputs available to a modern application
writer remains ad-hoc and is in most cases limited by the implementation. One key exception to this is the reactive-banana system, which provides a monad for
constructing primitive events and behaviors from which an FRP program may then be constructed. However, this approach is still inflexible as it requires library support
for the system which the FRP program will interact with. Further, classic FRP programs are vulnerable to time leaks and violations of causality due to the ability
to directly manipulate reactive values.
