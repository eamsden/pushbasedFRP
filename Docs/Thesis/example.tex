\chapter{Example Application}
\label{chapter:Example_Application}

TimeFlies is a library for Functional Reactive Programming, which is a paradigm
for creating interactive and time-dependent applications. This chapter presents
the design of one such application, and its implementation using TimeFlies. The
application is an OpenFlow controller which implements a learning switch. In
short, it is a re-implementation of the standard kind of switch used in local
area networks, using ``software-defined networking.'' This
is the application which is benchmarked for performance comparisons in
Chapter~\ref{chapter:Evaluation_and_Comparisons}.

\section{OpenFlow}
\label{section:Example_Application-OpenFlow}

The OpenFlow protocol~\cite{OpenflowSpec} is a protocol for software-defined
networking applications. In particular, it defines the communication between
switches (devices which quickly route packets from input ports to output ports
based on learned rules) and {\em controllers}, which are generally devices with
large computational resources, such as servers. OpenFlow allows switches to
report packets for which no rule exists to a controller, and provides a means
for the controller to install new rules on a switch either preemptively, or in
response to a reported packet.

One of the simplests tasks which may be implemented as a OpenFlow controller is
a ``learning switch''. Such a switch uses the source and destination hardware
addresses in network packets to make routing decisions. A table is kept which
records the ports where source addresses are observed on incoming packets. This
table thus contains knowledge of which hardware address(es) can be reached on
which port. Using this table, rules are constructed to route packets with
particular source-destination pairs to the correct port. When a packet is seen
for which no rule exists, it is reported to the controller, which updates the
table and rules based on the new knowledge, and broadcasts the packet so that
it can reach its intended destination and receive a response.

Our example application is a controller for a ``learning switch.'' We describe
its implementation as a TimeFlies signal function, and two approaches to running
this signal function using the TimeFlies evaluation interface.

\section{Implementation}
\label{section:Example_Application-Implementation}

The first component for our learning switch is the table which maps addresses
to ports. This is a stateful data structure which will be updated by input
events and possibly produce output events. We employ the {\tt accumulateList}
combinator to produce the signal function:

\begin{figure}
\begin{code}
-- | Function type to modify a table and produce messages to the switch
type TableAccumulator = 
     SwitchTable 
  -> ([(SwitchHandle EthernetFrame,
        TransactionID,
        CSMessage)],
      SwitchTable)

-- | Empty map
M.empty :: M.Map k v

-- | Reverse application
rapp :: a -> (a -> b) -> b
rapp x f = f x

-- | Accumulate a switch table, producing output messages
--   based on the incoming functions
switchTable ::    SVEvent TableAccumulator
              :~> SVEvent (SwitchHandle EthernetFrame,
                           TransactionID,
                           CSMessage)
switchTable = accumulateList rapp M.empty
\end{code}
\hrule
\caption{Signal function for switch table.}
\label{figure:switch_table_sf}
\end{figure}

Note that the input events are closures which expect the table as input and
produce an updated table as output. This enables us to write several different
event sources whose final events require state (the table) and produce
messages to the switch.

The first source of such events is packets forwarded to the controller by
switches. The closures carried by these events implement the response to
packet inputs. The signal function from input switch message events to events
from the accumulator is shown in Figure~\ref{figure:packet_in_sf}, but the
details of the switch routing algorithm are elided.

\begin{figure}
\begin{code}
getPacketIn ::     SVEvent (SwitchHandle EthernetFrame,
                            TransactionID,
                            SCMessage EthernetFrame)
               :~> SVEvent (SwitchHandle EthernetFrame,
                            TransactionID,
                            PacketInfo EthernetFrame)

-- | Handle a packet in event
handlePacketIn ::    ((SwitchHandle EthernetFrame,
                       TransactionID,
                       PacketInfo EthernetFrame), Double)
                  -> TableAccumulator

packetIn ::     SVEvent ((SwitchHandle EthernetFrame,
                          TransactionID,
                          PacketInfo EthernetFrame),
                         Double)
            :~> SVEvent TableAccumulator
packetIn = pureEventTransformer handlePacketIn

-- | Capture the occurrence time of an event.
captureTime :: SVEvent a :~> SVEvent (a, Double)
captureTime = uncancelLeft >>>
              first time >>>
              capture

-- | Build TableAccumulator events from input packets.
handleSCMessage ::     SVEvent (SwitchHandle EthernetFrame,
                                TransactionID,
                                SCMessage EthernetFrame)
                   :~> SVEvent TableAccumulator
handleSCMessage = getPacketIn >>> captureTime >>> packetIn

\end{code}
\hrule
\caption{Signal function handling incoming packets.}
\label{figure:packet_in_sf}
\end{figure}

The {\tt captureTime} signal function attaches the time to incoming packets. The
{\tt getPacketIn} signal function extracts packet-in messages from the variety
of message events that a switch may send to the controller. These two signal
functions feed into the {\tt packetIn} signal function, which builds the
closures for the accumulator using the {\tt handlePacketIn} pure function.

The other task is to periodically eliminate all rules which are older than a
specified threshold. For this purpose, we construct another signal function
which produces {\tt TableAccumulator} events. This signal function runs a timer
(the {\tt every} combinator) and produces events carrying closures which scan
the table for old rules, delete them, and produces messages to inform the
switches of their deletion. This signal function is shown in Figure~\ref{figure:clean_sf},
again eliding the implementation of the closures.

\begin{figure}
\begin{code}
-- | Take a function-value pair and return the application.
pairApp :: (a -> b, a) -> b
pairApp (f, x) = f x

-- | Alter a table and generate messages to remove expired entries and rules
cleanTable :: Double -> Double -> TableAccumulator


-- | Produce an event with the rule-cleaning accumulator.
cleanRules :: Double -> SVEmpty :~> SVEvent TableAccumulator
cleanRules t = copy >>>
               second (every t cleanRulesAcc) >>>
               first time >>>
               capture >>>
               pureEventTransformer pairApp
  where 
    cleanRulesAcc :: Double -> TableAccumulator
    cleanRulesAcc nowT table = cleanTable t nowT table
\end{code}
\hrule
\caption{Table-cleaning signal function.}
\label{figure:clean_sf}
\end{figure}

We now have all of the pieces for a signal function implementing a learning
switch, as shown in Figure~\ref{figure:learn_sf}.

\begin{figure}
\begin{code}
-- | A constant, for how long rules may survive
tableTime :: Double

-- | Make IO actions to send each packet.
sendCSMessage ::     SVEvent (SwitchHandle EthernetFrame,
                              TransactionID,
                              CSMessage)
                 :~> SVEvent (IO ())

-- | The learning switch
learn ::     SVEvent (SwitchHandle EthernetFrame,
                      TransactionID,
                      SCMessage EthernetFrame)
         :~> SVEvent (IO ())
learn = uncancelLeft >>> 
        first (cleanRules tableTime) >>>
        second handleSCMessage >>>
        union >>>
        switchTable >>>
        sendCSMessage
\end{code}
\hrule
\caption{The signal function for a learning switch.}
\label{figure:learn_sf}
\end{figure}

The learning switch composes the {\tt cleanRules} and {\tt handleSCMessage}
signal functions in parallel, but routes input only to {\tt handleSCMessage},
by using the {\tt uncancelLeft} routing combinator. The event streams are
merged and fed to the {\tt switchTable} signal function, and the resulting
packets are fed to the {\tt sendCSMessage} signal function which constructs
{\tt IO ()} events.

Note that this approach enables a modular design where each task is implemented
in its own signal function. Additional behaviors could be coded as additional
signal functions, whose events were also routed to the table accumulator.

Finally, we show how the signal function is interfaced with IO code using the
evaluation interface. First, we need a few extra definitions. For starters,
a controller must also handle switches when they first connect. We create a
separate signal function for this purpose, and compose it in parallel with our
learning switch signal function. We also need a definition of which port to
listen on, and an IO action for reading the time. These are shown in
Figure~\ref{figure:example_io_prelim}.

\begin{figure}
\begin{code}

-- | The port for the server to listen on.
port :: Word16

-- | Get the current time from the monotonic timer
getMonoTimeAsDouble :: IO Double

-- | Handle the connection of a new switch          
switchConnect ::     SVEvent (SwitchHandle EthernetFrame)
                 :~> SVEvent (IO ())
switchConnect = pureEventTransformer (void . handshake)

-- | Run switch connection and learning concurrently
learningSwitch ::     (SVEvent (SwitchHandle EthernetFrame) :^:
                       SVEvent (SwitchHandle EthernetFrame,
                                TransactionID,
                                SCMessage EthernetFrame))
                  :~> SVEvent (IO ())
learningSwitch = first switchConnect >>> second learn >>> union
\end{code}
\hrule
\caption{Evaluation preliminaries.}
\label{figure:example_io_prelim}
\end{figure}

Now we can create the main procedure for our learning switch. Note that
the code for placing an event on the input of a signal function is a simple
monadic sequence, sandwich between {\tt IORef} reads and writes of an opaque
value (Figure~\ref{figure:example_io}).

\begin{figure}
\begin{code}
-- | Entry point
main :: IO ()
main = do (argS:_) <- getArgs
          let sampleTime = read argS
          time <- getMonoTimeAsDouble
          reactiveRef <- newIORef $ initSFEval 
                                      (eventHandler id)
                                      (combineSamples sampleEvt sampleEvt)
                                      time
                                      learningSwitch
          ofEventManager <- 
            openFlowEventManager
              Nothing
              port 
              -- Switch addition inputs
              (\handle -> do rState <- readIORef reactiveRef
                             ((), rState') <- runSFEvalT 
                                                (push $ svLeft $ svOcc handle)
                                                rState
                             writeIORef reactiveRef rState')
              -- Switch message inputs:
              (\((tid, msg), handle) -> 
                 do rState <- readIORef reactiveRef
                              ((), rState') <- runSFEvalT 
                                                 (push $ svRight $
                                                  svOcc (handle, tid, msg))
                                                 rState
                              writeIORef reactiveRef rState')
          let evtMgr = getEventManager ofEventManager
              -- Sampling:
              sample = void $ registerTimeout evtMgr sampleTime $ 
                         do rState <- readIORef reactiveRef
                            time <- getMonoTimeAsDouble
                            ((), rState') <- runSFEvalT (step time) rState
                            writeIORef reactiveRef rState'
                            sample
          sample 
          loop $ evtMgr      
\end{code}
\hrule
\caption{Using the evaluation interface for IO.}
\label{figure:example_io}
\end{figure}
