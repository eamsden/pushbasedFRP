Functional Reactive Programming (FRP) is a promising class of abstractions for
interactive programs. FRP systems provide values defined at all points in time
({\em behaviors} or {\em signals}) and values defined at countably many points
in time ({\em events}) as abstractions.

Signal-function FRP is a subclass of FRP which does not provide direct access to
time-varying values to the programmer, but instead provides signal functions,
which are reactive transformers of signals and events, as first-class objects in
the program. 

All signal-function implementations of FRP to date have utilized demand-driven
or ``pull-based''  evaluation for both events and signals, producing output from
the FRP system whenever the consumer of the output is ready. This greatly
simplifies the implementation of signal-function FRP systems, but leads to
inefficient and wasteful evaluation of the FRP system when this strategy is
employed to evaluate events, because the components of the signal function which
process events must be computed whether or not there is an event occurrence.

In contrast, an input-driven or ``push-based'' system evaluates the network
whenever new input is available. This frees the system from evaluating the
network when nothing has changed, and then only the components necessary to
react to the input are re-evaluated. This form of evaluation has been applied to
events in standard FRP systems~\cite{Elliott2009} but not in signal-function FRP
systems.

I describe the design and implementation of a signal-function FRP system which
applies pull-based evaluation to signals and push-based evaluation to events (a
``push-pull'' system). The semantics of the system are discussed, and its
performance and expressiveness for practical examples of interactive programs
are compared to existing signal-function FRP systems through the implementation
of a networking application.
