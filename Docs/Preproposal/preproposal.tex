\documentclass[11pt]{artikel3}
\usepackage{fullpage, setspace, graphicx}
\usepackage[margin=1in]{geometry}
\usepackage{times}
\usepackage{url}

\title{RIT Department of Computer Science\\MSc Project/Thesis Pre-Proposal:\\\emph{Implementation of a Push Based Signal Function FRP System and Evaluation Interface}}
\author{Edward Amsden}
\date{\today}

\begin{document}
\maketitle

The sections shown below are adapted from the topic analysis forms provided in ``Writing the Doctoral Dissertation" (2nd edition) by Davis and Parker (pages 82-88). Your final document should be 1-2 pages including references. {\bf The final pre-proposal may present the items below in any format, but using prose (not bulleted lists).} 

\section{Problem Description}
The signal function family of FRP semantics provides distinct advantages over the alternative signal/behavior family. These advantages include explicit inputs and the elimination of large classes of space and time leaks. However, there does not yet exist an implementation that provides "push-based" semantics (i.e. evaluation on event occurrences, rather than polling for events) for signal functions. 

Most of the work in Functional Reactive Programming centers around the construction and evaluation of FRP programs, and does not concern itself with the interface between FRP programs and the rest of the world. In fact, most literature on FRP ignores completely the problem of evaluation interface for FRP programs. 

Do a monadic evaluation interface and signal vector typing enable a useful push-based FRP system? 

\section{Importance of Research}
Functional reactive programming enables a denotative method of defining reactive systems. The efficient evaluation
of such systems is still an open problem. Signal function style FRP semantics have several advantages over
classic FRP semantics, but current implementations use ``pull'' based evaluation, which has the undesirable consequences of
repeated unnecessary work in evaluating event non-occurrences, forcing input to be synchronous with output, and introducing sampling latency into event occurrences.

A signal function FRP system which uses ``pull'' based evaluation would eliminate these consequences, permitting more
efficient implementation with only the latency of processing, rather than the sampling interval. It would also permit
the decoupling of the input and output of the signal function network.

\section{Related Work}

FRP is a well-explored field. There is a wealth of research on the best semantics and most efficient implementation of FRP systems. I previously surveyed the literature regarding FRP for independent study credit~\cite{Amsden2011}. The proposed research would provide an efficient implementation of an elegant family of FRP semantics, and a compositional and intuitive means of providing input to and using output from such systems.

Signal function FRP or Arrowized FRP (AFRP) is one of two main families of semantics for Functional Reactive 
Programming~\cite{Nilsson2002}. It was recently extended with additional typechecked properties using the notion of signal 
vectors. 

Push based FRP has been demonstrated for classic FRP semantics~\cite{Elliott2009}, but it has not yet been explored in the context of signal function semantics.

\section{Methodology}
\label{sec:methodology}

I will explore an implementation which permits the separation of evaluation of continuous values and events. The implementation will hold the current signal function network as state in the monadic computation described using the evaluation monad. Event occurrences provided as input or produced within the network will mutate the network state. 

The evaluation monad will provide monadic actions for ``pushing'' an event occurrence on an input to the signal function network, stepping time and evaluating continous values, and sampling continuous output values.

Signal function networks will be described by a GADT using HLists~\cite{Kiselyov2004} and phantom types (Event $\alpha$ and Signal $\alpha$) to describe signal vectors. 

Possibilities to be explored for implementation include ``callbacks'' to be activated on output event occurrences, or on time
steps for continuous output, and a means of addressing multiple inputs or outputs to a signal function network.

\section{Potential Outcomes}

I expect to produce an implementation of signal function semantics that permits push-based evaluation. I expect that this implementation will include both constructs for producing a signal function network and a monadic interface for evaluating a signal function network. The constructs will include routing combinators, switching combinators (which will introduce information denoting where mutations to the network by events will take place), and stateful signal functions such as
numeric integration.

It is possible that this interface will also permit multiple inputs to and outputs from a signal function network.

\bibliography{preproposal}{}
\bibliographystyle{acm}

\end{document}
