Functional Reactive Programming (FRP) is a promising class of abstractions for encoding interactive and time-dependent programs in
functional languages, but has proven difficult to implement efficiently. Signal-function FRP is a subclass of FRP which does not provide
direct access to the time-varying  values by the programmer, but rather provides {\em signal functions} as first class objects.
All signal-function implementations of FRP to date have utilized demand-driven or ``pull-based''  evaluation, producing output from
the FRP system whenever the consumer of the output is ready. This greatly simplifies the implementation of signal-function FRP systems, but
leads to inefficient and wasteful evaluation of the FRP system.

In contrast, an input-driven or ``push-based'' system evaluates the network whenever new input is available. This frees the system from
evaluating the network when nothing has changed, and then only components necessary to react to the input are reevaluated.
This form of evaluation has been partially applied to standard FRP systems~\cite{Elliott2009} but not to signal-function FRP systems.

I describe ongoing work toward the implementation of a push-based signal-function FRP system. Informal semantics
and the current method of implementation are described, and challenges and possible solutions for a completed implementation
are discussed.
