\section{Introduction}
\label{section:Introduction}

Most of the useful programs which programmers are asked to write must react to inputs which are not available at the time the program starts,
and produce effects at many different times throughout the execution of the program. Examples of these programs include GUI applications,
web browsers, web applications, video games, multimedia authoring and playback tools, operating system shells and kernels, servers,
robotic control programs, and many others. This ability is called reactivity.

Functional reactive programming (FRP) is a promising class of abstractions for writing reactive programs in functional languages. FRP 
abstractions translate desirable properties of the underlying functional languages, such as higher-order functions and referential
transparency, to reactive constructs, generally without modifications to the underlying language. This permits the compositional
construction of reactive programs in purely functional languages.

Functional reactive programming was introduced, though not quite by name, with Fran~\cite{Elliott1997}, a compositional system for writing
reactive animations. From the start, two key challenges emerged: the enforcement of {\em causality}\footnote{Causality is the property 
that a value in an FRP program depends only on present or past values. Obviously, a program which violates this property is impossible to
evaluate, but in some FRP systems such programs are unfortunately quite easy to write down.} in FRP programs, and the efficient
evaluation of FRP programs.

The first challenge was addressed by representing FRP programs, not as compositions of signals and events, but as compositions of
{\em transformers} of behaviors and events called {\em signal functions}. By not permitting direct access to behaviors or events, but
representing them implicitly, signal functions prohibit accessing past or future time values, only values in the closure of the signal function
and current input values. Signal function FRP programs are written by composing signal functions, and are evaluated using a function which provides input to the
signal function and acts on the output values. This function is elided in existing literature on signal functions. 
A further advantage of signal function FRP programs is that they are more compositional, since additional signal functions may be composed
on the input or output side of a signal function, rather than simply the output side in classic (non-signal-function) FRP systems.

The second challenge was addressed for classic FRP by the creation of a hybrid push-pull FRP system~\cite{Elliott2009}. This system relied
on the distinction between reactivity and time dependence to decompose behaviors into phases of constant or simply time-dependent values,
with new phases initiated by event occurrences. However, this implementation did not address concerns of causality or compositionality.
Further, the existence of events as first-class values in classic FRP forced the use of an impure and resource-leaking technique to compare
events to determine which had the first occurrence after the current time. This technique made use of the Haskell {\tt unsafePerformIO} function
to fork two threads to compare the improving values of the next occurrence time for two events being merged. Further, since the classic FRP
interface permits access to only the output of a behavior or event, and is always bound to its implicit inputs, the best the push-based implementation
could do is to block when no computation needs to be performed. The computation cannot be suspended entirely as a value representation.

To address both challenges in a single FRP implementation, it seems desirable to combine the approaches, creating a push-based
signal-function FRP system. The current implementation approach for signal-function FRP is inherently pull-based~\cite{Nilsson2002}, but
recent work on N-ary FRP, a variant of signal-function FRP which enumerates inputs and distinguishes between events and behaviors\footnote{Behaviors are generally
referred to as {\em signals} in signal function literature. This is unfortunate, since a signal function may operate on events, signals, or some combination of the
two.}, provides a representation which suggests a method of push-based evaluation. In previous implementations of FRP, events were simply
option-valued signals. This approach has the advantage of being simple to implement in a pull-based setting, but does not have an obvious
semantics for event merging, and necessitates pull-based evaluation, since there is no way to separate the events which may be pushed from
the truly continuous values which must be repeatedly sampled.

N-ary FRP provides a model of a type system for a push-based signal-function FRP system. However, in order to represent signal functions
as continuations on collections of inputs rather than on single values, we represent the signal vectors as {\em phantom types}~\cite{Cheney03} over which
the signal function type is parameterized for both inputs and outputs. From these phantom types, we can type indices into the signal vector, which
contain the actual input or output values and form elements for a collection of inputs or outputs. Obviously the collection need not have all possible elements,
which forms the basis for evaluating only those parts of the network which depend on the updated values.