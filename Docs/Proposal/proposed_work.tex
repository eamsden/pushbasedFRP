\section{Proposed Work}
\label{section:ProposedWork}

Recent work on types for signal-function FRP systems has yielded N-ary FRP~\cite{Sculthorpe2011}, which types signal function inputs and outputs
with {\em signal vectors} rather than types of values. Signal vectors are phantom types: signal or event type constructor taking a value type,
or a product type of signal vectors. This permits individual components of a signal function's input or output to be designated as an event
or as a signal, and provides precisely the flexibility necessary to create a push-based implementation.

When signal functions are typed using value representations, where events are simply option values, the entire input and output must be constructed
for every step of a signal function. In particular, there is no obvious mechanism for producing signal values which contain only the updated components
of the signal, together with actually occurring events. 

By making use of phantom types which designate components of a signal vector, together with Generalized Algebraic Datatypes (GADTs) which
permit the specification of the input and output types of a signal function and type-safe indexing of the input and output components,
individual components of the input of a signal function may be processed as they become available, and may produce individual components of
the output.

I will create a library of combinators for creating signal-function FRP systems which may be evaluated when input is provided, and which
is typed with phantom types describing signal vectors~(Figure~\ref{fig:proposed_work_example_signalvectors}). These combinators will produce continuations
which accept values together with signal vector indexes and produce replacement continuations together with collections of signal vector indexed values
~(Figure~\ref{fig:proposed_work_example_signalfunctions}). 
The indexes allow combinators to use only the continuations necessary to process a specific input, as opposed to requiring the entire input
to be constructed for a signal function. In particular, sequential composition will detect output changes or event
occurrences from the left-side signal function and only evaluate the right-side signal function as necessary, parallel composition ({\tt first} and {\tt second})
will only evaluate their left or right side signal function depending on the index, and routing combinators will produce outputs for a given index
in the output signal vector only when there is an input presented on the corresponding index in the input signal vector. 

\begin{figure}
\begin{Verbatim}[commandchars=\\\{\}]
\PY{c+c1}{-- Signal Vectors}
\PY{k+kr}{data} \PY{k+kt}{SVEmpty}
\PY{k+kr}{data} \PY{k+kt}{SVSignal} \PY{o+ow}{::} \PY{o}{*} \PY{o+ow}{-\PYZgt{}} \PY{o}{*}
\PY{k+kr}{data} \PY{k+kt}{SVEvent}  \PY{o+ow}{::} \PY{o}{*} \PY{o+ow}{-\PYZgt{}} \PY{o}{*}
\PY{k+kr}{data} \PY{k+kt}{SVAppend} \PY{o+ow}{::} \PY{o}{*} \PY{o+ow}{-\PYZgt{}} \PY{o}{*} \PY{o+ow}{-\PYZgt{}} \PY{o}{*}

\PY{c+c1}{-- Signal Vector indexes}
\PY{k+kr}{data} \PY{k+kt}{SVIndex} \PY{o+ow}{::} \PY{p}{(}\PY{o}{*} \PY{o+ow}{-\PYZgt{}} \PY{o}{*}\PY{p}{)} \PY{o+ow}{-\PYZgt{}} \PY{o}{*} \PY{o+ow}{-\PYZgt{}} \PY{o}{*} \PY{k+kr}{where}
  \PY{k+kt}{SVISignal} \PY{o+ow}{::} \PY{n}{a} \PY{o+ow}{-\PYZgt{}} \PY{k+kt}{SVIndex} \PY{n}{p} \PY{p}{(}\PY{k+kt}{SVSignal} \PY{n}{p} \PY{n}{a}\PY{p}{)}
  \PY{k+kt}{SVIEvent}  \PY{o+ow}{::} \PY{n}{a} \PY{o+ow}{-\PYZgt{}} \PY{k+kt}{SVIndex} \PY{n}{p} \PY{p}{(}\PY{k+kt}{SVEvent} \PY{n}{p} \PY{n}{a}\PY{p}{)}
  \PY{k+kt}{SVILeft}   \PY{o+ow}{::} \PY{k+kt}{SVIndex} \PY{n}{p} \PY{n}{svl} \PY{o+ow}{-\PYZgt{}} \PY{k+kt}{SVIndex} \PY{n}{p} \PY{p}{(}\PY{k+kt}{SVAppend} \PY{n}{svl} \PY{n}{svr}\PY{p}{)}
  \PY{k+kt}{SVIRight}  \PY{o+ow}{::} \PY{k+kt}{SVIndex} \PY{n}{p} \PY{n}{svr} \PY{o+ow}{-\PYZgt{}} \PY{k+kt}{SVIndex} \PY{n}{p} \PY{p}{(}\PY{k+kt}{SVAppend} \PY{n}{svl} \PY{n}{svr}\PY{p}{)}
\end{Verbatim}

\hrule
\caption{Examples data types for push-based signal functions}
\label{fig:proposed_work_example_signalvectors}
\end{figure}

\begin{figure}
\begin{Verbatim}[commandchars=\\\{\}]
\PY{c+c1}{-- "Identity" type constructor}
\PY{k+kr}{newtype} \PY{k+kt}{Id} \PY{n}{a} \PY{o+ow}{=} \PY{k+kt}{Id} \PY{n}{a}

\PY{c+c1}{-- Signal Function (initialized) datatype}
\PY{k+kr}{data} \PY{k+kt}{SFInitialized} \PY{o+ow}{::} \PY{o}{*} \PY{o+ow}{-\PYZgt{}} \PY{o}{*} \PY{o+ow}{-\PYZgt{}} \PY{o}{*} \PY{k+kr}{where}
  \PY{k+kt}{SFInitialized} \PY{o+ow}{::} \PY{p}{(}\PY{k+kt}{DTime} \PY{o+ow}{-\PYZgt{}} \PY{p}{(}\PY{p}{[}\PY{k+kt}{SVIndex} \PY{k+kt}{Id} \PY{n}{svOut}\PY{p}{]}\PY{p}{,} \PY{k+kt}{SFInitialized} \PY{n}{svIn} \PY{n}{svOut}\PY{p}{)}\PY{p}{)} \PY{o+ow}{-\PYZgt{}} 
                   \PY{p}{(}\PY{k+kt}{SVIndex} \PY{k+kt}{Id} \PY{n}{svIn} \PY{o+ow}{-\PYZgt{}} \PY{p}{(}\PY{p}{[}\PY{k+kt}{SVIndex} \PY{k+kt}{Id} \PY{n}{svOut}\PY{p}{]}\PY{p}{,} \PY{k+kt}{SFInitialized} \PY{n}{svIn} \PY{n}{svOut}\PY{p}{)}\PY{p}{)} \PY{o+ow}{-\PYZgt{}}
                   \PY{k+kt}{SFInitialized} \PY{n}{svIn} \PY{n}{svOut}
\end{Verbatim}

\hrule
\caption{Example data type for push-based signal functions}
\label{fig:proposed_work_example_signalfunctions}
\end{figure}

Further, I will create an evaluation interface for signal functions created with this library, which will provide monadic actions to push
events into the signal function, activating predefined callbacks on output events (Figure \ref{fig:proposed_work_example_evalmonad}). Signals
will not be permitted on the inputs or outputs of evaluated signal functions, but only internally to the signal function.

\begin{figure}
\begin{Verbatim}[commandchars=\\\{\}]
\PY{c+c1}{-- SF Evaluation monad transformer}

\PY{k+kr}{data} \PY{k+kt}{SFEvalMonadT} \PY{o+ow}{::} \PY{o}{*} \PY{o+ow}{-\PYZgt{}} \PY{o}{*} \PY{o+ow}{-\PYZgt{}} \PY{o}{*} \PY{o+ow}{-\PYZgt{}} \PY{o}{*} \PY{o+ow}{-\PYZgt{}} \PY{o}{*}

\PY{k+kr}{instance} \PY{k+kt}{Monad} \PY{p}{(}\PY{k+kt}{SFEvalMonadT} \PY{n}{svIn} \PY{n}{svOut} \PY{n}{m}\PY{p}{)}

\PY{k+kr}{instance} \PY{k+kt}{SFMonadTransformer} \PY{p}{(}\PY{k+kt}{SFEvalMonadT} \PY{n}{svIn} \PY{n}{svOut}\PY{p}{)}

\PY{n+nf}{initSF} \PY{o+ow}{::} \PY{k+kt}{SF} \PY{n}{svIn} \PY{n}{svOut} \PY{o+ow}{-\PYZgt{}} \PY{k+kt}{SFState} \PY{n}{svIn} \PY{n}{svOut}

\PY{n+nf}{runSFEvalMonadT} \PY{o+ow}{::} \PY{p}{(}\PY{k+kt}{EventOnly} \PY{n}{svIn}\PY{p}{,} \PY{k+kt}{EventOnly} \PY{n}{svOut}\PY{p}{,} \PY{k+kt}{Monad} \PY{n}{m}\PY{p}{)} \PY{o+ow}{=\PYZgt{}} 
                   \PY{k+kt}{SFState} \PY{n}{svIn} \PY{n}{svOut} \PY{o+ow}{-\PYZgt{}}
                   \PY{p}{[}\PY{k+kt}{SVIndex} \PY{p}{(}\PY{o+ow}{-\PYZgt{}} \PY{n}{m} \PY{n+nb}{()}\PY{p}{)} \PY{n}{svOut}\PY{p}{]} \PY{o+ow}{-\PYZgt{}}
                   \PY{k+kt}{SFEvalMonadT} \PY{n}{svIn} \PY{n}{svOut} \PY{n}{m} \PY{n}{a} \PY{o+ow}{-\PYZgt{}}
                   \PY{n}{m} \PY{p}{(}\PY{n}{a}\PY{p}{,} \PY{k+kt}{SFState} \PY{n}{svIn} \PY{n}{svOut}\PY{p}{)}

\PY{n+nf}{push} \PY{o+ow}{::} \PY{p}{(}\PY{k+kt}{EventOnly} \PY{n}{svIn}\PY{p}{,} \PY{k+kt}{EventOnly} \PY{n}{svOut}\PY{p}{)} \PY{o+ow}{=\PYZgt{}}
		\PY{k+kt}{SVIndex} \PY{k+kt}{Id} \PY{n}{svIn} \PY{o+ow}{-\PYZgt{}}
		\PY{k+kt}{SFEvalMonadT} \PY{n}{svIn} \PY{n}{svOut} \PY{n}{m} \PY{n+nb}{()}
\end{Verbatim}

\hrule
\caption{Example functions and monadic combinators for evaluation of push-based signal functions}
\label{fig:proposed_work_example_evalmonad}
\end{figure}

I will provide a formal semantic description for the combinator library and evaluation interface which will permit me to argue that the
new implementation solves a particularly thorny problem with ``merging'' events.

This implementation will be compared to the existing pull-based signal-function implementation, Yampa, by implementing a simple
2-dimensional game with both Yampa and the new implementation, combining both with OpenGL~\cite{Akeley2011} and GLUT~\cite{Kilgard96} for graphics. CPU usage, framerate,
and response latency will be measured for both systems and comparisons given. In  addition, I expect the proposed implementation to be usable with
GLUT and other event-loop based libraries, or even combinations of event-based libraries, simply and without an additional thread for the
FRP system, something not possible with the present evaluation function in Yampa or with current classic FRP systems.

\subsection{Timeline}
\label{subsection:ProposedWork:Timeline}

\begin{tabular}{rp{34em}}
 \withRightVertBar{April 2, 2012}  & Submit draft proposal with proposed work, timeline, and introduction for review by Professor Fluet. \\
                                   & \\
 \withRightVertBar{April 23, 2012} & Complete initial implementation of FRP library, begin debugging and testing. \\
                                   & \\
 \withRightVertBar{April 30, 2012} & Finish proposal including background. Submit final proposal to \newline
                                     Professor Fluet, Professor Nunes-Harwitt and Professor Butler to\newline
                                     convene committee. \\
                                   &\\
 \withRightVertBar{May 31, 2012}   & Submit ongoing work writeup to Haskell Symposium. Include semantics and an interactive demonstration.\\
                                   & \\
 \withRightVertBar{June 11, 2012}  & Complete demonstration application (2D graphical game) in new FRP library and in Yampa.\\
                                   & \\
 \withRightVertBar{June 18, 2012}  & Complete testing and comparisons of demonstration application.\\
                                   & \\
 \withRightVertBar{June 25, 2012}  & Thesis draft submission.\\
                                   & \\
 \withRightVertBar{July 8, 2012}   & Thesis defense.
\end{tabular}